\section{Linear Operators}
\subsection{Continuous Operators}
\begin{definition}
	Let $(X,\norm{\cdot}_X)$ and $(Y,\norm{\cdot}_Y)$ be two normed spaces. An \bld{operator} is a linear mapping $T : X \to Y$. Moreover, we say that an operator $T : X \to Y$ is \bld{bounded} if there exists $c > 0$ such that 
	\begin{equation}
		\norm{T(x)}_Y \leq c\norm{x}_X
	\end{equation}
	\noindent holds for all $x \in X$.
\end{definition}

\subsection{The Hahn-Banach Theorem}

\begin{lemma}
	Let $V$ be a real vector space, $S \subsetneq V$ a linear subspace, $p : V \to \Rbb$ a sublinear functional, $f : S \to \Rbb$ linear and $x_0 \in V \setminus S$. Moreover, assume that $f \leq p$ on $S$. Then there exists $F : S + \Rbb x_0 \to \Rbb$ linear such that $F \leq p$ on $S + \Rbb x_0$ and $F\vert_S = f$.
\end{lemma}

\begin{theorem}[Hahn-Banach, $\Rbb$]
	Let $V$ be a vector space over $\Rbb$, $S \subseteq V$ a linear subspace and $f : S \to \Rbb$ linear. Moreover, let $p : V \to \Rbb$ be a sublinear functional such that $f \leq p$ on $S$. Then there exists $F : V \to \Rbb$ linear such that $F \leq p$ on $V$ and $F\vert_S = f$.	
\end{theorem}

\begin{theorem}[Hahn-Banach, $\mathbb{R}$ or $\mathbb{C}$]
	Let $V$ be a vector space over $\mathbb{K}$, $q : V \to \Rbb$ a seminorm, $S \subseteq V$ a linear subspace and $f : S \to \mathbb{K}$ linear with $\abs{f} \leq q$ on $S$. Then there exists $F : V \to \mathbb{K}$ linear with $F \vert_S = f$ and $\abs{F} \leq q$ on $V$.		
\end{theorem}

\begin{corollary}[Extension]
	Let $(X,\norm{\cdot})$ be a normed space over $\mathbb{K}$, $S \subseteq X$ a linear subspace and $f \in S^*$. Then there exists $F \in X^*$ such that $F\vert_S$ = f and $\norm{F}_{X^*} = \norm{f}_{S^*}$.
\end{corollary}

\begin{corollary}[Separation]
	Let $(X,\norm{\cdot})$ be a normed space over $\mathbb{K}$ and $x_0 \in X \setminus \cbr{0}$. Then there exists $f \in X^*$ with $\norm{f} = 1$ and $f(x_0) = \norm{x_0}$.
\end{corollary}

\subsection{Reflexivity}

\begin{proposition}
	Let $X$ be a normed vector space over $\mathbb{K}$. Then the mapping $\Phi : X \to X^{**}$ defined by $\Phi(x) := \varphi_x$, where $\varphi_x : X^* \to \mathbb{R}$ is defined by $\varphi_x(f) := f(x)$, is a linear isometry.
\end{proposition}

\begin{theorem}
	Let $X$ be a Banach space. Then $X$ is reflexive if and only if $X^*$ is reflexive.
\end{theorem}

\subsection{Hilbert Space Methods}

\begin{theorem}[Riesz's Representation Theorem]
	Let $(H,\inprod{\cdot}{\cdot})$ be a Hilbert space over $\mathbb{K}$. The mapping $\Psi : H \to H^*$ defined by $\del[1]{\Psi(x)}(y) := \inprod{x}{y}$ is an anti-linear isometric isomorphism.
\end{theorem}

\begin{corollary}
	Every Hilbert space is reflexive.
\end{corollary}

\begin{theorem}[Lax-Milgram]
	Let $H$ be a Hilbert space over $\mathbb{K}$ and let $a : H \times H \to \mathbb{K}$ be a sesquilinear form. Moreover, suppose that there are constants $0 < c_0 \leq C_0 < \infty$ such that
	\begin{center}
		\begin{tabular}{ll}
			$\abs{a(x,y)} \leq C_0 \norm{x}\norm{y}$ & (\textbf{Continuity}),\\
			$\Re a(x,x) \geq c_0\norm{x}^2$ & (\textbf{Coercivity}),
		\end{tabular}
	\end{center}
	\noindent for all $x,y \in H$. Then there exists a unique $A \in \mathcal{L}(H)$ such that 
	\begin{equation}
		a(x,y) = \inprod{Ax}{y}
	\end{equation}
	\noindent for all $x,y \in H$. Moreover, $A$ is invertible with
	\begin{equation}
		\norm{A} \leq C_0 \qquad \text{and} \qquad \norm[0]{A^{-1}} \leq \frac{1}{c_0}.	
	\end{equation}
\end{theorem}
