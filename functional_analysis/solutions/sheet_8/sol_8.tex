%%%%%%%%%%%%%%%%%%%%%%%%%%%%%%%%%%%%%%%%%%%%%%%%%%%%%%%%%%%%%%%%%%%%%%%%%%
%Author:																 %
%-------																 %
%Yannis Baehni at University of Zurich									 %
%baehni.yannis@uzh.ch													 %
%																		 %
%Version log:															 %
%------------															 %
%06/02/16 . Basic structure												 %
%04/08/16 . Layout changes including section, contents, abstract.		 %
%%%%%%%%%%%%%%%%%%%%%%%%%%%%%%%%%%%%%%%%%%%%%%%%%%%%%%%%%%%%%%%%%%%%%%%%%%

%Page Setup
\documentclass[
	12pt, 
	oneside, 
	a4paper,
	reqno,
	final
]{amsart}

\usepackage[
	left = 3cm, 
	right = 3cm, 
	top = 3cm, 
	bottom = 3cm
]{geometry}

%Headers and footers
\usepackage{fancyhdr}
	\pagestyle{fancy}
	%Clear fields
	\fancyhf{}
	%Header right
	\fancyhead[R]{
		\footnotesize
		Yannis B\"{a}hni\\
		\href{mailto:yannis.baehni@uzh.ch}{yannis.baehni@uzh.ch}
	}
	%Header left
	\fancyhead[L]{
		\footnotesize
		401-3581-67L Symplectic Geometry\\
		Autumn 2017
	}
	%Page numbering in footer
	\fancyfoot[C]{\thepage}
	%Separation line header and footer
	\renewcommand{\headrulewidth}{0.4pt}
	%\renewcommand{\footrulewidth}{0.4pt}
	
	\setlength{\headheight}{19pt} 

%Title
\usepackage[foot]{amsaddr}
\usepackage{upref}
\usepackage{newtxtext}
\usepackage[subscriptcorrection,nofontinfo,mtpcal,mtphrb]{mtpro2}
\usepackage{bm}
\usepackage{xspace}
\makeatletter
\usepackage{etoolbox}
\patchcmd{\abstract}{\scshape\abstractname}{\textbf{\abstractname}}{}{}

\usepackage[all,cmtip]{xy}

%Section, subsection and subsubsection font
%------------------------------------------
\makeatletter
	\renewcommand{\@secnumfont}{\bfseries}
	\renewcommand\section{\@startsection{section}{1}%
  	\z@{.7\linespacing\@plus\linespacing}{.5\linespacing}%
  	{\normalfont\bfseries\centering}}
	\renewcommand\subsection{\@startsection{subsection}{2}%
    	\z@{.5\linespacing\@plus.7\linespacing}{-.5em}%
    	{\normalfont\bfseries}}%
	\renewcommand\subsubsection{\@startsection{subsubsection}{3}%
    	\z@{.5\linespacing\@plus.7\linespacing}{-.5em}%
    	{\normalfont\bfseries}}%
%Formatting title of TOC
\renewcommand{\contentsnamefont}{\bfseries}
%Table of Contents
\setcounter{tocdepth}{3}

% Add bold to \section titles in ToC and remove . after numbers
\renewcommand{\tocsection}[3]{%
  \indentlabel{\@ifnotempty{#2}{\bfseries\ignorespaces#1 #2\quad}}\bfseries#3}
% Remove . after numbers in \subsection
\renewcommand{\tocsubsection}[3]{%
  \indentlabel{\@ifnotempty{#2}{\ignorespaces#1 #2\quad}}#3}
\let\tocsubsubsection\tocsubsection% Update for \subsubsection
%...

\newcommand\@dotsep{4.5}
\def\@tocline#1#2#3#4#5#6#7{\relax
  \ifnum #1>\c@tocdepth % then omit
  \else
    \par \addpenalty\@secpenalty\addvspace{#2}%
    \begingroup \hyphenpenalty\@M
    \@ifempty{#4}{%
      \@tempdima\csname r@tocindent\number#1\endcsname\relax
    }{%
      \@tempdima#4\relax
    }%
    \parindent\z@ \leftskip#3\relax \advance\leftskip\@tempdima\relax
    \rightskip\@pnumwidth plus1em \parfillskip-\@pnumwidth
    #5\leavevmode\hskip-\@tempdima{#6}\nobreak
    \leaders\hbox{$\m@th\mkern \@dotsep mu\hbox{.}\mkern \@dotsep mu$}\hfill
    \nobreak
    \hbox to\@pnumwidth{\@tocpagenum{\ifnum#1=1\bfseries\fi#7}}\par% <-- \bfseries for \section page
    \nobreak
    \endgroup
  \fi}
\AtBeginDocument{%
\expandafter\renewcommand\csname r@tocindent0\endcsname{0pt}
}
\def\l@subsection{\@tocline{2}{0pt}{2.5pc}{5pc}{}}
\def\l@subsubsection{\@tocline{2}{0pt}{4.5pc}{5pc}{}}
\makeatother

\advance\footskip0.4cm
\textheight=54pc    %a4paper
\textheight=50.5pc %letterpaper
\advance\textheight-0.4cm
\calclayout

%Font settings
%\usepackage{anyfontsize}
%Footnote settings
%\usepackage{mathptmx}
\usepackage{footmisc}
%	\renewcommand*{\thefootnote}{\fnsymbol{footnote}}
\usepackage{commath}
%Further math environments
%Further math fonts (loads amsfonts implicitely)
%Redefinition of \text
%\usepackage{amstext}
\usepackage{upref}
%Graphics
%\usepackage{graphicx}
%\usepackage{caption}
%\usepackage{subcaption}
%Frames
\usepackage{mdframed}
\allowdisplaybreaks
%\usepackage{interval}
\newcommand{\toup}{%
  \mathrel{\nonscript\mkern-1.2mu\mkern1.2mu{\uparrow}}%
}
\newcommand{\todown}{%
  \mathrel{\nonscript\mkern-1.2mu\mkern1.2mu{\downarrow}}%
}
\AtBeginDocument{\renewcommand*\d{\mathop{}\!\mathrm{d}}}
\renewcommand{\Re}{\operatorname{Re}}
\renewcommand{\Im}{\operatorname{Im}}
\DeclareMathOperator\Log{Log}
\DeclareMathOperator\Arg{Arg}
\DeclareMathOperator\sech{sech}
\DeclareMathOperator*\esssup{ess.sup}
\DeclareMathOperator\id{id}
\DeclareMathOperator\im{im}
\DeclareMathOperator\Vol{Vol}
\DeclareMathOperator\dist{dist}
%\usepackage{hhline}
%\usepackage{booktabs} 
%\usepackage{array}
%\usepackage{xfrac} 
%\everymath{\displaystyle}
%Enumerate
\usepackage{tikz}
\usetikzlibrary{external}
\tikzexternalize % activate!
\usetikzlibrary{patterns}
\pgfdeclarepatternformonly{adjusted lines}{\pgfqpoint{-1pt}{-1pt}}{\pgfqpoint{40pt}{40pt}}{\pgfqpoint{39pt}{39pt}}%
{
  \pgfsetlinewidth{.8pt}
  \pgfpathmoveto{\pgfqpoint{0pt}{0pt}}
  \pgfpathlineto{\pgfqpoint{39.1pt}{39.1pt}}
  \pgfusepath{stroke}
}
\usepackage{enumitem} 
%\renewcommand{\labelitemi}{$\bullet$}
%\renewcommand{\labelitemii}{$\ast$}
%\renewcommand{\labelitemiii}{$\cdot$}
%\renewcommand{\labelitemiv}{$\circ$}
%Colors
%\usepackage{color}
%\usepackage[cmtip, all]{xy}
%Theorems
\newtheoremstyle{main} 		             	 		%Stylename
  	{}	                                     		%Space above
  	{}	                                    		%Space below
  	{\itshape}			                     		%Body font
  	{}        	                             		%Indent
  	{\bfseries}   	                         		%Head font
  	{.}            	                        		%Head punctuation
  	{ }           	                         		%Head space 
  	{\thmname{#1}\thmnumber{ #2}\thmnote{ (#3)}}	%Head specification
\theoremstyle{main}
\newtheorem{definition}{Definition}[section]
\newtheorem{proposition}{Proposition}[section]
\newtheorem{corollary}{Corollary}[section]
\newtheorem{theorem}{Theorem}[section]
\newtheorem{lemma}{Lemma}[section]
%Roman style theorems
\newtheoremstyle{roman}
	{}
	{}
  	{}
  	{}
	{\bfseries}
	{.}
  	{ }
	{\thmname{#1}\thmnumber{ #2}\thmnote{ (#3)}}
\theoremstyle{roman}
\newtheorem{example}{Example}[section]
\newtheorem{solution}{Solution}[section]
\newtheorem{remark}{Remark}[section]
%Exercise style theorems
\newtheoremstyle{exercise}
  	{}
  	{}
  	{\small}
  	{}
  	{\bfseries}
  	{.}
 	{ }
  	{\thmname{#1}\thmnumber{ #2}\thmnote{ (#3)}}
\theoremstyle{exercise}
\newtheorem{exercise}{Exercise}[section]
%Changing default style of proof environment
\renewcommand*{\proofname}{\itshape Proof}
%German non-ASCII-Characters
%Graphics-Tool
%\usepackage{tikz}
%\usepackage{tikzscale}
%\usepackage{bbm}
%\usepackage{bera}
%Listing-Setup
%Bibliographie
\usepackage[backend=bibtex, style=alphabetic]{biblatex}
%\usepackage[babel, german = swiss]{csquotes}
\bibliography{../latex/bibliography}
%PDF-Linking
%\usepackage[hyphens]{url}
\usepackage[bookmarksopen=true,bookmarksnumbered=true]{hyperref}
%\PassOptionsToPackage{hyphens}{url}\usepackage{hyperref}
\hypersetup{
  colorlinks   = true, %Colours links instead of ugly boxes
  urlcolor     = blue, %Colour for external hyperlinks
  linkcolor    = blue, %Colour of internal links
  citecolor    = blue %Colour of citations
}
%Weierstrass-P symbol for power set
\newcommand{\powerset}{\raisebox{.15\baselineskip}{\Large\ensuremath{\wp}}}
\newcommand{\bld}[1]{\boldmath\textit{\textbf{#1}}\unboldmath}
\usepackage{pict2e}
\makeatletter
\DeclareRobustCommand{\intprod}{%
	\mathbin{\mathpalette\int@prod{(0.1,0)(0.9,0)(0.9,0.8)}}%
}
\newcommand{\int@prod}[2]{%
	\begingroup
	\sbox\z@{$\m@th#1+$}%
	\setlength\unitlength{\wd\z@}%
	\begin{picture}(1,1)
	\roundcap
	\polyline#2
	\end{picture}%
	\endgroup
}
\makeatother
\newcommand{\Sbb}{\mathbb{S}}
\newcommand{\dRrm}{\mathrm{dR}}
\newcommand{\Rbb}{\mathbb{R}}
\newcommand{\Lcal}{\mathcal{L}}
\newcommand{\Zbb}{\mathbb{Z}}
\newcommand{\Nbb}{\mathbb{N}}
\newcommand{\Xfrak}{\mathfrak{X}}


\title{Solutions Sheet 8}
\author{Yannis B\"{a}hni}
\address[Yannis B\"{a}hni]{University of Zurich, R\"{a}mistrasse 71, 8006 Zurich}
\email[Yannis B\"{a}hni]{\href{mailto:yannis.baehni@uzh.ch}{\nolinkurl{yannis.baehni@uzh.ch}}}

\begin{document}

\maketitle
\thispagestyle{fancy}

\setcounter{section}{1}

\begin{enumerate}[label = \textbf{Exercise \arabic*.},wide = 0pt, itemsep = 1.5ex]
	\item
	\item We will show that there exists a unique solution to the integral equation $f \in L^2(X)$. 
		\begin{lemma}
			Let $L \in L^2(X \times X)$ and $f \in L^2(X)$. For $x \in X$ define
			\begin{equation*}
				g_f(x) := \int_X L(x,y)f(y)dy.
			\end{equation*}
			Then $g_f \in L^2(X)$.
			\label{lem:L^2}
		\end{lemma}

		\begin{proof}
			We have that
			\begin{align*}
				\norm[0]{g_f}_{L^2(X)}^2 &= \int_X \abs[0]{g_f(x)}^2 dx\\
				&= \int_X \abs[3]{\int_X L(x,y)f(y)dy}^2 dx\\
				&\leq \int_X \del[3]{\int_X \abs{L(x,y)f(y)}dy}^2 dx\\
				&= \int_X \norm[0]{L(x,\cdot)f}_{L^1(X)}^2 dx\\
			&\leq \int_X \norm[0]{L(x,\cdot)}_{L^2(X)}^2 \norm[0]{f}_{L^2(X)}^2 dx\\
			&= \norm[0]{f}_{L^2(X)}^2 \int_X \int_X \abs[0]{L(x,y)}^2dy dx\\
			&= \norm[0]{f}_{L^2(X)}^2 \norm[0]{L}_{L^2(X \times X)}^2
			\end{align*}
		\noindent by H\"older and Fubini.
		\end{proof}
		Now we have to solve the equation
		\begin{equation*}
			cf + u = g_f.
		\end{equation*}
		Define $a : L^2(X) \times L^2(X) \to \mathbb{C}$ by
		\begin{equation*}
			a(f,\varphi) := \langle cf - g_f, \varphi \rangle_{L^2(X)}.
		\end{equation*}
		If $a$ satisfies the assumptions of \emph{Lax-Milgram}, we find $A \in \mathcal{L}\del[1]{L^2(X)}$, such that 
		\begin{equation*}
			a(f,\varphi) = \langle A(f),\varphi \rangle_{L^2(X)}
		\end{equation*}
		\noindent holds for all $f,\varphi \in L^2(X)$. Hence
		\begin{equation*}
			  \langle cf - g_f, \varphi \rangle_{L^2(X)} = \langle A(f),\varphi \rangle_{L^2(X)}
		\end{equation*}
		\noindent holds for all $f,\varphi \in L^2(X)$. Moreover, since $A$ is invertible, we find a unique $f_0 \in L^2(X)$, such that $A(f_0) = -u$. Therefore
		\begin{equation*}
			R_{L^2(X)}(cf_0 - g_{f_0})(\varphi) = \langle cf_0 - g_{f_0}, \varphi \rangle_{L^2(X)} = \langle -u,\varphi \rangle_{L^2(X)} = R_{L^2(X)}(-u)(\varphi)
		\end{equation*}
		\noindent holds for all $\varphi \in L^2(X)$. Thus $R_{L^2(X)}(cf_0 - g_{f_0}) = R_{L^2(X)}(-u)$ and so the \emph{Riesz representation theorem} yields that
		\begin{equation*}
			cf_0 - g_{f_0} = -u.
		\end{equation*}

		\begin{lemma}
			$a : L^2(X) \times L^2(X) \to \mathbb{C}$ is a continuous coercive sesquilinear form.
		\end{lemma}

		\begin{proof}
			Clearly $a$ is sesquilinear, i.e. antilinear in the first and linear in the second argument, by the corresponding properties of the $L^2(X)$ inner product and the simple observation that
			\begin{equation*}
				g_{f + \lambda h} = g_f + \lambda g_h
			\end{equation*}
			\noindent for $\lambda \in \mathbb{C}$ and $f,h \in L^2(X)$. Let $\varphi \in L^2(X)$. Then Cauchy-Schwarz together with lemma \ref{lem:L^2} yields
			\begin{align*}
				\abs{a(f,\varphi)} &= \abs[0]{\langle cf - g_f,\varphi \rangle_{L^2(X)}}\\
				&\leq \norm[0]{cf - g_f}_{L^2(X)}\norm[0]{\varphi}_{L^2(X)}\\
				&\leq \del[1]{c\norm[0]{f}_{L^2(X)} + \norm[0]{g_f}_{L^2(X)}}\norm{\varphi}_{L^2(X)}\\
				&\leq \del[1]{c\norm{f}_{L^2(X)} + \norm[0]{f}_{L^2(X)}\norm{L}_{L^2(X \times X)}} \norm{\varphi}_{L^2(X)}\\
				&\leq 2c\norm{f}_{L^2(X)}\norm{\varphi}_{L^2(X)} 
			\end{align*}
			\noindent since $\norm{L}_{L^2(X \times X)} < c$. Lastly, since $\norm{L}_{L^2(X \times X)} < c$, we find $c_0$ such that we have $\norm{L}_{L^2(X \times X)} < c_0 < c$ by trichotomy. Thus again lemma \ref{lem:L^2} together with Cauchy-Schwarz implies that 
			\begin{align*}
				\Re a(f,f) &= \Re \langle cf - g_f,f \rangle_{L^2(X)}\\
				&= \Re c\langle f,f \rangle_{L^2(X)} - \Re \langle g_f,f\rangle_{L^2(X)}\\
				&= c\norm{f}_{L^2(X)}^2 - \Re \langle g_f,f\rangle_{L^2(X)}\\
				&\geq c\norm{f}_{L^2(X)}^2 - \abs[0]{\langle g_f,f \rangle_{L^2(X)}}\\
				&\geq c \norm{f}_{L^2(X)}^2 - \norm[0]{g_f}_{L^2(X)}\norm{f}_{L^2(X)}\\
				&\geq c \norm{f}_{L^2(X)}^2 - \norm{f}_{L^2(X)}^2 \norm{L}_{L^2(X \times X)}\\
				&\geq (c  -c_0)\norm{f}_{L^2(X)}^2.
			\end{align*}
			Now $2c > 0$ since $c > \norm{L}_{L^2(X \times X)}$, $c - c_0 > 0$ and $c - c_0 \leq c \leq 2c$ since $c_0 > 0$.
		\end{proof}

	\item 
		~
		\begin{enumerate}[label = \textbf{\alph*.},wide = 0pt, itemsep = 1.5ex]
			\item Suppose that $M \setminus \wbar{A}$ is dense in $M$. Towards a contradiction, assume that $A$ is not nowhere dense. Hence $\mathring{\wbar{A}} \neq \varnothing$. Since $\mathring{\wbar{A}}$ is open by definition of the interior of a set, there exists $\varepsilon > 0$ and $x \in \mathring{\wbar{A}}$ such that $B_\varepsilon(x) \subseteq \mathring{\wbar{A}}$. Moreover, $\mathring{\wbar{A}} \subseteq \wbar{A}$ and thus $B_\varepsilon(x) \subseteq \wbar{A}$. This implies that $B_\varepsilon(x)$ and $M \setminus \wbar{A}$ are disjoint. But $M \setminus \wbar{A}$ is dense in $M$, hence we find a sequence $(x_n)_{n \in \mathbb{N}}$ in $M \setminus \wbar{A}$ such that $x_n \to x$. Hence there exists $N \in \mathbb{N}$ such that $x_n \in B_\varepsilon(x)$ for all $n \geq N$. This is not possible since $B_\varepsilon(x)$ does not contain any elements of $M \setminus \wbar{A}$. Contradiction. 

			\item 
				\begin{lemma}
					Let $(x_k)_{k \in \mathbb{N}}$ be an enumeration of $\mathbb{Q}$. For $n \in \mathbb{N}$ define
					\begin{equation}
						E_n := \bigcup_{k \in \mathbb{N}} \intoo[3]{x_k - \frac{1}{2^kn},x_k + \frac{1}{2^kn}}
					\end{equation}
					\noindent and
					\begin{equation}
						E := \bigcap_{n \in \mathbb{N}} E_n.
					\end{equation}
					Set $A := E^c$. Then $A$ is meager and $\lambda(A^c) = 0$, where $\lambda$ denotes the ordinary Lebesgue measure on $\mathbb{R}$. 
				\end{lemma}

				\begin{proof}
					We show first that $\lambda(A^c) = 0$. First observe that $(E_n)_{n \in \mathbb{N}}$ is a decreasing sequence of $\lambda$-measurable sets. Moreover, for any $n \in \mathbb{N}$ we have that 
					\begin{equation*}
						\lambda(E_n) \leq \frac{1}{n}\sum_{k \in \mathbb{N}}\frac{1}{2^{k-1}} = \frac{1}{n}\sum_{k \in \mathbb{N}_0}\frac{1}{2^{k}} = \frac{2}{n} < \infty 
					\end{equation*}
					\noindent by subadditivity of the measure. Elementary measure theory now tells us that
					\begin{equation*}
						\lambda(A^c) = \lambda(E) = \lambda\del[1]{\cap_{n \in \mathbb{N}} E_n} = \lim_{n \to \infty}\lambda(E_n) \leq \lim_{n \to \infty} \frac{2}{n} = 0. 	
					\end{equation*}
					Let us show that $A$ is meager. Since $A = E^c = \cup_{n \in \mathbb{N}} E_n^c$, we show that $E_n^c$ is nowhere dense for all $n \in \mathbb{N}$. By part \textbf{a.} we can also show that $\mathbb{R} \setminus \wbar{E_n^c}$ is dense in $\mathbb{R}$. For fixed $n \in \mathbb{N}$ we have that
					\begin{equation*}
						E_n^c = \bigcap_{k \in \mathbb{N}}\del[3]{\intoc[2]{-\infty, x_k - \frac{1}{2^kn}} \cup \intco[2]{x_k + \frac{1}{2^kn},\infty}}.
					\end{equation*}
					Thus $E_n^c$ is a closed set (finite unions and countable intersection of closed intervals) and so $\wbar{E_n^c} = E_n^c$. So $\mathbb{R} \setminus \wbar{E_n^c} = E_n$. But $\mathbb{Q} \subseteq E_n$ for all $n \in \mathbb{N}$ and thus $E_n$ is dense in $\mathbb{R}$.
				\end{proof}
		\end{enumerate}
	\item
		~
		\begin{enumerate}[label = \textbf{\alph*.},wide = 0pt, itemsep = 1.5ex]
			\item If $A = \varnothing$, we have that $\cap_{\alpha \in A} \mathcal{T}_\alpha = \mathcal{P}(X)$ since topologies on $X$ are subsets of $\mathcal{P}(X)$. Hence the intersection of the empty family of topologies on $X$ is the discrete topology.\\
				Consider now $A \neq \varnothing$. Clearly, $\varnothing, X \in \cap_{\alpha \in A} \mathcal{T}_\alpha$ since $\varnothing,X \in \mathcal{T}_{\alpha}$ for all $\alpha \in A$. Let $U_1,\dots,U_n \in \cap_{\alpha \in A}\mathcal{T}_\alpha$. Hence $U_1,\dots,U_n \in \mathcal{T}_\alpha$ for all $\alpha \in A$ and so $U_1 \cap \dots \cap U_n \in \mathcal{T}_\alpha$ for all $\alpha \in A$. Hence $U_1 \cap \dots \cap U_n \in \cap_{\alpha \in A}\mathcal{T}_\alpha$. Finally, suppose that $(U_\beta)_{\beta \in B}$ is a family in $\cap_{\alpha \in A} \mathcal{T}_\alpha$. Hence for all $\alpha \in A$ we have that $U_\beta \in \mathcal{T}_\alpha$ for all $\beta \in B$. So $\cup_{\beta \in B} U_\beta \in \mathcal{T}_\alpha$ for all $\alpha \in A$ and therefore $\cup_{\beta \in B} U_\beta \in \cap_{\alpha \in A} \mathcal{T}_\alpha$.
			\item Define
				\begin{equation*}
					\mathcal{B} := \cbr{U_1 \cap \dots \cap U_n : n \in \mathbb{N},U_i \in \mathcal{\altS} \text{ for all } i = 1,\dots,n}
				\end{equation*}
				\noindent and 
				\begin{equation*}
					\mathcal{T} := \cbr{\textstyle{\cup}_{\alpha \in A} B_\alpha : B_\alpha \in A \text{ for all } \alpha \in A}.
				\end{equation*}
		
				\begin{lemma}
					$\mathcal{T}_\mathcal{F} = \mathcal{T}$.
				\end{lemma}

				\begin{proof}
					By part \textbf{a.}, $\mathcal{T}_\mathcal{F}$ is a topology. We show that also $\mathcal{T}$ is a topology. By \cite[34]{lee:topological_manifolds:2011} it is enough to show that $\mathcal{B}$ satisfies the following two conditions:
					\begin{enumerate}[label = (\roman*)]
						\item $\cup_{B \in \mathcal{B}} B = X$.
						\item If $B_1,B_2 \in \mathcal{B}$ and $x \in B_1 \cap B_2$, there exists an element $B_3 \in \mathcal{B}$ such that $x \in B_3 \subseteq B_1 \cap B_2$.
					\end{enumerate}
					Then $\mathcal{T}$ is the unique topology on $X$ generated by $\mathcal{B}$, i.e. the collection of arbitrary unions of elements of $\mathcal{B}$. Since $\mathcal{F}$ is nonempty, there exists $f \in \mathcal{F}$. Clearly $X = f^{-1}(Y_f)$ and $Y_f$ is open in $Y_f$. Hence $f^{-1}(Y_f) \in \mathcal{\altS}$ and thus $X \in \cup_{B \in \mathcal{B}} B$. Suppose that $B_1,B_2 \in \mathcal{B}$ such that $B_1 \cap B_2 \neq \varnothing$. Hence we find $U_1,\dots,U_n,V_1,\dots,V_m \in \mathcal{\altS}$ such that $B_1 = U_1 \cap \dots \cap U_n$ and $B_2 = V_1 \cap \dots \cap V_m$. Suppose $x \in B_1 \cap B_2$. Then also $x \in U_1 \cap \dots \cap U_n \cap V_1 \cap \dots \cap V_m$. But 
					\begin{equation*}
						U_1 \cap \dots \cap U_n \cap V_1 \cap \dots \cap V_m \in \mathcal{B}
					\end{equation*}
					\noindent as a finite intersection of elements of $\mathcal{\altS}$. Hence $\mathcal{T}$ is a topology.\\
					Clearly, $\mathcal{\altS} \subseteq \mathcal{T}$, since already $\mathcal{\altS} \subseteq \mathcal{B}$. Since $\mathcal{T}_\mathcal{F}$ is the smallest topology containing $\mathcal{\altS}$, we get that $\mathcal{T}_\mathcal{F} \subseteq \mathcal{T}$.\\
					Let $U \in \mathcal{T}$. Then $U = \cup_{\alpha \in A}B_\alpha$ for some index set $A$ and $B_\alpha \in \mathcal{B}$ for all $\alpha \in A$. But each $B_\alpha$ is a finite intersection of elements of $\mathcal{\altS}$ and thus since $\mathcal{T}_\mathcal{F}$ is a topology containing $\mathcal{\altS}$, we have that $B_\alpha \in \mathcal{T}_\mathcal{F}$ for all $\alpha \in A$. But then also $U \in \mathcal{T}_\mathcal{F}$ as a union of sets in $\mathcal{T}_\mathcal{F}$. Hence $\mathcal{T} \subseteq \mathcal{T}_\mathcal{F}$. 
				\end{proof}
		\end{enumerate}
	\item Suppose that $x_n \rightharpoonup x$. Proposition 6.2.2 implies that the sequence $(x_n)_{n \in \mathbb{N}}$ is bounded, in particular $\sup_{n \in \mathbb{N}} \norm{x_n} < \infty$. Moreover, lemma 6.2.1 yields $f(x_n) \to f(x)$ for all $f \in X^*$. Since $Y \subseteq X^*$ we also have $f(x_n) \to f(x)$ for all $f \in Y$.\\
		Conversly, suppose $\norm{x_n} \leq M$ for some $M \geq 0$ and $f(x_n) \to f(x)$ for all $f \in Y$. Let $f \in X^*$. Since $Y$ is dense in $X^*$, we find a sequence $(f_k)_{k \in \mathbb{N}}$ in $Y$ such that $\norm{f_k - f} \to 0$. Hence
		\begin{align*}
			\abs{f(x_n) - f(x)} &= \abs{f(x_n) - f(x) + f_k(x_n) - f_k(x_n) + f_k(x) - f_k(x)}\\
			&\leq \abs{f(x_n) - f_k(x_n)} + \abs{f_k(x_n) - f_k(x)} + \abs{f_k(x) - f(x)}\\
			&\leq \norm{f - f_k}\del[0]{\norm{x_n} + \norm{x}} + \abs{f_k(x_n) - f_k(x)}\\
			&\leq \norm{f - f_k}\del[0]{M + \norm{x}} + \abs{f_k(x_n) - f_k(x)}\\
		\end{align*}
		\noindent and so
		\begin{equation*}
			\lim_{n \to \infty} \abs{f(x_n) - f(x)} \leq \norm{f - f_k}\del[0]{M + \norm{x}} \xrightarrow{k \to \infty} 0.
		\end{equation*}
		Thus $f(x_n) \to f(x)$ for all $f \in X^*$ and so lemma 6.2.1 implies $x_n \rightharpoonup x$.
\end{enumerate}
\printbibliography
\end{document}
