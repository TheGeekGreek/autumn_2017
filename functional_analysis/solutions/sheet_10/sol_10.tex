%%%%%%%%%%%%%%%%%%%%%%%%%%%%%%%%%%%%%%%%%%%%%%%%%%%%%%%%%%%%%%%%%%%%%%%%%%
%Author:																 %
%-------																 %
%Yannis Baehni at University of Zurich									 %
%baehni.yannis@uzh.ch													 %
%																		 %
%Version log:															 %
%------------															 %
%06/02/16 . Basic structure												 %
%04/08/16 . Layout changes including section, contents, abstract.		 %
%%%%%%%%%%%%%%%%%%%%%%%%%%%%%%%%%%%%%%%%%%%%%%%%%%%%%%%%%%%%%%%%%%%%%%%%%%

%Page Setup
\documentclass[
	12pt, 
	oneside, 
	a4paper,
	reqno,
	final
]{amsart}

\usepackage[
	left = 3cm, 
	right = 3cm, 
	top = 3cm, 
	bottom = 3cm
]{geometry}

%Headers and footers
\usepackage{fancyhdr}
	\pagestyle{fancy}
	%Clear fields
	\fancyhf{}
	%Header right
	\fancyhead[R]{
		\footnotesize
		Yannis B\"{a}hni\\
		\href{mailto:yannis.baehni@uzh.ch}{yannis.baehni@uzh.ch}
	}
	%Header left
	\fancyhead[L]{
		\footnotesize
		401-3581-67L Symplectic Geometry\\
		Autumn 2017
	}
	%Page numbering in footer
	\fancyfoot[C]{\thepage}
	%Separation line header and footer
	\renewcommand{\headrulewidth}{0.4pt}
	%\renewcommand{\footrulewidth}{0.4pt}
	
	\setlength{\headheight}{19pt} 

%Title
\usepackage[foot]{amsaddr}
\usepackage{upref}
\usepackage{newtxtext}
\usepackage[subscriptcorrection,nofontinfo,mtpcal,mtphrb]{mtpro2}
\usepackage{bm}
\usepackage{xspace}
\makeatletter
\usepackage{etoolbox}
\patchcmd{\abstract}{\scshape\abstractname}{\textbf{\abstractname}}{}{}

\usepackage[all,cmtip]{xy}

%Section, subsection and subsubsection font
%------------------------------------------
\makeatletter
	\renewcommand{\@secnumfont}{\bfseries}
	\renewcommand\section{\@startsection{section}{1}%
  	\z@{.7\linespacing\@plus\linespacing}{.5\linespacing}%
  	{\normalfont\bfseries\centering}}
	\renewcommand\subsection{\@startsection{subsection}{2}%
    	\z@{.5\linespacing\@plus.7\linespacing}{-.5em}%
    	{\normalfont\bfseries}}%
	\renewcommand\subsubsection{\@startsection{subsubsection}{3}%
    	\z@{.5\linespacing\@plus.7\linespacing}{-.5em}%
    	{\normalfont\bfseries}}%
%Formatting title of TOC
\renewcommand{\contentsnamefont}{\bfseries}
%Table of Contents
\setcounter{tocdepth}{3}

% Add bold to \section titles in ToC and remove . after numbers
\renewcommand{\tocsection}[3]{%
  \indentlabel{\@ifnotempty{#2}{\bfseries\ignorespaces#1 #2\quad}}\bfseries#3}
% Remove . after numbers in \subsection
\renewcommand{\tocsubsection}[3]{%
  \indentlabel{\@ifnotempty{#2}{\ignorespaces#1 #2\quad}}#3}
\let\tocsubsubsection\tocsubsection% Update for \subsubsection
%...

\newcommand\@dotsep{4.5}
\def\@tocline#1#2#3#4#5#6#7{\relax
  \ifnum #1>\c@tocdepth % then omit
  \else
    \par \addpenalty\@secpenalty\addvspace{#2}%
    \begingroup \hyphenpenalty\@M
    \@ifempty{#4}{%
      \@tempdima\csname r@tocindent\number#1\endcsname\relax
    }{%
      \@tempdima#4\relax
    }%
    \parindent\z@ \leftskip#3\relax \advance\leftskip\@tempdima\relax
    \rightskip\@pnumwidth plus1em \parfillskip-\@pnumwidth
    #5\leavevmode\hskip-\@tempdima{#6}\nobreak
    \leaders\hbox{$\m@th\mkern \@dotsep mu\hbox{.}\mkern \@dotsep mu$}\hfill
    \nobreak
    \hbox to\@pnumwidth{\@tocpagenum{\ifnum#1=1\bfseries\fi#7}}\par% <-- \bfseries for \section page
    \nobreak
    \endgroup
  \fi}
\AtBeginDocument{%
\expandafter\renewcommand\csname r@tocindent0\endcsname{0pt}
}
\def\l@subsection{\@tocline{2}{0pt}{2.5pc}{5pc}{}}
\def\l@subsubsection{\@tocline{2}{0pt}{4.5pc}{5pc}{}}
\makeatother

\advance\footskip0.4cm
\textheight=54pc    %a4paper
\textheight=50.5pc %letterpaper
\advance\textheight-0.4cm
\calclayout

%Font settings
%\usepackage{anyfontsize}
%Footnote settings
%\usepackage{mathptmx}
\usepackage{footmisc}
%	\renewcommand*{\thefootnote}{\fnsymbol{footnote}}
\usepackage{commath}
%Further math environments
%Further math fonts (loads amsfonts implicitely)
%Redefinition of \text
%\usepackage{amstext}
\usepackage{upref}
%Graphics
%\usepackage{graphicx}
%\usepackage{caption}
%\usepackage{subcaption}
%Frames
\usepackage{mdframed}
\allowdisplaybreaks
%\usepackage{interval}
\newcommand{\toup}{%
  \mathrel{\nonscript\mkern-1.2mu\mkern1.2mu{\uparrow}}%
}
\newcommand{\todown}{%
  \mathrel{\nonscript\mkern-1.2mu\mkern1.2mu{\downarrow}}%
}
\AtBeginDocument{\renewcommand*\d{\mathop{}\!\mathrm{d}}}
\renewcommand{\Re}{\operatorname{Re}}
\renewcommand{\Im}{\operatorname{Im}}
\DeclareMathOperator\Log{Log}
\DeclareMathOperator\Arg{Arg}
\DeclareMathOperator\sech{sech}
\DeclareMathOperator*\esssup{ess.sup}
\DeclareMathOperator\id{id}
\DeclareMathOperator\im{im}
\DeclareMathOperator\Vol{Vol}
\DeclareMathOperator\dist{dist}
%\usepackage{hhline}
%\usepackage{booktabs} 
%\usepackage{array}
%\usepackage{xfrac} 
%\everymath{\displaystyle}
%Enumerate
\usepackage{tikz}
\usetikzlibrary{external}
\tikzexternalize % activate!
\usetikzlibrary{patterns}
\pgfdeclarepatternformonly{adjusted lines}{\pgfqpoint{-1pt}{-1pt}}{\pgfqpoint{40pt}{40pt}}{\pgfqpoint{39pt}{39pt}}%
{
  \pgfsetlinewidth{.8pt}
  \pgfpathmoveto{\pgfqpoint{0pt}{0pt}}
  \pgfpathlineto{\pgfqpoint{39.1pt}{39.1pt}}
  \pgfusepath{stroke}
}
\usepackage{enumitem} 
%\renewcommand{\labelitemi}{$\bullet$}
%\renewcommand{\labelitemii}{$\ast$}
%\renewcommand{\labelitemiii}{$\cdot$}
%\renewcommand{\labelitemiv}{$\circ$}
%Colors
%\usepackage{color}
%\usepackage[cmtip, all]{xy}
%Theorems
\newtheoremstyle{main} 		             	 		%Stylename
  	{}	                                     		%Space above
  	{}	                                    		%Space below
  	{\itshape}			                     		%Body font
  	{}        	                             		%Indent
  	{\bfseries}   	                         		%Head font
  	{.}            	                        		%Head punctuation
  	{ }           	                         		%Head space 
  	{\thmname{#1}\thmnumber{ #2}\thmnote{ (#3)}}	%Head specification
\theoremstyle{main}
\newtheorem{definition}{Definition}[section]
\newtheorem{proposition}{Proposition}[section]
\newtheorem{corollary}{Corollary}[section]
\newtheorem{theorem}{Theorem}[section]
\newtheorem{lemma}{Lemma}[section]
%Roman style theorems
\newtheoremstyle{roman}
	{}
	{}
  	{}
  	{}
	{\bfseries}
	{.}
  	{ }
	{\thmname{#1}\thmnumber{ #2}\thmnote{ (#3)}}
\theoremstyle{roman}
\newtheorem{example}{Example}[section]
\newtheorem{solution}{Solution}[section]
\newtheorem{remark}{Remark}[section]
%Exercise style theorems
\newtheoremstyle{exercise}
  	{}
  	{}
  	{\small}
  	{}
  	{\bfseries}
  	{.}
 	{ }
  	{\thmname{#1}\thmnumber{ #2}\thmnote{ (#3)}}
\theoremstyle{exercise}
\newtheorem{exercise}{Exercise}[section]
%Changing default style of proof environment
\renewcommand*{\proofname}{\itshape Proof}
%German non-ASCII-Characters
%Graphics-Tool
%\usepackage{tikz}
%\usepackage{tikzscale}
%\usepackage{bbm}
%\usepackage{bera}
%Listing-Setup
%Bibliographie
\usepackage[backend=bibtex, style=alphabetic]{biblatex}
%\usepackage[babel, german = swiss]{csquotes}
\bibliography{../latex/bibliography}
%PDF-Linking
%\usepackage[hyphens]{url}
\usepackage[bookmarksopen=true,bookmarksnumbered=true]{hyperref}
%\PassOptionsToPackage{hyphens}{url}\usepackage{hyperref}
\hypersetup{
  colorlinks   = true, %Colours links instead of ugly boxes
  urlcolor     = blue, %Colour for external hyperlinks
  linkcolor    = blue, %Colour of internal links
  citecolor    = blue %Colour of citations
}
%Weierstrass-P symbol for power set
\newcommand{\powerset}{\raisebox{.15\baselineskip}{\Large\ensuremath{\wp}}}
\newcommand{\bld}[1]{\boldmath\textit{\textbf{#1}}\unboldmath}
\usepackage{pict2e}
\makeatletter
\DeclareRobustCommand{\intprod}{%
	\mathbin{\mathpalette\int@prod{(0.1,0)(0.9,0)(0.9,0.8)}}%
}
\newcommand{\int@prod}[2]{%
	\begingroup
	\sbox\z@{$\m@th#1+$}%
	\setlength\unitlength{\wd\z@}%
	\begin{picture}(1,1)
	\roundcap
	\polyline#2
	\end{picture}%
	\endgroup
}
\makeatother
\newcommand{\Sbb}{\mathbb{S}}
\newcommand{\dRrm}{\mathrm{dR}}
\newcommand{\Rbb}{\mathbb{R}}
\newcommand{\Lcal}{\mathcal{L}}
\newcommand{\Zbb}{\mathbb{Z}}
\newcommand{\Nbb}{\mathbb{N}}
\newcommand{\Xfrak}{\mathfrak{X}}


\title{Solutions Sheet 10}
\author{Yannis B\"{a}hni}
\address[Yannis B\"{a}hni]{University of Zurich, R\"{a}mistrasse 71, 8006 Zurich}
\email[Yannis B\"{a}hni]{\href{mailto:yannis.baehni@uzh.ch}{\nolinkurl{yannis.baehni@uzh.ch}}}

\begin{document}

\maketitle
\thispagestyle{fancy}

\setcounter{section}{1}

\begin{enumerate}[label = \textbf{Exercise \arabic*.},wide = 0pt, itemsep = 1.5ex]
	\item
		~
		\begin{enumerate}[label = \textbf{\alph*.},wide = 0pt, itemsep = 1.5ex]
			\item
				\begin{lemma}
					Let $\sbr{x} \in X/M$. Then
					\begin{equation*}
						\norm{\sbr{x}}_{X/M} = \inf_{m \in M} \norm{x - m}.
					\end{equation*}
				\end{lemma}

				\begin{proof}
					This immediately follows from
					\begin{equation*}
						\cbr[0]{ \norm{y} : y \in \sbr{x}} = \cbr[0]{ \norm{x - m} : m \in M}.
					\end{equation*}
					Indeed, if $y \in \sbr{x}$, by definition $x - y \in M$ and thus there exists some $m \in M$ such that $x - y = m$ or equivalently $y = x - m$. Conversly, $x - m \in \sbr{x}$.
				\end{proof}
				There are four things to check. 
				\begin{itemize}[leftmargin = *]
					\item \bld{(Well definedness)} Let $\sbr{x},\sbr[0]{y} \in X/M$ such that $\sbr{x} = \sbr[0]{y}$. Hence $x \sim y$ and thus we find $m_0 \in M$ such that $x - y = m_0$. Thus
						\begin{equation*}
							\norm[0]{\sbr{x}}_{X/M} = \inf_{m \in M}\norm{x - m} = \inf_{m \in M} \norm{y - (m - m_0)} = \inf_{\wtilde{m} \in M}\norm[0]{y - \wtilde{m}} = \norm[0]{\sbr[0]{y}}_{X/M}
						\end{equation*}
						\noindent since $M$ is a linear subspace.
					\item \bld{(Positivity)} Let $\sbr{x} \in X/M$. If $\sbr{x} = 0$ we have that $x \in M$. But then 
						\begin{equation*}
							\norm{\sbr{x}}_{X/M} = \inf_{m \in M} \norm{x - m} = 0.
						\end{equation*}
						Conversly, assume that $\norm{\sbr{x}}_{X/M} = 0$. By the definition of the infimum, we can construct a sequence $(m_n)_{n \in \mathbb{N}}$ in $M$ such that $\norm{x - m_n} \to 0$. But then $m_n \to x$ and since $M$ is closed we have that $x \in M$. Hence $\sbr{x} = 0$.
					\item \bld{(Homogeneity)} Let $\sbr{x} \in X/M$ and $\lambda \in \mathbb{K}$. The case $\lambda = 0$ is clear. So assume $\lambda \neq 0$. Then 
						\begin{align*}
							\norm{\lambda\sbr{x}}_{X/M} &= \norm{\sbr{\lambda x}}_{X/M}\\
							&= \inf_{m \in M} \norm{\lambda x - m}\\
							&= \inf_{m \in M} \abs{\lambda}\norm{x - m/\lambda}\\
							&= \abs{\lambda} \inf_{m \in M} \norm{x - m/\lambda}\\
							&= \abs{\lambda} \inf_{\wtilde{m} \in M} \norm[0]{x - \wtilde{m}}\\
							&= \abs{\lambda} \norm{\sbr{x}}_{X/M}
						\end{align*}
					\noindent since $M$ is a linear subspace.
				\item \bld{(Triangle inequality)} Let $\sbr{x},\sbr[0]{y} \in X/M$. Then 
					\begin{align*}
						\norm[0]{\sbr{x} + \sbr[0]{y}}_{X/M} &= \norm[0]{\sbr[0]{x + y}}_{X/M}\\
						&= \inf_{m \in M} \norm[0]{x + y - m}\\
						&= \inf_{m \in M} \norm[0]{x + y - 2m + m}\\
						&\leq \inf_{m \in M} \norm[0]{x - m} + \inf_{m \in M}\norm[0]{y - m} + \inf_{m \in M}\norm{m}\\
						&= \inf_{m \in M} \norm[0]{x - m} + \inf_{m \in M}\norm[0]{y - m}\\
						&= \norm[0]{\sbr{x}}_{X/M} + \norm[0]{\sbr[0]{y}}_{X/M}
					\end{align*}
					\noindent since $M$ is a linear subspace and thus $0 \in M$.
				\end{itemize}
			\item Let $x \in X$. By part \textbf{a.} we have that
				\begin{equation*}
					\norm[0]{\pi(x)}_{X/M} = \norm{\sbr{x}}_{X/M} = \inf_{m \in M} \norm{x - m} \leq \inf_{m \in M}\norm{x} + \inf_{m \in M}\norm{m} = \norm{x}.
				\end{equation*}
			\item Let $(\sbr[0]{x_n})_{n \in \mathbb{N}}$ be a Cauchy sequence in $X/M$. Then $(x_n)_{n \in \mathbb{N}}$ is a Cauchy sequence in $X$. Indeed, for any $m \in M$ we have that 
				\begin{equation*}
					\norm{x_n - x_k} \leq \norm{x_n - x_k - m} + \norm{m}
				\end{equation*}
				And thus
				\begin{equation*}
					\norm{x_n - x_k} \leq \inf_{m \in M}\norm{x_n - x_k - m} + \inf_{m \in M}\norm{m} = \norm{\sbr{x_n - x_k}}_{X/M} = \norm{\sbr{x_n} - \sbr{x_k}}_{X/M} \xrightarrow{n,k \to \infty} 0.
				\end{equation*}
				Since $X$ is a Banach space, there exists $x \in X$ such that $x_n \to x$. Then $\sbr{x_n} \to \sbr{x}$. Indeed, by part \textbf{b.} we have
				\begin{equation*}
					\lim_{n \to \infty}\sbr{x_n} = \lim_{n \to \infty}\pi(x_n) = \pi(x) = \sbr{x}.
				\end{equation*}
			\item Define $\wtilde{T} : X/\ker T \to T(X)$ by 
				\begin{equation*}
					\wtilde{T}(\sbr{x}) := T(x).
				\end{equation*}
				This mapping is well defined. Indeed, if $\sbr{x} = \sbr[0]{y} \in X/\ker T$, we have that $x - y \in \ker T$ and thus
				\begin{equation*}
					\wtilde{T}(\sbr{x}) = T(x) = T(x - y + y) = T(x - y) + T(y) = T(y) = \wtilde{T}(\sbr[0]{y})
				\end{equation*}
				\noindent by the linearity of $T$. Also $\wtilde{T}$ is linear. Let $\lambda \in \mathbb{K}$. Then we have
				\begin{equation*}
					\wtilde{T}(\sbr{x} + \lambda \sbr[0]{y}) = \wtilde{T}(\sbr[0]{x + \lambda y}) = T(x + \lambda y) = T(x) + \lambda T(y) = \wtilde{T}(\sbr{x}) + \lambda \wtilde{T}(\sbr[0]{y}).
				\end{equation*}
				Clearly, $\wtilde{T}$ is surjective. Also $\wtilde{T}$ is injective since if $\sbr{x} \in \ker \wtilde{T}$, we have that 
				\begin{equation*}
					0 = \wtilde{T}(\sbr{x}) = T(x)
				\end{equation*}
				\noindent and thus $x \in \ker T$ which implies $\sbr{x} = 0$. Next we verify the commutativity of the diagram. Let $x \in X$. Then
				\begin{equation*}
					(\iota \circ \wtilde{T} \circ \pi)(x) = \iota\del[1]{\wtilde{T}(\sbr{x})} = \iota\del[1]{T(x)} = T(x).
				\end{equation*}
				Lastly we show that $\norm[0]{\wtilde{T}} = \norm{T}$ which in particular implies $\wtilde{T} \in \mathcal{L}\del[1]{X/\ker T,T(X)}$. Indeed, by part \textbf{b.} we have that $\norm[0]{\pi(x)}_{X/M} \leq \norm{x}$ for all $x \in X$ and thus
				\begin{equation*}
					\norm[0]{\wtilde{T}(\sbr{x})} \leq \norm[0]{\wtilde{T}}\norm{\sbr{x}}_{X/M} = \norm[0]{\wtilde{T}}\norm{\pi(x)}_{X/M} \leq \norm[0]{\wtilde{T}}\norm{x} = \norm{T}\norm{x}
				\end{equation*}
				\noindent for all $\sbr{x} \in X/M$.
				\begin{itemize}[leftmargin = *]
					\item \bld{($\norm{T} \leq \norm[0]{\wtilde{T}}$)} Observe that
						\begin{equation*}
							\cbr[0]{x \in X : \norm{x} \leq 1} \subseteq \cbr[0]{x \in X : \norm{\sbr{x}}_{X/M} \leq 1}
						\end{equation*}
						\noindent by the continuity of $\pi$. Thus
						\begin{equation*}
							\norm{T} = \sup_{\norm[0]{x} \leq 1} \norm{T(x)} \leq \sup_{\norm[0]{\sbr[0]{x}}_{X/M} \leq 1} \norm{T(x)} = \sup_{\norm[0]{\sbr[0]{x}}_{X/M} \leq 1} \norm[0]{T(\sbr{x})} = \norm[0]{\wtilde{T}}.
						\end{equation*}
				\end{itemize}
	
		\end{enumerate}
\end{enumerate}
\printbibliography
\end{document}
