%%%%%%%%%%%%%%%%%%%%%%%%%%%%%%%%%%%%%%%%%%%%%%%%%%%%%%%%%%%%%%%%%%%%%%%%%%
%Author:																 %
%-------																 %
%Yannis Baehni at University of Zurich									 %
%baehni.yannis@uzh.ch													 %
%																		 %
%Version log:															 %
%------------															 %
%06/02/16 . Basic structure												 %
%04/08/16 . Layout changes including section, contents, abstract.		 %
%%%%%%%%%%%%%%%%%%%%%%%%%%%%%%%%%%%%%%%%%%%%%%%%%%%%%%%%%%%%%%%%%%%%%%%%%%

%Page Setup
\documentclass[
	12pt, 
	oneside, 
	a4paper,
	reqno,
	final
]{amsbook}

\usepackage[
	left = 3cm, 
	right = 3cm, 
	top = 3cm, 
	bottom = 3cm
]{geometry}

%Headers and footers
\usepackage{fancyhdr}
	\pagestyle{fancy}
	%Clear fields
	\fancyhf{}
	%Header right
	\fancyhead[R]{
		\footnotesize
		Yannis B\"{a}hni\\
		\href{mailto:yannis.baehni@uzh.ch}{yannis.baehni@uzh.ch}
	}
	%Header left
	\fancyhead[L]{
		\footnotesize
		401-3001-61L Algebraic Topology I\\
		Autumn Semester 2017
	}
	%Page numbering in footer
	\fancyfoot[C]{\thepage}
	%Separation line header and footer
	\renewcommand{\headrulewidth}{0.4pt}
	%\renewcommand{\footrulewidth}{0.4pt}
	
	\setlength{\headheight}{19pt} 

%Title
\usepackage[foot]{amsaddr}
\usepackage{newtxtext}
\usepackage[subscriptcorrection,nofontinfo,mtpcal,mtphrb]{mtpro2}
\usepackage{mathtools}
\usepackage{bm}
\usepackage{xspace}
\usepackage[all]{xy}
\usepackage{tikz-cd}
\makeatletter
\def\@textbottom{\vskip \z@ \@plus 1pt}
\let\@texttop\relax
\usepackage{etoolbox}
\patchcmd{\abstract}{\scshape\abstractname}{\textbf{\abstractname}}{}{}
\usepackage{chngcntr}
\counterwithout{figure}{chapter}
%Section, subsection and subsubsection font
%------------------------------------------
	\renewcommand{\@secnumfont}{\bfseries}
	\renewcommand\section{\@startsection{section}{1}%
  	\z@{.7\linespacing\@plus\linespacing}{.5\linespacing}%
  	{\normalfont\bfseries\boldmath\centering}}
	\renewcommand\subsection{\@startsection{subsection}{2}%
    	\z@{.5\linespacing\@plus.7\linespacing}{-.5em}%
    	{\normalfont\bfseries\boldmath}}%
	\renewcommand\subsubsection{\@startsection{subsubsection}{3}%
    	\z@{.5\linespacing\@plus.7\linespacing}{-.5em}%
    	{\normalfont\bfseries\boldmath}}%

		\renewenvironment{proof}{\textit{Proof}.}{\hfill\qedsymbol}

%ToC
%---
\makeatletter
\setcounter{tocdepth}{3}
% Add bold to \chapter titles in ToC and remove . after numbers
\renewcommand{\tocchapter}[3]{%
  	\indentlabel{\@ifnotempty{#2}{\bfseries\ignorespaces#1 #2: }}\bfseries#3}
\renewcommand{\tocappendix}[3]{%
  	\indentlabel{\@ifnotempty{#2}{\bfseries\ignorespaces#1 #2: }}\bfseries#3}
% Remove . after numbers in \section and \subsection
\renewcommand{\tocsection}[3]{%
  	\indentlabel{\@ifnotempty{#2}{\ignorespaces#1 #2\quad}}#3}
\renewcommand{\tocsubsection}[3]{%
  	\indentlabel{\@ifnotempty{#2}{\ignorespaces#1 #2\quad}}#3}
\let\tocsubsubsection\tocsubsection% Update for \subsubsection
%...
\newcommand\@dotsep{4.5}
\def\@tocline#1#2#3#4#5#6#7{\relax
  \ifnum #1>\c@tocdepth % then omit
  \else
    \par \addpenalty\@secpenalty\addvspace{#2}%
    \begingroup \hyphenpenalty\@M
    \@ifempty{#4}{%
      \@tempdima\csname r@tocindent\number#1\endcsname\relax
    }{%
      \@tempdima#4\relax
    }%
    \parindent\z@ \leftskip#3\relax \advance\leftskip\@tempdima\relax
    \rightskip\@pnumwidth plus1em \parfillskip-\@pnumwidth
    #5\leavevmode\hskip-\@tempdima{#6}\nobreak
    \leaders\hbox{$\m@th\mkern \@dotsep mu\hbox{.}\mkern \@dotsep mu$}\hfill
    \nobreak
    \hbox to\@pnumwidth{\@tocpagenum{\ifnum#1=0\bfseries\fi#7}}\par% <-- \bfseries for \chapter page
    \nobreak
    \endgroup
  \fi}
\AtBeginDocument{%
\expandafter\renewcommand\csname r@tocindent0\endcsname{0pt}
}
\def\l@subsection{\@tocline{2}{0pt}{2.5pc}{5pc}{}}
\def\l@subsubsection{\@tocline{2}{0pt}{4.5pc}{5pc}{}}
\makeatother

\advance\footskip0.4cm
\textheight=54pc    %a4paper
\textheight=50.5pc %letterpaper
\advance\textheight-0.4cm
\calclayout

%Font settings
%\usepackage{anyfontsize}
%Footnote settings
\usepackage{footmisc}
%	\renewcommand*{\thefootnote}{\fnsymbol{footnote}}
\usepackage{commath}
%Further math environments
%Further math fonts (loads amsfonts implicitely)
%Redefinition of \text
%\usepackage{amstext}
\usepackage{upref}
%Graphics
%\usepackage{graphicx}
%\usepackage{caption}
%\usepackage{subcaption}
%Frames
\usepackage{mdframed}
\allowdisplaybreaks
%\usepackage{interval}
\newcommand{\toup}{%
  \mathrel{\nonscript\mkern-1.2mu\mkern1.2mu{\uparrow}}%
}
\newcommand{\todown}{%
  \mathrel{\nonscript\mkern-1.2mu\mkern1.2mu{\downarrow}}%
}
\AtBeginDocument{\renewcommand*\d{\mathop{}\!\mathrm{d}}}
\renewcommand{\Re}{\operatorname{Re}}
\renewcommand{\Im}{\operatorname{Im}}
\DeclareMathOperator\Log{Log}
\DeclareMathOperator\Arg{Arg}
\DeclareMathOperator\id{id}
\DeclareMathOperator\sech{sech}
\DeclareMathOperator\Aut{Aut}
\DeclareMathOperator\h{h}
\DeclareMathOperator\sgn{sgn}
\DeclareMathOperator\arctanh{arctanh}
\DeclareMathOperator\supp{supp}
\DeclareMathOperator\ob{ob}
\DeclareMathOperator\mor{mor}
\DeclareMathOperator\M{M}
\DeclareMathOperator\dom{dom}
\DeclareMathOperator\cod{cod}
\DeclareMathOperator\im{im}
\DeclareMathOperator\Ab{Ab}
\DeclareMathOperator\coker{coker}
%\usepackage{hhline}
%\usepackage{booktabs} 
%\usepackage{array}
%\usepackage{xfrac} 
%\everymath{\displaystyle}
%Enumerate
\usepackage{tikz}
%\usepackgae{graphicx}
\usepackage{subcaption}
\usepackage{enumitem} 
%\renewcommand{\labelitemi}{$\bullet$}
%\renewcommand{\labelitemii}{$\ast$}
%\renewcommand{\labelitemiii}{$\cdot$}
%\renewcommand{\labelitemiv}{$\circ$}
%Colors
%\usepackage{color}
%\usepackage[cmtip, all]{xy}
%Main style theorem environment
\newtheoremstyle{main} 		             	 		%Stylename
  	{}	                                     		%Space above
  	{}	                                    		%Space below
  	{\itshape}			                     		%Body font
  	{}        	                             		%Indent
  	{\bfseries\boldmath}   	                         		%Head font
  	{.}            	                        		%Head punctuation
  	{ }           	                         		%Head space 
  	{\thmname{#1}\thmnumber{ #2}\thmnote{ (#3)}}	%Head specification
\theoremstyle{main}
\newtheorem{definition}{Definition}[chapter]
\newtheorem{proposition}{Proposition}[chapter]
\newtheorem{corollary}{Corollary}[chapter]
\newtheorem{theorem}{Theorem}[chapter]
\newtheorem{lemma}{Lemma}[chapter]
\newtheoremstyle{nonit} 		             	 		%Stylename
  	{}	                                     		%Space above
  	{}	                                    		%Space below
  	{}			                     		%Body font
  	{}        	                             		%Indent
  	{\bfseries\boldmath}   	                   		%Head font
  	{.}            	                        		%Head punctuation
  	{ }           	                         		%Head space 
  	{\thmname{#1}\thmnumber{ #2}\thmnote{ (#3)}}	%Head specification
\theoremstyle{nonit}
\newtheorem{remark}{Remark}[chapter]
\newtheorem{examples}{Examples}[chapter]
\newtheorem{example}{Example}[chapter]
\newtheorem{problem}{Problem}[chapter]
\newtheoremstyle{ex} 		             	 		%Stylename
  	{}	                                     		%Space above
  	{}	                                    		%Space below
  	{\small}			                     		%Body font
  	{}        	                             		%Indent
  	{\bfseries\boldmath}   	                         		%Head font
  	{.}            	                        		%Head punctuation
  	{ }           	                         		%Head space 
  	{\thmname{#1}\thmnumber{ #2}\thmnote{ (#3)}}	%Head specification
\theoremstyle{ex}
\newtheorem{exercise}{Exercise}[chapter]
%German non-ASCII-Characters
%Graphics-Tool
%\usepackage{tikz}
%\usepackage{tikzscale}
%\usepackage{bbm}
%\usepackage{bera}
%Listing-Setup
%Bibliographie
\usepackage[backend=bibtex, style=alphabetic]{biblatex}
%\usepackage[babel, german = swiss]{csquotes}
\bibliography{bibliography}
%PDF-Linking
%\usepackage[hyphens]{url}
\usepackage[bookmarksopen=true,bookmarksnumbered=true]{hyperref}
%\PassOptionsToPackage{hyphens}{url}\usepackage{hyperref}
\urlstyle{rm}
\hypersetup{
  colorlinks   = true, %Colours links instead of ugly boxes
  urlcolor     = blue, %Colour for external hyperlinks
  linkcolor    = blue, %Colour of internal links
  citecolor    = blue %Colour of citations
}
\newcommand{\bld}[1]{\boldmath\textit{\textbf{#1}}\unboldmath}
\newcommand{\eqclass}[1]{\sbr[0]{#1}}
\newcommand{\cat}[1]{\mathsf{#1}}
\newcommand{\Sbb}{\mathbb{S}}
\newcommand{\Zbb}{\mathbb{Z}}
\newcommand{\Nbb}{\mathbb{N}}
\newcommand{\Rbb}{\mathbb{R}}
\newcommand{\Hbb}{\mathbb{H}}
\newcommand{\Cbb}{\mathbb{C}}
\newcommand{\Tcal}{\mathcal{T}}
\newcommand{\SLrm}{\mathrm{SL}}
\newcommand{\PSLrm}{\mathrm{PSL}}
\newcommand{\SLrmstar}{\mathrm{S^*L}}
\newcommand{\PSLrmstar}{\mathrm{PS^*L}}
\newcommand{\GLrm}{\mathrm{GL}}
\newcommand{\Mrm}{\mathrm{M}}
\newcommand{\Isom}{\mathrm{Isom}}
\newcommand{\Mob}{\mathrm{M\ddot{o}b}}
\newcommand{\Ebb}{\mathbb{E}}
\newcommand{\Cscr}{\mathscr{C}}
\newcommand{\pwrm}{\mathrm{pw}}
\newcommand{\clos}[1]{\overline{#1}}
\newcommand{\Hcal}{\mathcal{H}}
\newcommand{\Hbcal}{\bm{\mathcal{H}}}
\newcommand{\Ucal}{\mathcal{U}}
\newcommand{\Ubcal}{\bm{\mathcal{U}}}
\renewcommand{\det}{\mathrm{det}}
\newcommand{\ab}{\mathrm{ab}}


\title{Solutions Sheet 7}
\author{Yannis B\"{a}hni}
\address[Yannis B\"{a}hni]{University of Zurich, R\"{a}mistrasse 71, 8006 Zurich}
\email[Yannis B\"{a}hni]{\href{mailto:yannis.baehni@uzh.ch}{\nolinkurl{yannis.baehni@uzh.ch}}}

\begin{document}

\maketitle
\thispagestyle{fancy}

\setcounter{section}{1}

\begin{enumerate}[label = \textbf{Exercise \arabic*.},wide = 0pt, itemsep = 1.5ex]
	\item
		\begin{lemma}
			$q : X \to \Rbb$ is a sublinear functional and $f \leq q$ on $Y$.
		\end{lemma}

		\begin{proof}
			Let $\lambda \geq 0$. Moreover, let $\cbr{A_1,\dots,A_n} \subseteq G$ for some $n \in \Nbb$. Then for any $x \in X$ we have that
			\begin{align*}
				\frac{1}{n} p\del[1]{A_1(\lambda x) + \dots + A_n(\lambda x)} &= \frac{1}{n} p\del[1]{\lambda A_1(x) + \dots + \lambda A_n(x)}\\
				&= \frac{1}{n} p\del[1]{\lambda \del[1]{A_1(x) + \dots + A_n(x)}}\\
				&= \lambda\frac{1}{n} p(A_1(x) + \dots + A_n(x))
			\end{align*}
			\noindent since each $A_i$ is linear and $p$ is a sublinear functional on $X$. Thus also $q(\lambda x) = \lambda q(x)$. Let $x,y \in X$. Furthermore, fix some $\varepsilon > 0$. By definition of the infimum, we find $A_1,\dots,A_n \in G$ and $B_1,\dots,B_m \in G$, such that
			\begin{equation*}
				q(x) \leq \frac{1}{n}p\del[1]{A_1(x) + \dots + A_n(x)} \leq q(x) + \frac{\varepsilon}{2}
			\end{equation*}
			\noindent and
			\begin{equation*}
				q(y) \leq \frac{1}{m}p\del[1]{B_1(y) + \dots + B_m(y)} \leq q(y) + \frac{\varepsilon}{2}.
			\end{equation*}
			We estimate
			\begin{align*}
				\frac{1}{n}p\del[1]{A_1(x) + \dots + A_n(x)} &= \frac{m}{mn}p\del[1]{A_1(x) + \dots + A_n(x)}\\
				&= \frac{1}{mn} \sum_{k = 1}^m p\del[1]{A_1(x) + \dots + A_n(x)}\\
				&\geq \frac{1}{mn} \sum_{k = 1}^m p\del[1]{B_k\del[1]{A_1(x) + \dots + A_n(x)}}\\
				&= \frac{1}{mn} \sum_{k = 1}^m p\del[1]{B_kA_1(x) + \dots + B_kA_n(x)}
			\end{align*}
			\noindent by the linearity of elements in $G$, the closedness of $G$ under composition and the property that $p(Ax) \leq p(x)$ holds for all $x \in X$ and $A \in G$. Similarly we estimate
			\begin{align*}
				\frac{1}{m}p\del[1]{B_1(y) + \dots + B_m(y)} &= \frac{n}{mn}p\del[1]{B_1(y) + \dots + B_m(y)}\\
				&= \frac{1}{mn} \sum_{k = 1}^n p\del[1]{B_1(y) + \dots + B_m(y)}\\
				&\geq \frac{1}{mn} \sum_{k = 1}^n p\del[1]{A_k\del[1]{B_1(y) + \dots + B_m(y)}}\\
				&= \frac{1}{mn} \sum_{k = 1}^n p\del[1]{A_kB_1(y) + \dots + A_kB_m(y)}.
			\end{align*}
			Hence the sublinearity of $p$ together with the commutativity of $G$ yields
			\begin{align*}
				q(x) + q(y) + \varepsilon \geq &  \frac{1}{n}p\del[1]{A_1(x) + \dots + A_n(x)} + \frac{1}{m}p\del[1]{B_1(y) + \dots + B_m(y)}\\
				\geq & \frac{1}{mn} \sum_{k = 1}^m p\del[1]{B_kA_1(x) + \dots + B_kA_n(x)}\\ &+ \frac{1}{mn} \sum_{k = 1}^n p\del[1]{A_kB_1(y) + \dots + A_kB_m(y)}\\
				=& \frac{1}{mn} \del{\sum_{k = 1}^m p\del{\sum_{\ell = 1}^nB_kA_\ell(x)} + \sum_{k = 1}^n p\del{\sum_{\ell = 1}^m A_kB_\ell(y)}}\\
				\geq& \frac{1}{mn} \del{p\del{\sum_{k = 1}^m\sum_{\ell = 1}^nB_kA_\ell(x)} + p\del{\sum_{k = 1}^n\sum_{\ell = 1}^m A_kB_\ell(y)}}\\
				\geq& \frac{1}{mn} p\del{\sum_{k = 1}^m\sum_{\ell = 1}^nB_kA_\ell(x) + \sum_{k = 1}^n\sum_{\ell = 1}^m A_kB_\ell(y)}\\
				=& \frac{1}{mn} p\del{\sum_{k = 1}^m\sum_{\ell = 1}^nB_kA_\ell(x) + \sum_{k = 1}^n\sum_{\ell = 1}^m B_\ell A_k(y)}\\
				=& \frac{1}{mn} p\del{\sum_{k = 1}^m\sum_{\ell = 1}^nB_kA_\ell(x) + \sum_{\ell = 1}^m\sum_{k = 1}^n B_\ell A_k(y)}\\
				=& \frac{1}{mn} p\del{\sum_{k = 1}^m\sum_{\ell = 1}^nB_kA_\ell(x) + \sum_{k = 1}^m\sum_{\ell = 1}^n B_k A_\ell(y)}\\
				=& \frac{1}{mn} p\del{\sum_{k = 1}^m\sum_{\ell = 1}^n\del[1]{B_kA_\ell(x) + B_kA_\ell(y)}}\\
				=& \frac{1}{mn} p\del{\sum_{k = 1}^m\sum_{\ell = 1}^nB_kA_\ell(x + y)}\\
				\geq& q(x + y).
			\end{align*}
			Since $\varepsilon$ was arbitrary, we conclude that
			\begin{equation*}
				q(x + y) \leq q(x) + q(y)
			\end{equation*}
			\noindent holds for all $x,y \in X$. Lastly, we show that $f \leq q$ on $Y$. Let $y \in Y$. Moreover, let $A_1,\dots,A_n \in G$. Then
			\begin{align*}
				f(y) &= \frac{1}{n}nf(y)\\
				&= \frac{1}{n}\sum_{k = 1}^n f(y)\\
				&= \frac{1}{n}\sum_{k = 1}^n f(A_ky)\\
				&= \frac{1}{n}f(A_1y + \dots + A_ny)\\
				&\leq \frac{1}{n}p(A_1y + \dots + A_ny)
			\end{align*}
			\noindent by the assumption that $f(Ay) = f(y)$, $Ay \in Y$, $f$ is linear and $f \leq p$ on $Y$ for all $y \in Y$ and $A \in G$. Taking the infimum over all finite sets of $G$ finally yields the result. 
		\end{proof}

		An application of \emph{Hahn-Banach} now yields the existence of a linear mapping $F : X \to \Rbb$ with $F\vert_Y = f$ and $F \leq q$ on $X$. Observe, that $q \leq p$ simply by choosing the finite set to be $\cbr{\id_X}$. Hence $F \leq p$ on $X$. Thus we have to show a final lemma.

		\begin{lemma}
			$\forall x \in X\forall A \in G\del[1]{F(Ax) = F(x)}$.
		\end{lemma}

		\begin{proof}
			Fix $x \in X$ and $A \in G$. Let $n \in \Nbb$ and consider $\cbr{\id_X,A,A^2,\dots,A^{n-1}} \subseteq G$. On one hand we have that 
			\begin{align*}
				F(Ax) - F(x) &= F(Ax - x)\\
				&\leq q(Ax - x)\\
				&\leq \frac{1}{n} p\del{\sum_{k = 0}^{n - 1} A^k(Ax - x)}\\
				&= \frac{1}{n} p\del{\sum_{k = 0}^{n - 1} \del[1]{A^{k+1}x - A^kx}}\\
				&= \frac{1}{n}p(A^nx - x)\\
				&\leq \frac{1}{n}\del{p(A^nx) + p(-x)}\\
				&\leq \frac{1}{n}\del{p(x) + p(-x)}
			\end{align*}
			\noindent and on the other
			\begin{align*}
				F(Ax) - F(x) &= F(Ax - x)\\
				&= -F(x - Ax)\\
				&\geq -q(x - Ax)\\
				&\geq -\frac{1}{n} p\del{\sum_{k = 0}^{n - 1} A^k(x - Ax)}\\
				&= -\frac{1}{n} p\del{\sum_{k = 0}^{n - 1} \del[1]{A^kx - A^{k+1}x}}\\
				&= -\frac{1}{n}p(x - A^nx)\\
				&\geq -\frac{1}{n}\del{p(x) + p(-A^nx)}\\
				&= -\frac{1}{n} \del{p(x) + p(A^n(-x))}\\
				&\geq -\frac{1}{n}\del{p(x) + p(-x)}.
			\end{align*}
			Hence
			\begin{equation*}
				\abs{F(Ax) - F(x)} \leq \frac{1}{n}\del{p(x) + p(-x)}.
			\end{equation*}
			Since $n \in \Nbb$ was arbitrary, we conclude that
			\begin{equation*}
				\abs{F(Ax) - F(x)} = 0
			\end{equation*}
			\noindent and thus
			\begin{equation*}
				F(Ax) = F(x).
			\end{equation*}
		\end{proof}

	\item See separate sheet.
	\item 
		\begin{lemma}
			Let $y \in H$ and define a mapping $\varphi_y : H \to \Kbb$ by $\varphi_y(x) := \langle A(y),x \rangle$. Then $\varphi_y \in \Lcal(H,\Kbb)$.
			\label{lem:family}
		\end{lemma}

		\begin{proof}
			Clearly, $\varphi_y$ is linear since $\langle \cdot,\cdot \rangle$ is linear in the second component. Moreover, $\varphi_y$ is bounded. Indeed, using Cauchy-Schwarz yields
			\begin{equation*}
				\abs[0]{\varphi_y(x)} = \abs[0]{\langle A(y), x \rangle} \leq \norm{A(y)} \norm{x}
			\end{equation*}
			\noindent for all $x \in H$.
		\end{proof}

		Thus we may define a family
		\begin{equation*}
			\Fcal := \cbr{\varphi_y : y \in \partial B_1(0)} \subseteq \Lcal(H,\Kbb).
		\end{equation*}

		Let $x \in H$. Then for any $y \in \partial B_1(0)$ we have that
		\begin{equation*}
			\abs[0]{\varphi_y(x)} = \abs[0]{\langle A(y),x \rangle} = \abs[0]{\langle y,A(x) \rangle} \leq \norm{y} \norm{A(x)} = \norm{A(x)}
		\end{equation*}
		\noindent by symmetry and again Cauchy-Schwarz. Hence
		\begin{equation*}
			\sup_{T \in \Fcal}\abs[0]{T(x)} = \sup_{y \in \partial B_1(0)} \abs[0]{\varphi_y(x)} \leq \norm{A (x)}
		\end{equation*}
		\noindent for all $x \in H$. Since any Hilbert space is a Banach space, an application of \emph{Banach-Steinhaus} yields the existence of a constant $c > 0$ such that
		\begin{equation*}
			\sup_{T \in \Fcal}\norm[0]{T} = \sup_{y \in \partial B_1(0)} \norm[0]{\varphi_y} \leq c.
		\end{equation*}
		For $x \in H$ such that $A(x) \neq 0$ we have that
		\begin{align*}
			\norm{A(x)}^2 &= \langle A(x),A(x) \rangle\\
			&= \norm{x} \langle A(x / \norm{x}), A(x)\rangle\\
			&= \norm{x} \varphi_{x/\norm{x}}(A(x))\\
			&\leq \norm{x} \abs[0]{\varphi_{x/\norm{x}}(A(x))}\\
			&\leq \norm{x} \norm{A(x)} \norm[0]{\varphi_{x/\norm{x}}}\\
			&\leq c\norm{x} \norm{A(x)}
		\end{align*}
		\noindent and thus dividing both sides by $\norm{A(x)}$ yields the boundedness of $A$.

	\item

	\item
		~
		\begin{enumerate}[label = \textbf{\alph*.},wide = 0pt, itemsep = 1.5ex]
			\item We define
				\begin{equation*}
					\Fcal := \cbr{B(\cdot,y) : y \in \partial B_1(0)}.
				\end{equation*}

				\begin{lemma}
					We have that $\Fcal \subseteq \Lcal(X,\Kbb)$ and for all $x \in X$, there exists $c_x \geq 0$ such that $\sup_{T \in \Fcal}\abs{T(x)} \leq c_x$.
				\end{lemma}

				\begin{proof}
					Let $y \in \partial B_1(0)$. Then $B(\cdot,y)$ is linear by definition of a bilinear functional. Moreover, for any $x \in X$ we have that
					\begin{equation*}
						\abs{B(x,y)} \leq c_y \norm{x}
					\end{equation*}
					\noindent for some $c_y \geq 0$ by continuity of $B$ in the first argument. Hence $\Fcal \subseteq \Lcal(X,\Kbb)$. Let $x \in X$. Then
					\begin{equation*}
						\abs{B(x,y)} \leq c_x \norm{y} = c_x
					\end{equation*}
					\noindent for some $c_x \geq 0$ by continuity of $B$ in the second argument. Thus
					\begin{equation*}
						\sup_{T \in \Fcal}\abs{T(x)} = \sup_{y \in \partial B_1(0)} \abs{B(x,y)} \leq c_x
					\end{equation*}
					\noindent for all $x \in X$.
				\end{proof}

				An application of \emph{Banach-Steinhaus} on the family $\Fcal$ yields the existence of a constant $c \geq 0$ such that
				\begin{equation*}
					\sup_{T \in \Fcal}\norm{T} \leq c.
				\end{equation*}
				Let $x,y \in X$. Then
				\begin{align*}
					\abs{B(x,y)} &= \norm{x}\norm{y} \abs[0]{B(x/\norm{x},y/\norm{y})}\\
					&\leq \norm{x}\norm{y} \sup_{\norm[0]{\xi} = 1}\abs[0]{B(\xi,y/\norm{y})}\\
					&\leq \norm{x}\norm{y} \sup_{\norm[0]{\zeta} = 1} \sup_{\norm[0]{\xi} = 1} \abs{B(\xi,\zeta)}\\
					&= \norm{x}\norm{y} \sup_{\norm[0]{\zeta} = 1}\norm{B(\cdot,\zeta)}\\
					&\leq c\norm{x}\norm{y}.
				\end{align*}

				\begin{lemma}
					Equip $X \times X$ with the norm $\norm{(x,y)} := \norm{x} + \norm{y}$. Then $B$ is continuous.
				\end{lemma}

				\begin{proof}
					Let $(x,y) \in X \times X$ and $(x_n,y_n)_{n \in \Nbb}$ be a sequence in $X \times X$ converging to $(x,y)$. We claim that $x_n \to x$ and $y_n \to y$ in $X$. Indeed
					\begin{equation*}
						\norm{x_n - x} \leq \norm{x_n - x} + \norm{y_n - y} = \norm{(x_n,y_n) - (x,y)} \to 0
					\end{equation*}
					\noindent as $n \to \infty$ and similarly
					\begin{equation*}
						\norm{y_n - y} \leq \norm{x_n - x} + \norm{y_n - y} = \norm{(x_n,y_n) - (x,y)} \to 0.
					\end{equation*}
					Moreover, since $y_n \to y$, $y_n$ is bounded, i.e. there exists some $M \geq 0$ such that $\norm{y_n} \leq M$ for all $n \in \Nbb$. Hence
					\begin{align*}
						\abs{B(x_n,y_n) - B(x,y)} &= \abs{B(x_n,y_n) - B(x,y_n) + B(x,y_n) - B(x,y)}\\
						&= \abs{B(x_n - x,y_n) + B(x,y_n - y)}\\
						&\leq \abs{B(x_n - x,y_n)} + \abs{B(x,y_n - y)}\\
						&\leq c \norm{x_n - x}\norm{y_n} + c \norm{x}\norm{y_n - y}\\
						&\leq cM \norm{x_n - x} + c\norm{x}\norm{y_n - y} \rightarrow 0
					\end{align*}
					\noindent as $n \to \infty$.
				\end{proof}
			\item
				~
				\begin{lemma}
					$B$ is a bilinear functional on $\cal{P}$ which is continuous in each argument separately.
				\end{lemma}

				\begin{proof}
					The bilinearity of $B$ directly follows from the linearity of the integral. Fix $q \in \cal{P}$. Then for any $p \in \cal{P}$ we have that
					\begin{equation*}
						\abs{B(p,q)} = \abs[3]{\int_0^1 p(t)q(t)dt} \leq \int_0^1 \abs{p(t)}\abs{q(t)}dt \leq \sup_{t \in \intcc{0,1}}\abs{q(t)} \int_0^1 \abs{p(t)}dt = c_q \norm{p}
					\end{equation*}
					\noindent since $q$ is continuous. Similarly, for each fixed $p \in \cal{P}$ we get that $\abs{B(p,q)} \leq c_p \norm{q}$ for all $q \in \cal{P}$.
				\end{proof}
		\end{enumerate}
\end{enumerate}

\end{document}
