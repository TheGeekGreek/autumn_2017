%%%%%%%%%%%%%%%%%%%%%%%%%%%%%%%%%%%%%%%%%%%%%%%%%%%%%%%%%%%%%%%%%%%%%%%%%%
%Author:																 %
%-------																 %
%Yannis Baehni at University of Zurich									 %
%baehni.yannis@uzh.ch													 %
%																		 %
%Version log:															 %
%------------															 %
%06/02/16 . Basic structure												 %
%04/08/16 . Layout changes including section, contents, abstract.		 %
%%%%%%%%%%%%%%%%%%%%%%%%%%%%%%%%%%%%%%%%%%%%%%%%%%%%%%%%%%%%%%%%%%%%%%%%%%

%Page Setup
\documentclass[
	12pt, 
	oneside, 
	a4paper,
	reqno,
	final
]{amsart}

\usepackage[
	left = 3cm, 
	right = 3cm, 
	top = 3cm, 
	bottom = 3cm
]{geometry}

%Headers and footers
\usepackage{fancyhdr}
	\pagestyle{fancy}
	%Clear fields
	\fancyhf{}
	%Header right
	\fancyhead[R]{
		\footnotesize
		Yannis B\"{a}hni\\
		\href{mailto:yannis.baehni@uzh.ch}{yannis.baehni@uzh.ch}
	}
	%Header left
	\fancyhead[L]{
		\footnotesize
		401-3581-67L Symplectic Geometry\\
		Autumn 2017
	}
	%Page numbering in footer
	\fancyfoot[C]{\thepage}
	%Separation line header and footer
	\renewcommand{\headrulewidth}{0.4pt}
	%\renewcommand{\footrulewidth}{0.4pt}
	
	\setlength{\headheight}{19pt} 

%Title
\usepackage[foot]{amsaddr}
\usepackage{upref}
\usepackage{newtxtext}
\usepackage[subscriptcorrection,nofontinfo,mtpcal,mtphrb]{mtpro2}
\usepackage{bm}
\usepackage{xspace}
\makeatletter
\usepackage{etoolbox}
\patchcmd{\abstract}{\scshape\abstractname}{\textbf{\abstractname}}{}{}

\usepackage[all,cmtip]{xy}

%Section, subsection and subsubsection font
%------------------------------------------
\makeatletter
	\renewcommand{\@secnumfont}{\bfseries}
	\renewcommand\section{\@startsection{section}{1}%
  	\z@{.7\linespacing\@plus\linespacing}{.5\linespacing}%
  	{\normalfont\bfseries\centering}}
	\renewcommand\subsection{\@startsection{subsection}{2}%
    	\z@{.5\linespacing\@plus.7\linespacing}{-.5em}%
    	{\normalfont\bfseries}}%
	\renewcommand\subsubsection{\@startsection{subsubsection}{3}%
    	\z@{.5\linespacing\@plus.7\linespacing}{-.5em}%
    	{\normalfont\bfseries}}%
%Formatting title of TOC
\renewcommand{\contentsnamefont}{\bfseries}
%Table of Contents
\setcounter{tocdepth}{3}

% Add bold to \section titles in ToC and remove . after numbers
\renewcommand{\tocsection}[3]{%
  \indentlabel{\@ifnotempty{#2}{\bfseries\ignorespaces#1 #2\quad}}\bfseries#3}
% Remove . after numbers in \subsection
\renewcommand{\tocsubsection}[3]{%
  \indentlabel{\@ifnotempty{#2}{\ignorespaces#1 #2\quad}}#3}
\let\tocsubsubsection\tocsubsection% Update for \subsubsection
%...

\newcommand\@dotsep{4.5}
\def\@tocline#1#2#3#4#5#6#7{\relax
  \ifnum #1>\c@tocdepth % then omit
  \else
    \par \addpenalty\@secpenalty\addvspace{#2}%
    \begingroup \hyphenpenalty\@M
    \@ifempty{#4}{%
      \@tempdima\csname r@tocindent\number#1\endcsname\relax
    }{%
      \@tempdima#4\relax
    }%
    \parindent\z@ \leftskip#3\relax \advance\leftskip\@tempdima\relax
    \rightskip\@pnumwidth plus1em \parfillskip-\@pnumwidth
    #5\leavevmode\hskip-\@tempdima{#6}\nobreak
    \leaders\hbox{$\m@th\mkern \@dotsep mu\hbox{.}\mkern \@dotsep mu$}\hfill
    \nobreak
    \hbox to\@pnumwidth{\@tocpagenum{\ifnum#1=1\bfseries\fi#7}}\par% <-- \bfseries for \section page
    \nobreak
    \endgroup
  \fi}
\AtBeginDocument{%
\expandafter\renewcommand\csname r@tocindent0\endcsname{0pt}
}
\def\l@subsection{\@tocline{2}{0pt}{2.5pc}{5pc}{}}
\def\l@subsubsection{\@tocline{2}{0pt}{4.5pc}{5pc}{}}
\makeatother

\advance\footskip0.4cm
\textheight=54pc    %a4paper
\textheight=50.5pc %letterpaper
\advance\textheight-0.4cm
\calclayout

%Font settings
%\usepackage{anyfontsize}
%Footnote settings
%\usepackage{mathptmx}
\usepackage{footmisc}
%	\renewcommand*{\thefootnote}{\fnsymbol{footnote}}
\usepackage{commath}
%Further math environments
%Further math fonts (loads amsfonts implicitely)
%Redefinition of \text
%\usepackage{amstext}
\usepackage{upref}
%Graphics
%\usepackage{graphicx}
%\usepackage{caption}
%\usepackage{subcaption}
%Frames
\usepackage{mdframed}
\allowdisplaybreaks
%\usepackage{interval}
\newcommand{\toup}{%
  \mathrel{\nonscript\mkern-1.2mu\mkern1.2mu{\uparrow}}%
}
\newcommand{\todown}{%
  \mathrel{\nonscript\mkern-1.2mu\mkern1.2mu{\downarrow}}%
}
\AtBeginDocument{\renewcommand*\d{\mathop{}\!\mathrm{d}}}
\renewcommand{\Re}{\operatorname{Re}}
\renewcommand{\Im}{\operatorname{Im}}
\DeclareMathOperator\Log{Log}
\DeclareMathOperator\Arg{Arg}
\DeclareMathOperator\sech{sech}
\DeclareMathOperator*\esssup{ess.sup}
\DeclareMathOperator\id{id}
\DeclareMathOperator\im{im}
\DeclareMathOperator\Vol{Vol}
\DeclareMathOperator\dist{dist}
%\usepackage{hhline}
%\usepackage{booktabs} 
%\usepackage{array}
%\usepackage{xfrac} 
%\everymath{\displaystyle}
%Enumerate
\usepackage{tikz}
\usetikzlibrary{external}
\tikzexternalize % activate!
\usetikzlibrary{patterns}
\pgfdeclarepatternformonly{adjusted lines}{\pgfqpoint{-1pt}{-1pt}}{\pgfqpoint{40pt}{40pt}}{\pgfqpoint{39pt}{39pt}}%
{
  \pgfsetlinewidth{.8pt}
  \pgfpathmoveto{\pgfqpoint{0pt}{0pt}}
  \pgfpathlineto{\pgfqpoint{39.1pt}{39.1pt}}
  \pgfusepath{stroke}
}
\usepackage{enumitem} 
%\renewcommand{\labelitemi}{$\bullet$}
%\renewcommand{\labelitemii}{$\ast$}
%\renewcommand{\labelitemiii}{$\cdot$}
%\renewcommand{\labelitemiv}{$\circ$}
%Colors
%\usepackage{color}
%\usepackage[cmtip, all]{xy}
%Theorems
\newtheoremstyle{main} 		             	 		%Stylename
  	{}	                                     		%Space above
  	{}	                                    		%Space below
  	{\itshape}			                     		%Body font
  	{}        	                             		%Indent
  	{\bfseries}   	                         		%Head font
  	{.}            	                        		%Head punctuation
  	{ }           	                         		%Head space 
  	{\thmname{#1}\thmnumber{ #2}\thmnote{ (#3)}}	%Head specification
\theoremstyle{main}
\newtheorem{definition}{Definition}[section]
\newtheorem{proposition}{Proposition}[section]
\newtheorem{corollary}{Corollary}[section]
\newtheorem{theorem}{Theorem}[section]
\newtheorem{lemma}{Lemma}[section]
%Roman style theorems
\newtheoremstyle{roman}
	{}
	{}
  	{}
  	{}
	{\bfseries}
	{.}
  	{ }
	{\thmname{#1}\thmnumber{ #2}\thmnote{ (#3)}}
\theoremstyle{roman}
\newtheorem{example}{Example}[section]
\newtheorem{solution}{Solution}[section]
\newtheorem{remark}{Remark}[section]
%Exercise style theorems
\newtheoremstyle{exercise}
  	{}
  	{}
  	{\small}
  	{}
  	{\bfseries}
  	{.}
 	{ }
  	{\thmname{#1}\thmnumber{ #2}\thmnote{ (#3)}}
\theoremstyle{exercise}
\newtheorem{exercise}{Exercise}[section]
%Changing default style of proof environment
\renewcommand*{\proofname}{\itshape Proof}
%German non-ASCII-Characters
%Graphics-Tool
%\usepackage{tikz}
%\usepackage{tikzscale}
%\usepackage{bbm}
%\usepackage{bera}
%Listing-Setup
%Bibliographie
\usepackage[backend=bibtex, style=alphabetic]{biblatex}
%\usepackage[babel, german = swiss]{csquotes}
\bibliography{../latex/bibliography}
%PDF-Linking
%\usepackage[hyphens]{url}
\usepackage[bookmarksopen=true,bookmarksnumbered=true]{hyperref}
%\PassOptionsToPackage{hyphens}{url}\usepackage{hyperref}
\hypersetup{
  colorlinks   = true, %Colours links instead of ugly boxes
  urlcolor     = blue, %Colour for external hyperlinks
  linkcolor    = blue, %Colour of internal links
  citecolor    = blue %Colour of citations
}
%Weierstrass-P symbol for power set
\newcommand{\powerset}{\raisebox{.15\baselineskip}{\Large\ensuremath{\wp}}}
\newcommand{\bld}[1]{\boldmath\textit{\textbf{#1}}\unboldmath}
\usepackage{pict2e}
\makeatletter
\DeclareRobustCommand{\intprod}{%
	\mathbin{\mathpalette\int@prod{(0.1,0)(0.9,0)(0.9,0.8)}}%
}
\newcommand{\int@prod}[2]{%
	\begingroup
	\sbox\z@{$\m@th#1+$}%
	\setlength\unitlength{\wd\z@}%
	\begin{picture}(1,1)
	\roundcap
	\polyline#2
	\end{picture}%
	\endgroup
}
\makeatother
\newcommand{\Sbb}{\mathbb{S}}
\newcommand{\dRrm}{\mathrm{dR}}
\newcommand{\Rbb}{\mathbb{R}}
\newcommand{\Lcal}{\mathcal{L}}
\newcommand{\Zbb}{\mathbb{Z}}
\newcommand{\Nbb}{\mathbb{N}}
\newcommand{\Xfrak}{\mathfrak{X}}


\title{Solutions Sheet 1}
\author{Yannis B\"{a}hni}
\address[Yannis B\"{a}hni]{University of Zurich, R\"{a}mistrasse 71, 8006 Zurich}
\email[Yannis B\"{a}hni]{\href{mailto:yannis.baehni@uzh.ch}{yannis.baehni@uzh.ch}}

\begin{document}
\maketitle
\thispagestyle{fancy}

\begin{enumerate}[label = \textbf{Exercise \arabic*.},wide = 0pt, itemsep=1.5ex]
\item ~ 
	\begin{enumerate}[label = \textbf{\alph*.},wide = 0pt, itemsep=1.5ex]
		\item The first part can be shown for an arbitrary set $X$. Clearly $\varnothing,X \in \Tcal$ since $X^c = \varnothing$ is countable. Let $(U_\iota)_{\iota \in I}$ be a family of sets in $\Tcal$. If $U_\iota = \varnothing$ for all $\iota \in I$ we have that $\cup_{\iota \in I} U_\iota = \varnothing \in \Tcal$. So assume that $U_{\iota_0} \neq \varnothing$ for some $\iota_0 \in I$. But then $U^c_{\iota_0}$ is countable, and so is $(\cup_{\iota \in I} U_\iota)^c = \cap_{\iota \in I}U_\alpha^c \subseteq U^c_{\iota_0}$. Lastly, let $U_1,\dots,U_n \in \Tcal$ for $n\in \mathbb{Z}$, $n \geq 1$. If $U_\iota = \varnothing$ for some $\iota$, then $\cap_{\iota = 1}^n U_\iota = \varnothing$ and thus $\cap_{\iota = 1}^n U_\iota \in \Tcal$. So assume that $U_\iota \neq \varnothing$ for $\iota = 1,\dots,n$. Then $(\cap_{\iota = 1}^n U_\iota)^c = \cup_{\iota = 1}^n U_\iota^c$ which is a finite union of countable sets, which is countable. Hence $\Tcal$ is indeed a topology on $X$.\\
			We claim that $(X,\Tcal)$ is not Hausdorff when $X$ is uncountable. Towards a contradiction assume that $(X,\Tcal)$ is Hausdorff. Let $p,q \in X$ with $p \neq q$. Hence there exist (open) neighbourhoods $U$ and $V$ of $p$ and $q$ respectively such that $U \cap V = \varnothing$. Now $X = U \cup U^c$, where $U^c$ is countable and clearly nonempty. But $U \cap V = \varnothing$ implies $U \subseteq V^c$ which therefore yields that $U$ is also countable. Hence $X$ is a union of two countable sets and thus countable. Contradiction.
		\item We prove both times the contrapositive. Assume that there is a family $(A_\iota)_{\iota \in I}$ of closed subsets of $X$ having the finite intersection property such that $\cap_{\iota \in I} A_\iota = \varnothing$. Then $\cup_{\iota \in I} A^c_\iota = (\cap_{\iota \in I} A_\iota)^c = X$. Since each $A_\iota$ is closed, $A_\iota^c$ is open for all $\iota \in I$ and thus $(A^c_\iota)_{\iota \in I}$ is an open cover for $X$. We claim that $(A_\iota^c)_{\iota \in I}$ does not admit any finite subcover. Towards a contradiction, assume that it does. Hence we find $\iota_1,\dots,\iota_n \in I$, $n \in \Zbb$, $n \geq 1$, such that $\cup_{k = 1}^n A_{\iota_k}^c = X$. But then $\cap_{k = 1}^n A_{\iota_k} = \varnothing$, contradicting the finite intersection property of the family $(A_\iota)_{\iota \in I}$.\\
			Conversly, suppose that there exists an open cover $(A_\iota)_{\iota \in I}$ of $X$ which does not admit a finite subcover. We claim that the closed family $(A_\iota^c)_{\iota \in I}$ has the finite intersection property and $\cap_{\iota \in I}A_\iota = \varnothing$. Let $\iota_1,\dots,\iota_n \in I$, $n \in \Zbb$, $n \geq 1$. Since $(A_{\iota_k})_{k = 1}^n$ cannot cover $X$, otherwise it would be a finite subcover of $(A_\iota)_{\iota \in I}$, we have that $\cap_{k = 1}^n A_{\iota_k}^c \neq \varnothing$. Thus $(A_\iota^c)_{\iota \in I}$ has the finite intersection property. Since $(A_\iota)_{\iota \in I}$ covers $X$ we have that $\cap_{\iota \in I} A_\iota^c = \varnothing$.
	\end{enumerate}
\item
	~
	\begin{enumerate}[label = \textbf{\alph*.},wide = 0pt, itemsep=1.5ex]
		\item Clearly, $\varnothing,X \in \Tcal_d$. Let $(U_\iota)_{\iota \in I}$ be a family of elements in $\Tcal_d$ and $x \in \cup_{\iota \in I}U_\iota$. Then there exists $\iota \in I$ such that $x \in U_\iota$. Furthermore, we find $\varepsilon > 0$ such that $B_\varepsilon(x) \subseteq U_\iota$. Hence $B_\varepsilon(x) \subseteq \cup_{\iota \in I} U_\iota$. Let $U_1,\dots,U_n \in \Tcal$ for $n\in \mathbb{Z}$, $n \geq 1$, and $x \in \cap_{\iota = 1}^n U_\iota$. Hence there exist $\varepsilon_1,\dots,\varepsilon_n > 0$ such that $B_{\varepsilon_\iota}(x) \subseteq U_\iota$ for $\iota = 1,\dots,n$ and so $B_{\wtilde{\varepsilon}}(x) \subseteq \cap_{\iota = 1}^n U_\iota$ for $\wtilde{\varepsilon} := \min\cbr[0]{\varepsilon_1,\dots,\varepsilon_n}$. Thus $\Tcal_d$ is a topology on $X$. 
		\item We will use the fact that two metrics induce the same topology if and only if they induce the same convergence. Let $\wtilde{M} := \intoo{0,\infty}$. Define $f : \wtilde{M} \to \wtilde{M}$ by $f(x) := 1/x$. Then clearly $d_2 = \wtilde{d_2}\vert_M$ and $d_1 = \wtilde{d_1}\vert_M$, where 
		\begin{equation*}
			\wtilde{d_2} : \wtilde{M} \times \wtilde{M} \xrightarrow{f \times f}{} \wtilde{M} \times \wtilde{M} \xrightarrow{\abs[0]{\cdot,\cdot}}{} \Rbb
		\end{equation*}
		\noindent and
		\begin{equation*}
			\wtilde{d_1} : \wtilde{M} \times \wtilde{M} \xrightarrow{f \times f}{} \wtilde{M} \times \wtilde{M} \xrightarrow{\wtilde{d_2}}{} \Rbb.
		\end{equation*}
		It is easy to show that $\wtilde{d_2}$ is a metric. Let $x \in M$ and $(x_n)_{n \in \mathbb{N}}$ be a sequence in $M$. Assume that $x_n \xrightarrow{d_1}{} x$. Then 
		\begin{equation*}
			d_2(x_n,x) = \wtilde{d_1}\del[1]{f(x_n),f(x)} \to 0
		\end{equation*}
		\noindent and
		\begin{equation*}
			d_1(x_n,x) = \wtilde{d_2}\del[1]{f(x_n),f(x)} \to  0
		\end{equation*}
		\noindent by the continuity of $f$ on $\wtilde{M}$.\\
		$(M,d_1)$ is complete since $M$ is a closed subset of the complete metric space $\Rbb$. Consider the sequence $(n)_{n \in \mathbb{N}}$ in $M$. Clearly, it is a Cauchy sequence in $(M,d_2)$ since $\frac{1}{n} \xrightarrow{\abs[0]{\cdot}}{} 0$. Assume that it converges also in $(M,d_2)$. Since the induced topologies of $d_1$ and $d_2$ are the same, we would get that $(n)_{n \in \Nbb}$ also converges in $(M,d_1)$. But this is absurd. Hence $(M,d_2)$ cannot be complete. 
	\end{enumerate}
\end{enumerate}
\printbibliography
\end{document}
