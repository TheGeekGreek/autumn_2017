%%%%%%%%%%%%%%%%%%%%%%%%%%%%%%%%%%%%%%%%%%%%%%%%%%%%%%%%%%%%%%%%%%%%%%%%%%
%Author:																 %
%-------																 %
%Yannis Baehni at University of Zurich									 %
%baehni.yannis@uzh.ch													 %
%																		 %
%Version log:															 %
%------------															 %
%06/02/16 . Basic structure												 %
%04/08/16 . Layout changes including section, contents, abstract.		 %
%%%%%%%%%%%%%%%%%%%%%%%%%%%%%%%%%%%%%%%%%%%%%%%%%%%%%%%%%%%%%%%%%%%%%%%%%%

%Page Setup
\documentclass[
	12pt, 
	oneside, 
	a4paper,
	reqno,
	final
]{amsart}

\usepackage[
	left = 3cm, 
	right = 3cm, 
	top = 3cm, 
	bottom = 3cm
]{geometry}

%Headers and footers
\usepackage{fancyhdr}
	\pagestyle{fancy}
	%Clear fields
	\fancyhf{}
	%Header right
	\fancyhead[R]{
		\footnotesize
		Yannis B\"{a}hni\\
		\href{mailto:yannis.baehni@uzh.ch}{yannis.baehni@uzh.ch}
	}
	%Header left
	\fancyhead[L]{
		\footnotesize
		401-3581-67L Symplectic Geometry\\
		Autumn 2017
	}
	%Page numbering in footer
	\fancyfoot[C]{\thepage}
	%Separation line header and footer
	\renewcommand{\headrulewidth}{0.4pt}
	%\renewcommand{\footrulewidth}{0.4pt}
	
	\setlength{\headheight}{19pt} 

%Title
\usepackage[foot]{amsaddr}
\usepackage{upref}
\usepackage{newtxtext}
\usepackage[subscriptcorrection,nofontinfo,mtpcal,mtphrb]{mtpro2}
\usepackage{bm}
\usepackage{xspace}
\makeatletter
\usepackage{etoolbox}
\patchcmd{\abstract}{\scshape\abstractname}{\textbf{\abstractname}}{}{}

\usepackage[all,cmtip]{xy}

%Section, subsection and subsubsection font
%------------------------------------------
\makeatletter
	\renewcommand{\@secnumfont}{\bfseries}
	\renewcommand\section{\@startsection{section}{1}%
  	\z@{.7\linespacing\@plus\linespacing}{.5\linespacing}%
  	{\normalfont\bfseries\centering}}
	\renewcommand\subsection{\@startsection{subsection}{2}%
    	\z@{.5\linespacing\@plus.7\linespacing}{-.5em}%
    	{\normalfont\bfseries}}%
	\renewcommand\subsubsection{\@startsection{subsubsection}{3}%
    	\z@{.5\linespacing\@plus.7\linespacing}{-.5em}%
    	{\normalfont\bfseries}}%
%Formatting title of TOC
\renewcommand{\contentsnamefont}{\bfseries}
%Table of Contents
\setcounter{tocdepth}{3}

% Add bold to \section titles in ToC and remove . after numbers
\renewcommand{\tocsection}[3]{%
  \indentlabel{\@ifnotempty{#2}{\bfseries\ignorespaces#1 #2\quad}}\bfseries#3}
% Remove . after numbers in \subsection
\renewcommand{\tocsubsection}[3]{%
  \indentlabel{\@ifnotempty{#2}{\ignorespaces#1 #2\quad}}#3}
\let\tocsubsubsection\tocsubsection% Update for \subsubsection
%...

\newcommand\@dotsep{4.5}
\def\@tocline#1#2#3#4#5#6#7{\relax
  \ifnum #1>\c@tocdepth % then omit
  \else
    \par \addpenalty\@secpenalty\addvspace{#2}%
    \begingroup \hyphenpenalty\@M
    \@ifempty{#4}{%
      \@tempdima\csname r@tocindent\number#1\endcsname\relax
    }{%
      \@tempdima#4\relax
    }%
    \parindent\z@ \leftskip#3\relax \advance\leftskip\@tempdima\relax
    \rightskip\@pnumwidth plus1em \parfillskip-\@pnumwidth
    #5\leavevmode\hskip-\@tempdima{#6}\nobreak
    \leaders\hbox{$\m@th\mkern \@dotsep mu\hbox{.}\mkern \@dotsep mu$}\hfill
    \nobreak
    \hbox to\@pnumwidth{\@tocpagenum{\ifnum#1=1\bfseries\fi#7}}\par% <-- \bfseries for \section page
    \nobreak
    \endgroup
  \fi}
\AtBeginDocument{%
\expandafter\renewcommand\csname r@tocindent0\endcsname{0pt}
}
\def\l@subsection{\@tocline{2}{0pt}{2.5pc}{5pc}{}}
\def\l@subsubsection{\@tocline{2}{0pt}{4.5pc}{5pc}{}}
\makeatother

\advance\footskip0.4cm
\textheight=54pc    %a4paper
\textheight=50.5pc %letterpaper
\advance\textheight-0.4cm
\calclayout

%Font settings
%\usepackage{anyfontsize}
%Footnote settings
%\usepackage{mathptmx}
\usepackage{footmisc}
%	\renewcommand*{\thefootnote}{\fnsymbol{footnote}}
\usepackage{commath}
%Further math environments
%Further math fonts (loads amsfonts implicitely)
%Redefinition of \text
%\usepackage{amstext}
\usepackage{upref}
%Graphics
%\usepackage{graphicx}
%\usepackage{caption}
%\usepackage{subcaption}
%Frames
\usepackage{mdframed}
\allowdisplaybreaks
%\usepackage{interval}
\newcommand{\toup}{%
  \mathrel{\nonscript\mkern-1.2mu\mkern1.2mu{\uparrow}}%
}
\newcommand{\todown}{%
  \mathrel{\nonscript\mkern-1.2mu\mkern1.2mu{\downarrow}}%
}
\AtBeginDocument{\renewcommand*\d{\mathop{}\!\mathrm{d}}}
\renewcommand{\Re}{\operatorname{Re}}
\renewcommand{\Im}{\operatorname{Im}}
\DeclareMathOperator\Log{Log}
\DeclareMathOperator\Arg{Arg}
\DeclareMathOperator\sech{sech}
\DeclareMathOperator*\esssup{ess.sup}
\DeclareMathOperator\id{id}
\DeclareMathOperator\im{im}
\DeclareMathOperator\Vol{Vol}
\DeclareMathOperator\dist{dist}
%\usepackage{hhline}
%\usepackage{booktabs} 
%\usepackage{array}
%\usepackage{xfrac} 
%\everymath{\displaystyle}
%Enumerate
\usepackage{tikz}
\usetikzlibrary{external}
\tikzexternalize % activate!
\usetikzlibrary{patterns}
\pgfdeclarepatternformonly{adjusted lines}{\pgfqpoint{-1pt}{-1pt}}{\pgfqpoint{40pt}{40pt}}{\pgfqpoint{39pt}{39pt}}%
{
  \pgfsetlinewidth{.8pt}
  \pgfpathmoveto{\pgfqpoint{0pt}{0pt}}
  \pgfpathlineto{\pgfqpoint{39.1pt}{39.1pt}}
  \pgfusepath{stroke}
}
\usepackage{enumitem} 
%\renewcommand{\labelitemi}{$\bullet$}
%\renewcommand{\labelitemii}{$\ast$}
%\renewcommand{\labelitemiii}{$\cdot$}
%\renewcommand{\labelitemiv}{$\circ$}
%Colors
%\usepackage{color}
%\usepackage[cmtip, all]{xy}
%Theorems
\newtheoremstyle{main} 		             	 		%Stylename
  	{}	                                     		%Space above
  	{}	                                    		%Space below
  	{\itshape}			                     		%Body font
  	{}        	                             		%Indent
  	{\bfseries}   	                         		%Head font
  	{.}            	                        		%Head punctuation
  	{ }           	                         		%Head space 
  	{\thmname{#1}\thmnumber{ #2}\thmnote{ (#3)}}	%Head specification
\theoremstyle{main}
\newtheorem{definition}{Definition}[section]
\newtheorem{proposition}{Proposition}[section]
\newtheorem{corollary}{Corollary}[section]
\newtheorem{theorem}{Theorem}[section]
\newtheorem{lemma}{Lemma}[section]
%Roman style theorems
\newtheoremstyle{roman}
	{}
	{}
  	{}
  	{}
	{\bfseries}
	{.}
  	{ }
	{\thmname{#1}\thmnumber{ #2}\thmnote{ (#3)}}
\theoremstyle{roman}
\newtheorem{example}{Example}[section]
\newtheorem{solution}{Solution}[section]
\newtheorem{remark}{Remark}[section]
%Exercise style theorems
\newtheoremstyle{exercise}
  	{}
  	{}
  	{\small}
  	{}
  	{\bfseries}
  	{.}
 	{ }
  	{\thmname{#1}\thmnumber{ #2}\thmnote{ (#3)}}
\theoremstyle{exercise}
\newtheorem{exercise}{Exercise}[section]
%Changing default style of proof environment
\renewcommand*{\proofname}{\itshape Proof}
%German non-ASCII-Characters
%Graphics-Tool
%\usepackage{tikz}
%\usepackage{tikzscale}
%\usepackage{bbm}
%\usepackage{bera}
%Listing-Setup
%Bibliographie
\usepackage[backend=bibtex, style=alphabetic]{biblatex}
%\usepackage[babel, german = swiss]{csquotes}
\bibliography{../latex/bibliography}
%PDF-Linking
%\usepackage[hyphens]{url}
\usepackage[bookmarksopen=true,bookmarksnumbered=true]{hyperref}
%\PassOptionsToPackage{hyphens}{url}\usepackage{hyperref}
\hypersetup{
  colorlinks   = true, %Colours links instead of ugly boxes
  urlcolor     = blue, %Colour for external hyperlinks
  linkcolor    = blue, %Colour of internal links
  citecolor    = blue %Colour of citations
}
%Weierstrass-P symbol for power set
\newcommand{\powerset}{\raisebox{.15\baselineskip}{\Large\ensuremath{\wp}}}
\newcommand{\bld}[1]{\boldmath\textit{\textbf{#1}}\unboldmath}
\usepackage{pict2e}
\makeatletter
\DeclareRobustCommand{\intprod}{%
	\mathbin{\mathpalette\int@prod{(0.1,0)(0.9,0)(0.9,0.8)}}%
}
\newcommand{\int@prod}[2]{%
	\begingroup
	\sbox\z@{$\m@th#1+$}%
	\setlength\unitlength{\wd\z@}%
	\begin{picture}(1,1)
	\roundcap
	\polyline#2
	\end{picture}%
	\endgroup
}
\makeatother
\newcommand{\Sbb}{\mathbb{S}}
\newcommand{\dRrm}{\mathrm{dR}}
\newcommand{\Rbb}{\mathbb{R}}
\newcommand{\Lcal}{\mathcal{L}}
\newcommand{\Zbb}{\mathbb{Z}}
\newcommand{\Nbb}{\mathbb{N}}
\newcommand{\Xfrak}{\mathfrak{X}}


\title{Solutions Sheet 9}
\author{Yannis B\"{a}hni}
\address[Yannis B\"{a}hni]{University of Zurich, R\"{a}mistrasse 71, 8006 Zurich}
\email[Yannis B\"{a}hni]{\href{mailto:yannis.baehni@uzh.ch}{\nolinkurl{yannis.baehni@uzh.ch}}}

\begin{document}

\maketitle
\thispagestyle{fancy}

\setcounter{section}{1}

\begin{enumerate}[label = \textbf{Exercise \arabic*.},wide = 0pt, itemsep = 1.5ex]
	\item We may assume that $f \neq 0$ since otherwise we would have convergence in norm. Thus the continuity of $f$ implies $\norm{f}_p \neq 0$ for all $1 \leq p < \infty$ since also $\abs{f}^p$ is continuous.
		\begin{lemma}
			Let $1 \leq p < \infty$. Then $g_n,h_n,k_n \in L^p(\mathbb{R})$ for all $n \in \mathbb{N}$.	
		\end{lemma}

		\begin{proof}
			We have that 
			\begin{align*}
				&\norm{g_n}_p^p = \int_\mathbb{R} \abs[0]{f(x - n)}^p dx = \int_\mathbb{R} \abs[0]{f(y)} dy = \norm{f}_p^p,\\
				&\norm{h_n}_p^p = n^{-1}\int_\mathbb{R} \abs[0]{f(x/n)}^p dx = \int_\mathbb{R} \abs[0]{f(y)}^p dy = \norm{f}_p^p,\\
				&\norm{k_n}_p^p = \int_\mathbb{R} \abs[0]{f(x) e^{inx}}^p dx = \int_\mathbb{R} \abs{f(x)}^p dx = \norm{f}_p^p 
			\end{align*}
			\noindent and since $f \in C_c^\infty(\mathbb{R})$ implies that $f \in L^p(\mathbb{R})$, the claim follows.
		\end{proof}

		\begin{lemma}
			Let $1 < p < \infty$. Then $g_n,h_n,k_n \rightharpoonup 0$ in $L^p(\mathbb{R})$.
		\end{lemma}

		\begin{proof}
			We make use of lemma 6.2.1 and theorem 2.2.6, which provides an antilinear isometric isomorphism $\del[1]{L^p(\mathbb{R})}^* \cong L^q(\mathbb{R})$ where $q$ is the dual exponent of $p$. Since $f \in C_c^\infty(\mathbb{R})$, there exists some $M > 0$ such that $\supp(f) \subseteq \intcc{-M,M}$. It is easy to verify that $\supp(g_n) \subseteq \intcc{-M + n, M + n}$ for all $n \in \mathbb{N}$. Let $\varphi \in L^q(\mathbb{R})$. Then H\"older's inequality implies  
			\begin{align*}
				\abs[3]{\int_\mathbb{R} \wbar{\varphi}(x)g_n(x)dx} &\leq \int_\mathbb{R} \abs{\varphi(x)g_n(x)}dx\\
				&= \int_\mathbb{R} \abs[0]{\varphi(x)g_n(x)\chi_{\supp(g_n)}(x)}dx\\
				&\leq \int_\mathbb{R}\abs[0]{\varphi(x)g_n(x)\chi_{\intcc{-M + n,M + n}}(x)}dx\\
				&= \norm[0]{\varphi \chi_{\intcc{-M + n,M + n}}g_n}_1\\
				&\leq \norm[0]{\varphi\chi_{\intcc{-M+n,M+n}}}_q\norm{g_n}_p\\
				&=\norm[0]{\varphi\chi_{\intcc{-M+n,M+n}}}_q\norm{f}_p \to 0
			\end{align*}
			\noindent since dominated convergence yields
			\begin{equation*}
				\lim_{n \to \infty} \int_\mathbb{R} \abs{\varphi(x)}^q \chi_{\intcc{-M + n,M + n}}(x) dx = \int_\mathbb{R} \abs{\varphi(x)}^q \lim_{n \to \infty}\chi_{\intcc{-M + n,M + n}}(x) dx = 0
			\end{equation*}
			\noindent which application is justified by the fact that $ \abs{\varphi(x)}^q \chi_{\intcc{-M + n,M + n}}(x) \leq \abs{\varphi(x)}^q \in L^1(\mathbb{R})$ and $\chi_{\intcc{-M + n,M + n}}(x) \to 0$ for all $x \in \mathbb{R}$ (take $n$ just sufficiently large).\\
			Observe that
			\begin{align*}
				\int_\mathbb{R} \wbar{\varphi}(x)k_n(x)dx &= \sqrt{\frac{2\pi}{2\pi}} \int_\mathbb{R} \wbar{\varphi}(x)f(x)e^{inx}dx\\
				&= \sqrt{\frac{2\pi}{2\pi}} \int_\mathbb{R} \wbar{\varphi}(x)f(x)e^{inx}dx\\
				&= \sqrt{\frac{2\pi}{2\pi}} \int_\mathbb{R} \wbar{\varphi}(x)f(x)e^{-i(-n)x}dx\\
				&= \sqrt{2\pi} \widehat{\wbar{\varphi}f}(-n) \to 0
			\end{align*}
			\noindent by the Riemann-Lebesgue lemma since by H\"olders inequality, $\wbar{\varphi}f \in L^1(\mathbb{R})$.
		\end{proof}

		\begin{lemma}
			Let $X$ be a normed space and $(x_n)_{n \in \mathbb{N}}$ a sequence in $X$ such that $x_n \rightharpoonup x$. If $x_n \to y$ for some $y \in X$, then $x = y$.
		\end{lemma}

		\begin{proof}
			Suppose that $x_n \to y$. Then since $\mathcal{T}_W \subseteq \mathcal{T}_{\norm[0]{\cdot}}$, we have that $x_n \rightharpoonup y$. But $(X,\mathcal{T}_W)$ is Hausdorff and thus limits are unique. Hence $x = y$.
		\end{proof}

		\begin{corollary}
			Let $1 < p < \infty$. Then $g_n$, $h_n$ and $k_n$ do not converge in norm.
		\end{corollary}

		\begin{proof}
			Since all three sequences converge weakly to $0$, we only have to show that they do not converge towards $0$ in $L^p(\mathbb{R})$. However, this is immediate from the first lemma, since all sequences have constant norm $\norm{f}_p \neq 0$ and hence the limit should have also nonzero norm.
		\end{proof}

		Let us now investigate the case $p = 1$.

	\item
		~
		\begin{enumerate}[label = \textbf{\alph*.},wide = 0pt, itemsep = 1.5ex]
			\item 
			\item Suppose that $x_n \rightharpoonup x$ and $\norm{x_n} \to \norm{x}$. By lemma 6.2.1. we have that $f(x_n) \to f(x)$ for all $f \in H^*$. Using the \emph{Riesz representation theorem} this is equivalent to $\inprod{y}{x_n} \to \inprod{y}{x}$ for all $y \in H$. But then
				\begin{equation*}
					\norm{x - x_n}^2 = \inprod{x - x_n}{x - x_n} = \norm{x}^2 - 2\Re\inprod{x}{x_n} + \norm{x_n}^2 \to 0
				\end{equation*}
				\noindent since $\Re$ is a continuous function and $\inprod{x}{x_n} \to \norm{x}^2$.
		\end{enumerate}

	\item 
		~
		\begin{enumerate}[label = \textbf{\alph*.},wide = 0pt, itemsep = 1.5ex]
			\item 
				\begin{lemma}
					Let $0 < \varepsilon < 1$ and define 
					\begin{equation*}
						I_\varepsilon(f) := \varepsilon^{-1}\int_0^\varepsilon f(x) dx
					\end{equation*}
					\noindent for $f \in L^\infty(0,1)$. Then $I_\varepsilon \in \del[1]{L^\infty(0,1)}^*$ and $\norm{I_\varepsilon} = 1$ for all $0 < \varepsilon < 1$.
				\end{lemma}

				\begin{proof}
					 Let $f \in L^\infty(0,1)$. Then we have that $\abs{f} \leq \norm{f}_\infty$ $\lambda$-a.e. Hence 
					 \begin{equation}
						 \abs[0]{I_\varepsilon(f)} = \varepsilon^{-1} \abs[3]{\int_0^\varepsilon f(x) dx} \leq \varepsilon^{-1} \int_0^\varepsilon \abs{f(x)}dx \leq \norm{f}_\infty
						 \label{eq:3.a.}
					 \end{equation}
					 \noindent and thus $I_\varepsilon$ is bounded and thus continuous. Clearly $I_\varepsilon$ is $\mathbb{C}$-linear by the $\mathbb{C}$-linearity of the integral. Moreover, using (\ref{eq:3.a.}) we get that
					 \begin{equation*}
						 \norm{I_\varepsilon} = \sup_{\norm[0]{f}_\infty = 1}\abs{I_\varepsilon(f)} \leq \sup_{\norm[0]{f} = 1}\norm{f}_\infty = 1.
					 \end{equation*}
					 Conversly, setting $f := \chi_{\intoo{0,1}} \in L^\infty(0,1)$, we get that $\abs{I_\varepsilon(f)} = 1$ and hence by $\norm{f} = 1$
					 \begin{equation*}
						 \norm{I_\varepsilon} = \sup_{\norm[0]{g}_\infty = 1} \abs{I_\varepsilon(g)} \geq \abs{I_\varepsilon(f)} = 1. 
					 \end{equation*}
				\end{proof}
		\end{enumerate}

	\item 
		~
		\begin{enumerate}[label = \textbf{\alph*.},wide = 0pt, itemsep = 1.5ex]
			\item First we show that $\norm{\cdot}_\sigma$ is well defined. Let $x^* \in X^*$, then we have 
				\begin{equation*}
					\norm[0]{x^*}_\sigma = \sum_{k = 1}^\infty 2^{-k}\abs[0]{x^*(x_k)} \leq \sum_{k = 1}^\infty 2^{-k} \norm[0]{x^*} \norm{x_k} = \norm[0]{x^*}\sum_{k = 1}^\infty 2^{-k} = \norm[0]{x^*} < \infty
				\end{equation*}
				\noindent since $x_k \in S_X$ and $\sum_{k = 1}^\infty 2^{-k} = 1$. Hence $\norm[0]{x^*}_\sigma \leq \norm[0]{x^*}$ holds. Let $\lambda \in \mathbb{K}$. Then we have that
				\begin{equation*}
					\norm[0]{\lambda x^*}_\sigma = \sum_{k = 1}^\infty 2^{-k} \abs[0]{\lambda x^*(x_k)} = \sum_{k = 1}^\infty 2^{-k} \abs{\lambda} \abs[0]{x^*(x_k)} = \abs{\lambda} \sum_{k = 1}^\infty 2^{-k} \abs[0]{x^*(x_k)} = \abs{\lambda}\norm[0]{x^*}_\sigma.
				\end{equation*}
				Let $y^* \in X^*$. Then the triangle inequality follows from
				\begin{align*}
					\norm[0]{x^* + y^*}_\sigma &= \sum_{k = 1}^\infty 2^{-k}\abs[0]{x^*(x_k) + y^*(x_k)}\\
					&\leq \sum_{k = 1}^\infty 2^{-k}\abs[0]{x^*(x_k)} + \sum_{k = 1}^\infty 2^{-k}\abs[0]{y^*(x_k)}\\
					&= \norm[0]{x^*}_\sigma + \norm[0]{y^*}_\sigma.
				\end{align*}
				Lastly, clearly $\norm[0]{x^*} = 0$ if $x^* = 0$. Conversly, suppose that $\norm[0]{x^*} = 0$. Hence $x^*(x_k) = 0$ for all $k \in \mathbb{N}$. Let $Y \in X$. Since $\overline{\mathrm{span}\cbr[0]{x_k : k \in \mathbb{N}}} = X$, we find a sequence $(y_n)_{n \in \mathbb{N}}$ in $\mathrm{span}\cbr{x_k : k \in \mathbb{N}}$, such that $y_n \to y$. Moreover, for each $n \in \mathbb{N}$ we have that $y_n = \sum_{k = 1}^\infty \lambda^{(n)}_k x_k$ for $\lambda^{(n)}_k \in \mathbb{K}$ and $\lambda^{(n)}_k = 0$ for all but finitely many $k \in \mathbb{N}$. Hence
				\begin{equation*}
					x^*(y) = \lim_{n \to \infty}x^*(y_n) = \lim_{n \to \infty}\sum_{k = 1}^\infty \lambda^{(n)}_kx^*(x_k) = 0
				\end{equation*}
				\noindent by the continuity of $x^*$ and so $x^* = 0$.
		\end{enumerate}

\end{enumerate}
\printbibliography
\end{document}
