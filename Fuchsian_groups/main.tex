%%%%%%%%%%%%%%%%%%%%%%%%%%%%%%%%%%%%%%%%%%%%%%%%%%%%%%%%%%%%%%%%%%%%%%%%%%
%Author:																 %
%-------																 %
%Yannis Baehni at University of Zurich									 %
%baehni.yannis@uzh.ch													 %
%																		 %
%Version log:															 %
%------------															 %
%06/02/16 . Basic structure												 %
%04/08/16 . Layout changes including section, contents, abstract.		 %
%%%%%%%%%%%%%%%%%%%%%%%%%%%%%%%%%%%%%%%%%%%%%%%%%%%%%%%%%%%%%%%%%%%%%%%%%%

%Page Setup
\documentclass[
	12pt, 
	oneside, 
	a4paper,
	reqno,
	final
]{amsbook}

\usepackage[
	left = 3cm, 
	right = 3cm, 
	top = 3cm, 
	bottom = 3cm
]{geometry}

%Headers and footers
\usepackage{fancyhdr}
	\pagestyle{fancy}
	%Clear fields
	\fancyhf{}
	%Header right
	\fancyhead[R]{
		\footnotesize
		Yannis B\"{a}hni\\
		\href{mailto:yannis.baehni@uzh.ch}{yannis.baehni@uzh.ch}
	}
	%Header left
	\fancyhead[L]{
		\footnotesize
		401-3001-61L Algebraic Topology I\\
		Autumn Semester 2017
	}
	%Page numbering in footer
	\fancyfoot[C]{\thepage}
	%Separation line header and footer
	\renewcommand{\headrulewidth}{0.4pt}
	%\renewcommand{\footrulewidth}{0.4pt}
	
	\setlength{\headheight}{19pt} 

%Title
\usepackage[foot]{amsaddr}
\usepackage{newtxtext}
\usepackage[subscriptcorrection,nofontinfo,mtpcal,mtphrb]{mtpro2}
\usepackage{mathtools}
\usepackage{bm}
\usepackage{xspace}
\usepackage[all]{xy}
\usepackage{tikz-cd}
\makeatletter
\def\@textbottom{\vskip \z@ \@plus 1pt}
\let\@texttop\relax
\usepackage{etoolbox}
\patchcmd{\abstract}{\scshape\abstractname}{\textbf{\abstractname}}{}{}
\usepackage{chngcntr}
\counterwithout{figure}{chapter}
%Section, subsection and subsubsection font
%------------------------------------------
	\renewcommand{\@secnumfont}{\bfseries}
	\renewcommand\section{\@startsection{section}{1}%
  	\z@{.7\linespacing\@plus\linespacing}{.5\linespacing}%
  	{\normalfont\bfseries\boldmath\centering}}
	\renewcommand\subsection{\@startsection{subsection}{2}%
    	\z@{.5\linespacing\@plus.7\linespacing}{-.5em}%
    	{\normalfont\bfseries\boldmath}}%
	\renewcommand\subsubsection{\@startsection{subsubsection}{3}%
    	\z@{.5\linespacing\@plus.7\linespacing}{-.5em}%
    	{\normalfont\bfseries\boldmath}}%

		\renewenvironment{proof}{\textit{Proof}.}{\hfill\qedsymbol}

%ToC
%---
\makeatletter
\setcounter{tocdepth}{3}
% Add bold to \chapter titles in ToC and remove . after numbers
\renewcommand{\tocchapter}[3]{%
  	\indentlabel{\@ifnotempty{#2}{\bfseries\ignorespaces#1 #2: }}\bfseries#3}
\renewcommand{\tocappendix}[3]{%
  	\indentlabel{\@ifnotempty{#2}{\bfseries\ignorespaces#1 #2: }}\bfseries#3}
% Remove . after numbers in \section and \subsection
\renewcommand{\tocsection}[3]{%
  	\indentlabel{\@ifnotempty{#2}{\ignorespaces#1 #2\quad}}#3}
\renewcommand{\tocsubsection}[3]{%
  	\indentlabel{\@ifnotempty{#2}{\ignorespaces#1 #2\quad}}#3}
\let\tocsubsubsection\tocsubsection% Update for \subsubsection
%...
\newcommand\@dotsep{4.5}
\def\@tocline#1#2#3#4#5#6#7{\relax
  \ifnum #1>\c@tocdepth % then omit
  \else
    \par \addpenalty\@secpenalty\addvspace{#2}%
    \begingroup \hyphenpenalty\@M
    \@ifempty{#4}{%
      \@tempdima\csname r@tocindent\number#1\endcsname\relax
    }{%
      \@tempdima#4\relax
    }%
    \parindent\z@ \leftskip#3\relax \advance\leftskip\@tempdima\relax
    \rightskip\@pnumwidth plus1em \parfillskip-\@pnumwidth
    #5\leavevmode\hskip-\@tempdima{#6}\nobreak
    \leaders\hbox{$\m@th\mkern \@dotsep mu\hbox{.}\mkern \@dotsep mu$}\hfill
    \nobreak
    \hbox to\@pnumwidth{\@tocpagenum{\ifnum#1=0\bfseries\fi#7}}\par% <-- \bfseries for \chapter page
    \nobreak
    \endgroup
  \fi}
\AtBeginDocument{%
\expandafter\renewcommand\csname r@tocindent0\endcsname{0pt}
}
\def\l@subsection{\@tocline{2}{0pt}{2.5pc}{5pc}{}}
\def\l@subsubsection{\@tocline{2}{0pt}{4.5pc}{5pc}{}}
\makeatother

\advance\footskip0.4cm
\textheight=54pc    %a4paper
\textheight=50.5pc %letterpaper
\advance\textheight-0.4cm
\calclayout

%Font settings
%\usepackage{anyfontsize}
%Footnote settings
\usepackage{footmisc}
%	\renewcommand*{\thefootnote}{\fnsymbol{footnote}}
\usepackage{commath}
%Further math environments
%Further math fonts (loads amsfonts implicitely)
%Redefinition of \text
%\usepackage{amstext}
\usepackage{upref}
%Graphics
%\usepackage{graphicx}
%\usepackage{caption}
%\usepackage{subcaption}
%Frames
\usepackage{mdframed}
\allowdisplaybreaks
%\usepackage{interval}
\newcommand{\toup}{%
  \mathrel{\nonscript\mkern-1.2mu\mkern1.2mu{\uparrow}}%
}
\newcommand{\todown}{%
  \mathrel{\nonscript\mkern-1.2mu\mkern1.2mu{\downarrow}}%
}
\AtBeginDocument{\renewcommand*\d{\mathop{}\!\mathrm{d}}}
\renewcommand{\Re}{\operatorname{Re}}
\renewcommand{\Im}{\operatorname{Im}}
\DeclareMathOperator\Log{Log}
\DeclareMathOperator\Arg{Arg}
\DeclareMathOperator\id{id}
\DeclareMathOperator\sech{sech}
\DeclareMathOperator\Aut{Aut}
\DeclareMathOperator\h{h}
\DeclareMathOperator\sgn{sgn}
\DeclareMathOperator\arctanh{arctanh}
\DeclareMathOperator\supp{supp}
\DeclareMathOperator\ob{ob}
\DeclareMathOperator\mor{mor}
\DeclareMathOperator\M{M}
\DeclareMathOperator\dom{dom}
\DeclareMathOperator\cod{cod}
\DeclareMathOperator\im{im}
\DeclareMathOperator\Ab{Ab}
\DeclareMathOperator\coker{coker}
%\usepackage{hhline}
%\usepackage{booktabs} 
%\usepackage{array}
%\usepackage{xfrac} 
%\everymath{\displaystyle}
%Enumerate
\usepackage{tikz}
%\usepackgae{graphicx}
\usepackage{subcaption}
\usepackage{enumitem} 
%\renewcommand{\labelitemi}{$\bullet$}
%\renewcommand{\labelitemii}{$\ast$}
%\renewcommand{\labelitemiii}{$\cdot$}
%\renewcommand{\labelitemiv}{$\circ$}
%Colors
%\usepackage{color}
%\usepackage[cmtip, all]{xy}
%Main style theorem environment
\newtheoremstyle{main} 		             	 		%Stylename
  	{}	                                     		%Space above
  	{}	                                    		%Space below
  	{\itshape}			                     		%Body font
  	{}        	                             		%Indent
  	{\bfseries\boldmath}   	                         		%Head font
  	{.}            	                        		%Head punctuation
  	{ }           	                         		%Head space 
  	{\thmname{#1}\thmnumber{ #2}\thmnote{ (#3)}}	%Head specification
\theoremstyle{main}
\newtheorem{definition}{Definition}[chapter]
\newtheorem{proposition}{Proposition}[chapter]
\newtheorem{corollary}{Corollary}[chapter]
\newtheorem{theorem}{Theorem}[chapter]
\newtheorem{lemma}{Lemma}[chapter]
\newtheoremstyle{nonit} 		             	 		%Stylename
  	{}	                                     		%Space above
  	{}	                                    		%Space below
  	{}			                     		%Body font
  	{}        	                             		%Indent
  	{\bfseries\boldmath}   	                   		%Head font
  	{.}            	                        		%Head punctuation
  	{ }           	                         		%Head space 
  	{\thmname{#1}\thmnumber{ #2}\thmnote{ (#3)}}	%Head specification
\theoremstyle{nonit}
\newtheorem{remark}{Remark}[chapter]
\newtheorem{examples}{Examples}[chapter]
\newtheorem{example}{Example}[chapter]
\newtheorem{problem}{Problem}[chapter]
\newtheoremstyle{ex} 		             	 		%Stylename
  	{}	                                     		%Space above
  	{}	                                    		%Space below
  	{\small}			                     		%Body font
  	{}        	                             		%Indent
  	{\bfseries\boldmath}   	                         		%Head font
  	{.}            	                        		%Head punctuation
  	{ }           	                         		%Head space 
  	{\thmname{#1}\thmnumber{ #2}\thmnote{ (#3)}}	%Head specification
\theoremstyle{ex}
\newtheorem{exercise}{Exercise}[chapter]
%German non-ASCII-Characters
%Graphics-Tool
%\usepackage{tikz}
%\usepackage{tikzscale}
%\usepackage{bbm}
%\usepackage{bera}
%Listing-Setup
%Bibliographie
\usepackage[backend=bibtex, style=alphabetic]{biblatex}
%\usepackage[babel, german = swiss]{csquotes}
\bibliography{bibliography}
%PDF-Linking
%\usepackage[hyphens]{url}
\usepackage[bookmarksopen=true,bookmarksnumbered=true]{hyperref}
%\PassOptionsToPackage{hyphens}{url}\usepackage{hyperref}
\urlstyle{rm}
\hypersetup{
  colorlinks   = true, %Colours links instead of ugly boxes
  urlcolor     = blue, %Colour for external hyperlinks
  linkcolor    = blue, %Colour of internal links
  citecolor    = blue %Colour of citations
}
\newcommand{\bld}[1]{\boldmath\textit{\textbf{#1}}\unboldmath}
\newcommand{\eqclass}[1]{\sbr[0]{#1}}
\newcommand{\cat}[1]{\mathsf{#1}}
\newcommand{\Sbb}{\mathbb{S}}
\newcommand{\Zbb}{\mathbb{Z}}
\newcommand{\Nbb}{\mathbb{N}}
\newcommand{\Rbb}{\mathbb{R}}
\newcommand{\Hbb}{\mathbb{H}}
\newcommand{\Cbb}{\mathbb{C}}
\newcommand{\Tcal}{\mathcal{T}}
\newcommand{\SLrm}{\mathrm{SL}}
\newcommand{\PSLrm}{\mathrm{PSL}}
\newcommand{\SLrmstar}{\mathrm{S^*L}}
\newcommand{\PSLrmstar}{\mathrm{PS^*L}}
\newcommand{\GLrm}{\mathrm{GL}}
\newcommand{\Mrm}{\mathrm{M}}
\newcommand{\Isom}{\mathrm{Isom}}
\newcommand{\Mob}{\mathrm{M\ddot{o}b}}
\newcommand{\Ebb}{\mathbb{E}}
\newcommand{\Cscr}{\mathscr{C}}
\newcommand{\pwrm}{\mathrm{pw}}
\newcommand{\clos}[1]{\overline{#1}}
\newcommand{\Hcal}{\mathcal{H}}
\newcommand{\Hbcal}{\bm{\mathcal{H}}}
\newcommand{\Ucal}{\mathcal{U}}
\newcommand{\Ubcal}{\bm{\mathcal{U}}}
\renewcommand{\det}{\mathrm{det}}
\newcommand{\ab}{\mathrm{ab}}


\title{Seminar: Introduction to Fuchsian Groups}
\author{Yannis B\"{a}hni}
\address[Yannis B\"{a}hni]{University of Zurich, R\"{a}mistrasse 71, 8006 Zurich}
\email[Yannis B\"{a}hni]{\href{mailto:yannis.baehni@uzh.ch}{\nolinkurl{yannis.baehni@uzh.ch}}}

\begin{document}

\maketitle

\begin{abstract}
	Aim of this talk is to present the first ten pages in the book \emph{Fuchsian groups} by \emph{Svetlana Katok}. It is a short introduction into the rudiments of \emph{hyperbolic geometry} and introduces an action of the \emph{projective special linear group} on the \emph{upper half-plane}. The main theorem proven in this paper is the \emph{classification theorem for the isometries of the upper half-plane}.
\end{abstract}

\tableofcontents

\section{Hyperbolic Geometry}
\subsection{The Hyperbolic Metric}
The main object of our study is the so-called \emph{hyperbolic plane}. There are three equivalent models of this space, each of which is useful in certain contexts. We are only concerned with two of them and introduce now the first. 

\begin{definition}[Poincar\'e Half-Space Model]
	The upper half-plane
	\begin{equation}
		\Hbb := \cbr[0]{z \in \Cbb : \Im(z) > 0}
	\end{equation}
	\noindent equipped with the metric
	\begin{equation}
		g := \frac{dx^2 + dy^2}{y^2}
	\end{equation}
	\noindent is a model for the hyperbolic plane, the \bld{Poincar\'e half-space model}.
	\label{def:half_space_model}
\end{definition}

\begin{remark}
	The tuple $(\Hbb,g)$ defined in \ref{def:half_space_model} is generally called a \emph{Riemannian manifold} (if we equip $\Hbb$ with the smooth structure generated by the atlas $\cbr[0]{(\Hbb,\id_\Hbb)}$). We will not use this level of abstraction. However it is important to know, that the metric $g$ induces an inner product $g_p: T_p\Hbb \times T_p\Hbb \to \Rbb$ on the tangent space $T_p\Hbb$ at each $p \in \Hbb$. However, since we do only have one chart for $\Hbb$, we have that $T_p\Hbb \cong \Cbb$ (\cite[56]{lee:smooth_manifolds:2013} together with \cite[53]{lee:smooth_manifolds:2013}). Explicitely, if $p := (x_0,y_0) \in \Hbb$ and $\xi_1 + i\eta_1,\xi_2 + i\eta_2 \in \Cbb$ we have that
	\begin{equation}
		g_p(\xi_1 + i\eta_1, \xi_2 + i\eta_2) = \frac{1}{y_0^2}(\xi_1\xi_2 + \eta_1\eta_2).
	\end{equation}
	Because of this, we can define the notion of \emph{lengths} and \emph{angles} on more abstract objects. But it is important to observe, that those notions strongly depend on the choice of metric and maybe contradict our intuition. 
\end{remark}

\begin{definition}[Hyperbolic Length]
	Let $I := \intcc{0,1}$ and $\gamma := x(t) + iy(t) \in \Cscr^1_{\pwrm}(I,\Hbb)$. Then the \bld{hyperbolic length} of $\gamma$, written $\h(\gamma)$, is defined to be
	\begin{equation}
		\h(\gamma) := \int_0^1 \frac{\sqrt{\del[1]{\frac{dx}{dt}}^2 + \del[1]{\frac{dy}{dt}}^2}}{y(t)} dt = \int_0^1 \frac{\abs[1]{\frac{d\gamma}{dt}}}{y(t)}dt.
	\end{equation}
\end{definition}

\begin{definition}[Hyperbolic Distance]
	Let $z,w \in \Hbb$. The \bld{hyperbolic distance} between $z$ and $w$, written $\rho(z,w)$, is defined to be
	\begin{equation}
		\rho(z,w) := \inf_{\substack{\gamma \in \Cscr^1_\pwrm(I,\Hbb)\\ \gamma(0) = z\\ \gamma(1) = w}} \h(\gamma)
	\end{equation}
\end{definition}

Recall that $\Rbb^\times := \Rbb \setminus \cbr[0]{0}$.

\begin{definition}
	Let $n \in \Zbb$, $n \geq 1$ and $\det : \Mrm(n,\Rbb) \to \Rbb$ denote the determinant function. Then we define:
	\begin{center}
		\begin{tabular}{ll}
			$\GLrm(n,\Rbb) := \det^{-1}(\Rbb^\times)$ & (\textbf{General Linear Group}),\\
			$\GLrm^+(n,\Rbb) := \det^{-1}(\intoo{0,\infty})$ & (\textbf{Positive General Linear Group}),\\
			$\SLrm(n,\Rbb) := \det^{-1}(1)$ & (\textbf{Special Linear Group}),\\
			$\PSLrm(n,\Rbb) := \SLrm(n,\Rbb)/ Z(\SLrm(n,\Rbb))$ & (\textbf{Projective Special Linear Group}).\\
		\end{tabular}
	\end{center}
\end{definition}

\begin{remark}
	We have that $Z(\SLrm(n,\Rbb)) = \cbr[0]{\pm I}$. Thus $\PSLrm(n,\Rbb) = \SLrm(n,\Rbb)/\cbr[0]{\pm I}$.
\end{remark}

Recall that if $G$ is a group and $X$ a set, a \emph{left action of $G$ on $X$} is a mapping $G \times X \to X$, written $(g,x) \mapsto g \cdot x$, with the following properties:
\begin{enumerate}[label = \textup{(}\roman*\textup{)}]
	\item $g_1 \cdot (g_2 \cdot x) = (g_1g_2) \cdot x$ for all $x \in X$ and $g_1,g_2 \in G$.
	\item $1 \cdot x = x$ for all $x \in X$.
\end{enumerate}
If $X$ is a topological space and $G$ acts on $X$, we say that the action is an \emph{action by homeomorphisms} if for each $g \in G$ the mapping $x \mapsto g \cdot x$ is a homeomorphism (see \cite[78--79]{lee:topological_manifolds:2011}). Moreover, this induces a homomorphism of $G$ into $S_X$ (see \cite[54]{grillet:abstract_algebra:2007}).

\begin{proposition}
	Let $z \in \Hbb$. Define 
	\begin{equation}
		A \cdot z := \frac{az + b}{cz + d}, \qquad \begin{pmatrix}
			a & b\\
			c & d
		\end{pmatrix} \in \GLrm^+(2,\Rbb).
	\end{equation}
	This defines an action by homeomorphisms of $\GLrm^+(2,\Rbb)$ on $\Hbb$.
	\label{prop:action}
\end{proposition}

\begin{proof}
	First we show that $A \cdot z \in \Hbb$ for any $A \in \GLrm^+(2,\Rbb)$ and $z \in \Hbb$. This immediately follows by
	\begin{equation}
\Im\del[1]{A \cdot z} = \frac{1}{2i}\del[1]{A \cdot z - \overline{A \cdot z}} = \frac{1}{2i}\frac{(ad - bc)(z - \overline{z})}{\abs[0]{cz + d}^2} = \frac{\det(A)\Im(z)}{\abs[0]{cz + d}^2} > 0
		\label{eq:imaginary_part}
	\end{equation}
	\noindent since $\det(A),\Im(z) > 0$ and $z \neq -d/c$. Furthermore it is easy to check that this defines indeed an action. It is an action by homeomorphisms since $z \mapsto A \cdot z$ is a bijection and clearly continuous as a well-defined rational function. 
\end{proof}

\begin{corollary}
	The action of $\GLrm^+(2,\Rbb)$ on $\Hbb$ defined in proposition \ref{prop:action} descends to an action by homeomorphisms of $\PSLrm(2,\Rbb)$ on $\Hbb$.
\end{corollary}

\begin{proof}
	Since $\SLrm(2,\Rbb) \leq \GLrm^+(2,\Rbb)$ the action restricts to an action of $\SLrm(2,\Rbb)$ on $\Hbb$. Since $\cbr[0]{\pm I} \subseteq \ker(\SLrm(2,\Rbb) \to S_\Hbb)$, the mapping $\SLrm(2,\Rbb) \to S_\Hbb$ factors uniquely through the canonical projection $\pi : \SLrm(2,\Rbb) \to \SLrm(2,\Rbb)/\cbr[0]{\pm I} = \PSLrm(2,\Rbb)$ by \cite[23]{grillet:abstract_algebra:2007}.	
\end{proof}

\begin{definition}[M\"obius Transformations]
	Define the group of \bld{M\"obius transformations} on $\Hbb$ by
	\begin{equation}
		\Mob(\Hbb) := \cbr[0]{z \mapsto A \cdot z : A \in \PSLrm(2,\Rbb)}.
	\end{equation}
\end{definition}

\begin{remark}
	By construction it is clear that $\Mob(\Hbb) \cong \PSLrm(2,\Rbb)$.
\end{remark}

\begin{proposition}
	$\cbr[0]{z \mapsto A \cdot z : A \in \GLrm^+(2,\Rbb)} \subseteq \Mob(\Hbb)$.
\end{proposition}

\begin{proof}
	Let $A \in \GLrm^+(2,\Rbb)$. Then $\wtilde{A} := \frac{1}{\sqrt{\det A}}A \in \SLrm(2,\Rbb)$ and it is easy to check that $z \mapsto A \cdot z$ is the same mapping as $z \mapsto \wtilde{A} \cdot z$.
\end{proof}

\begin{definition}[Isometries]
	A transformation of $\Hbb$ onto itself is called an \bld{isometry} if it preserves the hyperbolic distance on $\Hbb$. We shall denote the set of all isometries on $\Hbb$ by $\Isom(\Hbb)$.
\end{definition}

\begin{proposition}
	$\Mob(\Hbb) \subseteq \Isom(\Hbb)$.
	\label{prop:inclusion}
\end{proposition}

\begin{proof}
	Let $f \in \Mob(\Hbb)$ and $\gamma := x(t) + iy(t) \in \Cscr^1_\pwrm(I,\Hbb)$. Then there exists a partition $a_0 < \dots < a_n$ of $I$ such that $\gamma$ restricted to each subinterval is continuously differentiable. By (\ref{eq:imaginary_part}) we have that
	\begin{equation*}
		\Im(f \circ \gamma) = \frac{y}{\abs[0]{c\gamma + d}^2}
	\end{equation*}
	\noindent and \cite[25]{lieb:funktionentheorie:2003} yields
	\begin{equation*}
		\frac{d(f \circ \gamma)}{dt} = \frac{\partial f}{\partial z}\frac{d\gamma}{dt} + \frac{\partial f}{\partial \wbar{z}}\frac{d\wbar{\gamma}}{dt} = \frac{\partial f}{\partial z}\frac{d\gamma}{dt} = \frac{1}{(c\gamma + d)^2}\frac{d\gamma}{dt}
	\end{equation*}
	\noindent for all $t \neq a_\nu$, $\nu = 0,\dots,n$, by the holomorphicity of $f$. Thus
	\begin{equation*}
		\h(f \circ \gamma) =  \sum_{\nu = 1}^n \int_{a_{\nu - 1}}^{a_{\nu}} \frac{\abs[1]{\frac{d(f \circ \gamma)}{dt}}}{\Im(f(\gamma(t)))}dt = \sum_{\nu = 1}^n\int_{a_{\nu - 1}}^{a_\nu} \frac{\abs[1]{\frac{d\gamma}{dt}}}{y(t)}dt = \h(\gamma).
	\end{equation*}
\end{proof}

\subsection{Geodesics}
The fifth postulate of Euclid's geometry says that given any line in the plane and a point which does not belong to it, then there is a unique line through this point never intersecting the other line. This no longer holds in the hyperbolic plane. Thus we speak of non-Euclidean geometry. In this section we find out, what curves in $\Hbb$ correspond to straight lines in the Eudlidean plane.

\begin{definition}[Geodesics]
	The shortest curves $\gamma \in \Cscr^\infty(J,\Hbb)$, where $J \subseteq \Rbb$ is an open interval, with respect to the hyperbolic metric $g$, i.e. such that $\h$ is minimal, are called \bld{geodesics}.
\end{definition}

The next proposition is exercise 1.1. \cite[21]{katok:Fuchsian_groups:1992}.

\begin{proposition}
	Let $A \subseteq \Hbb$ be a semicircle with center on the real axis or a line parallel to the imaginary axis. Then there exists $f \in \Mob(\Hbb)$ such that $f(A) \subseteq i\Rbb_{>0}$.
	\label{prop:mapping_to_imaginary_axis}
\end{proposition}

\begin{proof}
	First consider the case where $A$ is a line parallel to the imaginary axis. $A$ may be written as the image of the curve $\gamma : \intoo{0,\infty} \to \Hbb$ defined by $\gamma(t) := \alpha + it$, where $\alpha \in \Rbb$ is the thought point of intersection of $A$ with the real axis. For $\beta \in \Rbb$ define $f_\beta : \Hbb \to \Cbb$ by
	\begin{equation}
		f_\beta(z) := -\frac{1}{z - \alpha} + \beta. 	
	\end{equation}
	It is immediate that $f_\beta \in \Mob(\Hbb)$ since $f_\beta(z) = A \cdot z$ for 
	\begin{equation*}
		A := \begin{pmatrix}
			\beta & - (\alpha\beta + 1)\\
			1 & -\alpha
		\end{pmatrix}.
	\end{equation*}
	For $t \in \intoo{0,\infty}$ we have that
	\begin{equation*}
		(f_\beta \circ \gamma)(t) = \beta + i\frac{1}{t}.
	\end{equation*}
	Thus $f_0$ maps $A$ to the positive imaginary axis. Next consider $A$ to be a semicircle with radius $r > 0$ and center on the real axis. Let $\alpha \in \Rbb$ denote the left thought point of intersection of $A$ with the real axis. Then $A$ is the image of the curve $\gamma : \intoo{0,\pi} \to \Hbb$ defined by $\gamma(t) := re^{it} + \alpha + r$. For any $t \in \intoo{0,\pi}$ we have 
	\begin{align*}
		(f_\beta \circ \gamma)(t) &= -\frac{1}{r}\frac{1}{e^{it} + 1} + \beta = -\frac{1}{2r}\frac{e^{-it} + 1}{1 + \cos t} + \beta = -\frac{1}{2r} + \frac{i}{2r} \frac{\sin t}{1 + \cos t} + \beta.
	\end{align*}
	Hence $f_{1/2r}$ maps $A$ to the positive imaginary axis.
\end{proof}

\begin{theorem}[Geodesics of the Hyperbolic Plane]
	The geodesics in $\Hbb$ are precisely the semicircles with center on the real axis and the straight lines parallel to the imaginary axis. Furthermore, through any two points belonging to $\Hbb$ there is exactly one geodesic segment connecting them. 
	\label{thm:geodesics}
\end{theorem}

\begin{proof}
	Let $a,b \in \intoo{0,\infty}$ such that $a < b$ and $\gamma := x(t) + iy(t) \in \Cscr^1_{\pwrm}(I,\Hbb)$ joining $ia$ and $ib$. Then we have that
	\begin{equation*}
		\h(\gamma) =  \int_0^1 \frac{\sqrt{\del[1]{\frac{dx}{dt}}^2 + \del[1]{\frac{dy}{dt}}^2}}{y(t)} dt \geq  \int_0^1 \frac{\frac{dy}{dt}}{y(t)} dt = \int_a^b \frac{dy}{y} = \log \frac{b}{a} = \h(\gamma_0)
	\end{equation*}
	\noindent where $\gamma_0 := (1 - t)ia + tib$ is the straight line on the imaginary axis from $ia$ to $ib$.\\
	Let $z,w \in \Hbb$. Then there exists a unique circle with center on the real axis or a unique line parallel to the imaginary axis which goes through those points, say $A$. By proposition \ref{prop:mapping_to_imaginary_axis} there exists $f \in \Mob(\Hbb)$ such that $f(A)$ is a subset of the positive imaginary axis. The statement now follows from proposition \ref{prop:inclusion} and the above.
\end{proof}

\begin{definition}
	Let $z,w \in \Hbb$. The unique geodesic segment joining them is denoted by $\sbr[0]{z,w}$.
\end{definition}

\begin{corollary}
	Let $z,w \in \Hbb$ with $z \neq w$. Then
	\begin{equation}
		\rho(z,w) = \rho(z,\xi) + \rho(\xi,w)
	\end{equation}
	\noindent holds if and only if $\xi \in \sbr[0]{z,w}$. Moreover, any transformation in $\Mob(\Hbb)$ maps geodesics onto geodesics.
\end{corollary}

\begin{theorem}[Explicit Formulas for the Hyperbolic Distance]
	For $z,w \in \Hbb$ we have that
	\begin{enumerate}[label = \textup{(}\alph*\textup{)}]
		\item 
			\begin{equation}
				\rho(z,w) = \log \frac{\abs[0]{z - \wbar{w}} + \abs[0]{z - w}}{\abs[0]{z - \wbar{w}} - \abs[0]{z - w}};
			\end{equation}
		\item
			\begin{equation}
				\cosh \rho(z,w) = 1 + \frac{\abs[0]{z - w}^2}{2 \Im(z)\Im(w)};
			\end{equation}
		\item
			\begin{equation}
				\sinh \PARENS{\frac{1}{2}\rho(z,w)} = \frac{\abs[0]{z - w}}{2\sqrt{\Im(z)\Im(w)}};
			\end{equation}
		\item
			\begin{equation}
				\cosh \PARENS{\frac{1}{2}\rho(z,w)} = \frac{\abs[0]{z - \wbar{w}}}{2\sqrt{\Im(z)\Im(w)}};
			\end{equation}
		\item
			\begin{equation}
				\tanh \PARENS{\frac{1}{2}\rho(z,w)} = \abs[3]{\frac{z - w}{z - \wbar{w}}}.
			\end{equation}
	\end{enumerate}
	\label{thm:formulas}
\end{theorem}

\begin{proof}
	Using the definitions of the hyperbolic functions and the identities
	\begin{align*}
		& \sinh^2 \frac{x}{2} = \frac{\cosh x - 1}{2};\\
		& \cosh x = \sqrt{1 + \sinh^2 x};\\
		& \sinh x = \sgn x \sqrt{\cosh^2 x - 1};\\
		& \arctanh x = \frac{1}{2} \log \frac{1 + x}{1 - x}
	\end{align*}
	\noindent it is tedious to show that (a)$\Rightarrow$(b)$\Rightarrow$(c)$\Rightarrow$(d)$\Rightarrow$(e)$\Rightarrow$(a). We shall prove (c). By proposition \ref{prop:inclusion} we have that the left-hand side is invariant under $f \in \Mob(\Hbb)$. We show that also the right-hand side is invariant under $f$. Equation (\ref{eq:imaginary_part}) and some algebraic manipulations immediately yield
	\begin{equation*}
		\frac{\abs[0]{f(z) - f(w)}}{2\sqrt{\Im(f(z))\Im(f(w))}} = \frac{\abs[0]{z - w}}{2\sqrt{\Im(z)\Im(w)}}
	\end{equation*}
	Hence by an application of proposition \ref{prop:mapping_to_imaginary_axis} we may assume that $z = ia$ and $w = ib$ for $a,b \in \intoo{0,\infty}$, $a < b$. Theorem \ref{thm:geodesics} yields $\rho(z,w) = \log \frac{b}{a}$ and it is easy to see that the equality in (c) holds.
\end{proof}

Recall that $\what{\Cbb} := \Cbb \cup \cbr[0]{\infty}$ denotes the \emph{Riemann sphere} and $\what{\Rbb} := \Rbb \cup \cbr[0]{\infty}$.

\begin{definition}[Cross-Ratio]
	The \bld{cross-ratio} of distinct points $z_1,z_2,z_3,z_4 \in \what{\Cbb}$ is defined to be
	\begin{equation}
		(z_1,z_2;z_3,z_4) := \frac{(z_1 - z_2)(z_3 - z_4)}{(z_2 - z_3)(z_4 - z_1)}.
	\end{equation}
\end{definition}

The next theorem uses results about fractional linear transformations and thus we only state but do not prove it.

\begin{theorem}
	Let $z,w \in \Hbb$ be distinct and let the geodesic joining $z$ and $w$ have endpoints $z^*,w^* \in \what{\Rbb}$, choosen in such a way that $z$ lies between $z^*$ and $w$. Then
	\begin{equation}
		\rho(z,w) = \log (w,z^*;z,w^*).
	\end{equation}
\end{theorem}

\begin{definition}[Poincar\'e Ball Model]
	The unit disk $\Ebb := \cbr[0]{z \in \Cbb : \abs[0]{z} < 1}$ equipped with the metric
	\begin{equation}
		\wtilde{g} := 4\frac{\abs[0]{dz}^2}{\PARENS{1 - \abs[0]{z}^2}^2}
	\end{equation}
	\noindent is a model for the hyperbolic plane, the \bld{Poincar\'e ball model}.
\end{definition}

Recall, that if $(M,g)$ and $(\wtilde{M},\wtilde{g})$ are two Riemannian manifolds, a diffeomorphism $F : M \to \wtilde{M}$ is said to be a \emph{Riemannian isometry} if $F^* \wtilde{g} = g$. If there exists a Riemannian isometry for two Riemannian manifolds, they are said to be \emph{isometric}.

\begin{proposition}
	$(\Hbb, g)$ and $(\Ebb, \wtilde{g})$ are isometric.
\end{proposition}

\begin{proof}
	Consider the \bld{generalized Cayley transform} $f : \Hbb \to \Ebb$ defined by 
	\begin{equation}
		f(z) := \frac{iz + 1}{z + i}.
	\end{equation}
	Since $f$ is a fractional linear transformation, it is clearly a diffeomorphism. Moreover
	\begin{equation*}
		f'(z) = -\frac{2}{(z + i)^2} \qquad \text{and} \qquad 1 - \abs[0]{f(z)}^2 = 4\frac{\Im(z)}{\abs[0]{z + i}^2}.	
	\end{equation*}
	Therefore
	\begin{equation*}
		f^*\wtilde{g} = 4 \frac{f^*(dzd\wbar{z})}{\PARENS{1 - \abs[0]{f(z)}^2}^2} = 4 \frac{\abs[0]{f'(z)}^2dzd\wbar{z}}{\PARENS{1 - \abs[0]{f(z)}^2}^2} = \frac{dzd\wbar{z}}{(\Im(z))^2} = g
	\end{equation*}
	\noindent by \cite[41]{lee:Riemannian_manifolds:1997}.
\end{proof}

As an application of the ball model we give the following proposition.

\begin{proposition}
$(\Hbb,\rho)$ is a metric space whose metric topology is the same as the standard topology on $\Hbb$ (i.e. the metric topology induced by the Euclidean metric).
\label{prop:metric}
\end{proposition}

\begin{proof}
Let $z,w \in \Hbb$. Then clearly $\rho(z,w) \geq 0$ and by parametrization invariance we get that $\rho(z,w) = \rho(w,z)$. Also the triangle inequality holds. Also if $z = w$, we have that $\rho(z,w) = 0$. Assume that $z \neq w$. Then there exists a unique geodesic segment $\sbr[0]{z,w}$ joining them. But $\h(\sbr[0]{z,w}) > 0$ as one can easily see. Thus $\rho$ is indeed a metric.\\
Left to show is that the metric topology induced by the hyperbolic distance function coincides with the usual Euclidean topology. However, since the proof is quite long if one to do it formally, we only state a reference. By \cite[127]{anderson:hyperbolic_geometry:2005} any hyperbolic circle in $\Ebb$ is a Euclidean one and vice versa (with different centers in general). Then exercise 4.7 \cite[129]{anderson:hyperbolic_geometry:2005} shows that if $x + iy \in \Hbb$ is a Euclidean center of a Euclidean circle with radius $r$, then the hyperbolic center is given by $x + i \sqrt{y^2 - r^2}$ and the hyperbolic radius $R$ satisfies $r = y \tanh(R)$. From this it follows that Euclidean discs and hyperbolic disks coincide (with possibly different radius and center). However, since the set of all disks in a metric space is a basis for the topology induced by the metric, we have that the Euclidean and the hyperbolic topologies are the same.
\end{proof}

\begin{proposition}
Let $f \in \Isom(\Hbb)$. Then $f$ is continuous.
\label{prop:continuous}
\end{proposition}

\begin{proof}
By proposition \ref{prop:metric} it is enough to show that $f$ is continuous with respect to the hyperbolic metric. But this is immediate since $\rho\PARENS{f(z),f(w)} = \rho(z,w)$ for all $z,w \in \Hbb$ implies that $f$ is Lipschitz continuous.
\end{proof}

\begin{definition}[Euclidean Boundaries and Closures]
	Define
		\begin{center}
		\begin{tabular}{ll}
			$\partial \Ebb := \Sigma := \cbr[0]{z \in \Cbb : \abs[0]{z} = 1}$ & (\textbf{Principal Circle, Boundary of $\Ebb$}),\\
			$\partial \Hbb := \what{\Rbb}$ & (\textbf{Boundary of $\Hbb$ in $\what{\Cbb}$}),\\
			$\wbar{\Ebb} := \Ebb \cup \Sigma$ & (\textbf{Closure of $\Ebb$}),\\
			$\wbar{\Hbb} := \Hbb \cup \what{\Rbb}$ & (\textbf{Closure of $\Hbb$}).\\
		\end{tabular}
	\end{center}
\end{definition}

\subsection{Isometries}
We have already seen that M\"obius transformations are isometries on $\Hbb$. Is is also true that $\Isom(\Hbb) \subseteq \Mob(\Hbb)$? Unfortunately, the answer is no. However, $\Isom(\Hbb) \setminus \Mob(\Hbb)$ is not as complicated as one might think. Roughly speaking, the isometries on $\Hbb$ are the union of so-called orientation-preserving, i.e. M\"obius transformations and orientation-reversing transformations.

\begin{proposition}
	$(\Isom(\Hbb),\circ)$ is a group.
\end{proposition}

\begin{proof}
	We show $\Isom(\Hbb) \leq S_\Hbb$. Let $f \in \Isom(\Hbb)$. By definition,	$f$ is surjective. Let $z,w \in \Hbb$ such that $f(z) = f(w)$. Then 
	\begin{equation*}
		\rho(z,w) = \rho\PARENS{f(z),f(w)} = 0
	\end{equation*}
	\noindent which implies that $z = w$ since $\rho$ is a metric by proposition \ref{prop:metric}. Thus $f$ is bijective. Clearly $\id_\Hbb \in \Isom(\Hbb)$ and for $g \in \Isom(\Hbb)$ we have that
	\begin{equation*}
		\rho\PARENS{(g \circ f)(z),(g\circ f)(w)} = \rho\PARENS{f(z),f(w)} = \rho(z,w) 
	\end{equation*}
	\noindent and so $g \circ f \in \Isom(\Hbb)$. Moreover $f^{-1}$ is distance preserving by
	\begin{equation*}
		\rho\PARENS{f^{-1}(z),f^{-1}(w)} = \rho\PARENS{(f \circ f^{-1})(z), (f \circ f^{-1})(w)} = \rho(z,w).
	\end{equation*}
\end{proof}

\begin{definition}
	Let $\det : \Mrm(2,\Rbb) \to \Rbb$ denote the determinant mapping. Define 
	\begin{equation}
		\PSLrmstar(2,\Rbb) := \SLrmstar(2,\Rbb)/\cbr[0]{\pm I}.
	\end{equation}
	\noindent where $\SLrmstar(2,\Rbb) := \det^{-1}(\pm 1)$.
\end{definition}

\begin{proposition}
	$\PSLrm(2,\Rbb) \leq \PSLrmstar(2,\Rbb)$ and $\sbr[0]{\PSLrmstar(2,\Rbb) : \PSLrm(2,\Rbb)} = 2$.
	\label{prop:index}
\end{proposition}

\begin{proof}
	Clearly $\PSLrm(2,\Rbb) \leq \PSLrmstar(2,\Rbb)$. We claim that
	\begin{equation*}
		\cbr[0]{A \cdot \PSLrm(2,\Rbb) : A \in \PSLrmstar(2,\Rbb)} = \cbr[0]{\PSLrm(2,\Rbb), \pi(E) \cdot \PSLrm(2,\Rbb)}
	\end{equation*}
	\noindent for 
	\begin{equation*}
		E := \begin{pmatrix}
			-1 & 0\\
			0 & 1
		\end{pmatrix}.
	\end{equation*}
	The inclusion $\supseteq$ trivially holds. So assume that $A \cdot \PSLrm(2,\Rbb)$ belongs to the right-hand side. Since $A \in \PSLrmstar(2,\Rbb)$, we have that either $A = A_1\cbr[0]{\pm I}$ where $\det(A_1) = 1$ or $A = A_2 \cbr[0]{\pm I}$ where $\det(A_2) = -1$. In the first case simply $A \in \PSLrm(2,\Rbb)$ and in the second we write $A = E(EA_2)\cbr[0]{\pm I}$. Then $\det(EA_2) = \det(E)\det(A_2) = 1$ and thus $A \in E \cdot \PSLrm(2,\Rbb)$.
\end{proof}

\begin{theorem}[Isometries of the Hyperbolic Plane]
	We have that
	\begin{equation}
		\Isom(\Hbb) = \Mob(\Hbb) \cup \cbr[0]{z \mapsto A \cdot \wbar{z} : A \in \det^{-1}(-1)}.
	\end{equation}
	Furthermore
	\begin{equation}
		\Isom(\Hbb) \cong \PSLrmstar(2,\Rbb)
	\end{equation}
	\noindent and $\sbr[0]{\Isom(\Hbb) : \Mob(\Hbb)} = 2$.
\end{theorem}

\begin{proof}
	Let $f \in \Isom(\Hbb)$ and $I$ denote the positive imaginary axis. Then $f(I)$ is a geodesic. By proposition \ref{prop:mapping_to_imaginary_axis} there exists $g \in \Mob(\Hbb)$ such that $g(f(I)) \subseteq I$. Assume that $(g \circ f)(i) \neq i$. Thus $(g \circ f)(i) = ia$, for some $a \in \Rbb_{>0}$. Hence applying some dilation $z \mapsto \frac{z}{a}$ fixes $i$. Also we can force $g \circ f$ to map $(i,\infty)$ and $(0,i)$ by composing with $z \mapsto -\frac{1}{z}$. By proposition \ref{prop:inclusion} those two maps are isometries. Hence we may assume that $g \circ f$ fixes $I$. Let $z := x + iy \in \Hbb$ and $u + iv := (g \circ f )(z)$. Furthermore let $t > 0$. Then
	\begin{equation*}
		\rho(z,it) = \rho\PARENS{(g \circ f)(z),(g \circ f)(it)} = \rho(u + iv, it).
	\end{equation*}
	Applying theorem \ref{thm:formulas} (c) yields
	\begin{equation*}
		v\PARENS{x^2 + (y - t)^2} = y\PARENS{u^2 + (v - t)^2}
	\end{equation*}
	Thus dividing both sides by $t^2$ and letting $t \to \infty$ leads to $v = y$. But if $v = y$ the above equation gives that $x^2 = u^2$. Hence $u = \pm x$ and so we get that
	\begin{equation*}
		(g \circ f)(z) = z \qquad \text{or} \qquad (g \circ f)(z) = - \wbar{z}.
	\end{equation*}
	By proposition \ref{prop:continuous}, any isometry is continuous. Thus we have that either $g \circ f = \id_\Hbb$ or $g \circ f = -\wbar{\id}_\Hbb$. If $g \circ f = \id_\Hbb$, we have that $f = g^{-1} \in \Mob(\Hbb)$. In the other case we have that $f = g^{-1} \circ -\wbar{\id}_\Hbb$.\\
	Consider a mapping $\Phi : \SLrmstar(2,\Rbb) \to \Isom(\Hbb)$ defined by 
	\begin{equation*}
		\Phi(A) \coloneq \ccases{z \mapsto \pi(A) \cdot z & \det(A) = 1,\\
			z \mapsto A \cdot \wbar{z} & \det(A) = -1.}
	\end{equation*}
	This clearly defines a group homomorphism. We claim $\ker \Phi = \cbr[0]{\pm I}$. Let $A \in \ker \Phi$. If $\det(A) = 1$, we have that 
	\begin{equation*}
		\frac{az + b}{cz + d} = z
	\end{equation*}
	\noindent for all $z \in \Hbb$ or equivalently
	\begin{equation*}
		cz^2 + (d - a)z - b = 0
	\end{equation*}
	\noindent for all $z \in \Hbb$. The fundamental theorem of algebra now implies that $c = b = 0$ and $a = d$ which yields $a = d = \pm 1$. Moreover $\Phi$ is 
	clearly surjective and thus the first isomorphism theorem \cite[23]{grillet:abstract_algebra:2007} yields
	\begin{equation*}
		\Isom(\Hbb) \cong \PSLrmstar(2,\Rbb).
	\end{equation*}
	Finally, by proposition \ref{prop:index} we have that $\sbr[0]{\PSLrmstar(2,\Rbb) : \PSLrm(2,\Rbb)} = 2$ and therefore $\Mob(\Hbb) \cong \PSLrm(2,\Rbb)$ yields $\sbr[0]{\Isom(\Hbb) : \Mob(\Hbb)} = 2$.
\end{proof}

\begin{definition}
	Let $f \in \Isom(\Hbb)$ and let $A \in \PSLrmstar(2,\Rbb)$ be the corresponding matrix. Then $\det(A)$ is called the \bld{orientation} of the isometry $f$. If $\det(A) = 1$, $f$ is said to be a \bld{orientation preserving} isometry, while if $\det(A) = -1$, $f$ is said to be a \bld{orientation reversing} isometry.
\end{definition}

\printbibliography

\end{document}
