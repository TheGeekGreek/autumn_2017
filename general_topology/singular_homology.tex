\chapter{Singular Homology}
\section{Free Abelian Groups}

\begin{proposition}
	The forgetful functor $U : \mathsf{Ab} \to \mathsf{Set}$ admits a left adjoint.
	\label{prop:free_adj_forget}
\end{proposition}

\begin{proof}
	We have to construct a functor $F : \mathsf{Set} \to \mathsf{Ab}$. Let $S$ be a set. Define 
	\begin{equation*}
		F(S) := \cbr[1]{f \in \mathbb{Z}^S : \supp f \text{ is finite}}.
	\end{equation*}
	Equipped with pointwise addition, $F(S)$ is an abelian group. There is a natural inclusion $\iota : S \hookrightarrow U\del[1]{F(S)}$ sending $x \in S$ to the function taking the value one at $x$ and zero else. Hence we may regard elements of $F(S)$ as formal linear combinations $\sum_{x \in S}m_x x$, where $m_x \in \mathbb{Z}$ for all $x \in S$. Let $G \in \ob(\mathsf{Ab})$ be an abelian group and $\varphi \in \mathsf{Ab}\del[1]{F(S), G}$ a morphism of groups. Define $\wbar{\varphi} \in \mathsf{Set}\del[1]{S,U(G)}$ by $\wbar{\varphi} := U(\varphi) \circ \iota$. Conversly, if we have $f \in \mathsf{Set}\del[1]{S,U(G)}$, define $\wbar{f} \in \mathsf{Ab}\del[1]{F(S),G}$ by $\wbar{f}\del[1]{\sum_{x \in S} m_x x} := \sum_{x \in S} m_x f\del[1]{\iota^{-1}(x)}$. This is well defined since all but finitely many $m_x$ are zero. It is easy to check that $\wbar{f}$ is indeed a morphism of groups. Let $\varphi \in \mathsf{Ab}\del[1]{F(S),G}$. Then
	\begin{align*}
		\wbar{\wbar{\varphi}}\del[4]{\sum_{x \in S} m_x x} &= \sum_{x \in S}m_x \wbar{\varphi}\del[1]{\iota^{-1}(x)}\\
		&= \sum_{x \in S} m_x \del[1]{U(\varphi) \circ \iota}\del[1]{\iota^{-1}(x)}\\
		&= \sum_{x \in S} m_x U(\varphi)(x)\\
		&= \sum_{x \in S} m_x \varphi(x)\\
		&= \varphi \del[4]{\sum_{x \in S} m_x x}.
	\end{align*}
	And for $f \in \mathsf{Set}\del[1]{S,U(G)}$ we have that
	\begin{equation*}
		\wbar{\wbar{f}}(x) = \del[1]{U(\wbar{f}) \circ \iota}(x) = \wbar{f}\del[1]{\iota(x)} = f(x). 
	\end{equation*}
	\noindent Hence $\wbar{\wbar{\varphi}} = \varphi$ and $\wbar{\wbar{f}} = f$ and so we have a bijection
	\begin{equation*}
		\mathsf{Ab}\del[1]{F(S),G} \cong \mathsf{Set}\del[1]{S,U(G)}.
	\end{equation*}
	The mapping $f \mapsto \wbar{f}$ will be referred to as \bld{extending by linearity}. To check naturality in $S$ and $G$ is left as an exercise.
\end{proof}

\begin{exercise}
	Check the naturality of the bijection in proposition \ref{prop:free_adj_forget}. Also check that $F : \mathsf{Set} \to \mathsf{Ab}$ is indeed a functor. $F$ is called the \bld{free functor from $\mathsf{Set}$ to $\mathsf{Ab}$}.	
\end{exercise}

\begin{definition}[Free Abelian Group]
	Let $F : \mathsf{Set} \to \mathsf{Ab}$ be the free functor. For any set $S$, we call $F(S)$ the \bld{free group generated by $S$}.
\end{definition}
