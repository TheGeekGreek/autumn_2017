\section*{The Jordan-Brouwer Separation Theorem}

\begin{lemma}
	\label{lem:topologizing_a_set}
	Let $X \in \ob(\mathsf{Top})$, $S \in \ob(\mathsf{Set})$ and $f : U(X) \to S$ a bijection. Then $S$ can be equipped with a topology such that $f$ becomes a homeomorphism.
\end{lemma}

\begin{proof}
	Let $\mathcal{T}$ be the topology on $X$. Then it is easy to see, that 
	\begin{equation*}
		\varphi(\mathcal{T}) := \cbr[0]{\varphi(U) : U \in \mathcal{T}}
	\end{equation*}
	\noindent is the right topology on $S$.
\end{proof}

\begin{lemma}
	Given a sequential diagram 
	\begin{equation*}
		\begin{tikzcd}
			X_0 \arrow[r,hook,"\iota_0"] & X_1 \arrow[r,hook,"\iota_1"] & X_2 \arrow[r,hook,"\iota_2"] & \cdots
		\end{tikzcd}
	\end{equation*}
	\noindent in $\mathsf{Top}$, we have that $\colim_n X_n \cong \bigcup_{n \in \omega}X_n$.
\end{lemma}

\begin{proof}
	Using the construction in proposition \ref{prop:filtered_colimits_set}, we have that a sequential colimit  in $\mathsf{Set}$ is given by $\coprod_{n \in \omega}X_n /{\sim}$, where it is easy to check that in this case $x \in X_n \sim y \in X_m$ if and only if $x = y$. Moreover, we have that $\coprod_{n \in \omega} X_n/{\sim} \cong \bigcup_{n \in \omega}X_n$ in $\mathsf{Set}$, as one can easily show by considering the map $\sbr[0]{x_n} \mapsto x_n$. Using lemma \ref{lem:topologizing_a_set}, we get that $\coprod_{n \in \omega} X_n/{\sim} \cong \bigcup_{n \in \omega}X_n$ in $\mathsf{Top}$. Moreover, it is easy to check that a set $U \subseteq \bigcup_{n \in \omega}X_n$ is open if and only if $U \cap X_n$ is open in $X_n$ for all $n \in \omega$.
\end{proof}

\begin{definition}[$T_1$]
	A topological space $X$ is said to be a \bld{$T_1$-space}, if $\cbr{x}$ is closed in $X$ for every $x \in X$.
\end{definition}

\begin{definition}[Weakly Hausdorff]
	A topological space $X$ is said to be a \bld{weakly Hausdorff space}, if for any map $f \in \mathsf{Top}(K,X)$ for a compact Hausdorff space $K$, $f(K)$ is closed in $X$.
\end{definition}

\begin{exercise}
	\label{ex:inclusions_Hausdorff}
	Show that any Hausdorff space is a weakly Hausdorff space and that any weakly Hausdorff space is a $T_1$-space, but both contraries are not true.
\end{exercise}

\begin{exercise}
	\label{ex:compact_subspace}
	Let $X$ be a weakly Hausdorff space. Assume that $f \in \mathsf{Top}(K,X)$ for a compact Hausdorff space $K$. Show that $f(K)$ is a compact subspace of $X$.
\end{exercise}

\begin{proposition}
	Given a sequence 
	\begin{equation*}
		\begin{tikzcd}
			X_0 \arrow[r,"i_0"] & X_1 \arrow[r,"i_1"] & X_2 \arrow[r,"i_2"] & \dots
		\end{tikzcd}
	\end{equation*}
	\noindent of closed embeddings of weakly Hausdorff spaces, then
	\begin{equation*}
		\textstyle\colim_n C_\bullet(X_n) = C_\bullet \del[1]{\colim_nX_n}.
	\end{equation*}
\end{proposition}

\begin{proof}
	Let $X := \bigcup_{n \in \omega} X_n$. The main part is the following lemma:

	\begin{lemma}
		Let $f \in \mathsf{Top}(K,X)$ for a compact Hausdorff space $K$. Then $f(K)$ is contained in one of the $X_n$.	
	\end{lemma}

	\begin{proof}
		Towards a contradiction, assume that $f(K)$ is not conatined in any $X_n$. Hence we find a sequence $(x_n)_{n \in \omega}$ in $K$, such that $f(x_n) \notin X_n$ for all $n \in \omega$. Define
		\begin{equation*}
			S_m := \cbr[0]{f(x_k) : k \geq m},
		\end{equation*}
		\noindent for $m \in \omega$. Then $S_{m + 1} \subseteq S_m$, $\bigcap_{m \in \omega} S_m = \varnothing$ and $S_m \cap X_n$ is finite for all $n \in \omega$. By exercise \ref{ex:inclusions_Hausdorff}, we get that $S_m \cap X_n$ is closed in $X_n$ for all $n \in \omega$. Hence by the definition of the colimit topology, $S_m$ is closed in $X$ for all $m \in \omega$. Thus $Y_m := X \setminus S_m$ is open in $X$ and easily seen to be an open cover for $X$. By construction, no finite subcover of it can cover $f(K)$, and hence we have a contradiction to the fact that $f(K)$ is compact by exercise \ref{ex:compact_subspace}.
	\end{proof}\\
	By the previous lemma, any singular $n$-simplex $\sigma : \Delta^n \to X$ is contained in some $X_n$ and so the result follows from the definition of the colimit in $\mathsf{Comp}$.
\end{proof}

\begin{proposition}
	\label{prop:I_m}
	Let $X \subseteq \mathbb{S}^n$ homeomorphic to $I^m$, $0 \leq m \leq n$. Then $\wtilde{H}_k(\mathbb{S}^n \setminus X) = 0$ for $k \in \omega$.
\end{proposition}

\begin{proof}
	If $m = 0$, then $X$ is a single point in $\mathbb{S}^n$. Hence $\mathbb{S}^n \setminus X \cong \mathbb{R}^n$ and thus $\wtilde{H}_k(\mathbb{S}^n \setminus X) = 0$, since $\mathbb{R}^n$ is contractible. Now let $0 < m \leq n$ and suppose the claim holds for $m - 1$. Let $f : X \to I^m$ be a homeomorphism and define 
	\begin{equation*}
		I^+ := \cbr[0]{x \in I^m : x_1 \geq 1/2} \qquad \text{and} \qquad I^- := \cbr[0]{x \in I^m : x_1 \leq 1/2}.
	\end{equation*}
	Moreover, define $X^{\pm} := f^{-1}(I^{\pm})$ and $Y := X^+ \cap X^-$. Then $Y \approx I^{m - 1}$ and $\mathbb{S}^n \setminus X^+$, $\mathbb{S}^n \setminus X^-$ is an open cover for $\mathbb{S}^n \setminus Y$. Since $\mathbb{S}^n \setminus X^+ \cap \mathbb{S}^n \setminus X^- = \mathbb{S}^n \setminus X$, by Mayer-Vietoris we get a long exact sequence in reduced homology:
	\begin{equation*}
		\begin{tikzcd}[column sep=2ex]
			\dots\wtilde{H}_{k + 1}(\mathbb{S}^n \setminus Y) \arrow[r] & \wtilde{H}_k(\mathbb{S}^n\setminus X) \arrow[r] & \wtilde{H}_k(\mathbb{S}^n \setminus X^+) \oplus \wtilde{H}_k(\mathbb{S}^n \setminus X^-) \arrow[r] & \wtilde{H}_k(\mathbb{S}^n \setminus Y)\dots
		\end{tikzcd}
	\end{equation*}
	By hypothesis, the end terms vanish and thus we get an isomorphism
	\begin{equation*}
		\begin{tikzcd}[column sep=15ex]
			\wtilde{H}_k(\mathbb{S}^n\setminus X) \arrow[r,"{(H_k(\iota^+),H_k(\iota^-))}"] & \wtilde{H}_k(\mathbb{S}^n \setminus X^+) \oplus \wtilde{H}_k(\mathbb{S}^n \setminus X^-).
		\end{tikzcd}
	\end{equation*}
	Now take some nonzero element $\langle c \rangle \in \wtilde{H}_k(\mathbb{S}^n \setminus X)$ (if there exists no nonzero element, we are done). Since we have an isomorphism, either $H_k(\iota^+)\langle c \rangle$ or $H_k(\iota^-)\langle c \rangle$ must be nonzero. Without loss of generality, assume $H_k(\iota^+)\langle c \rangle \neq 0$. In the same manner we can split $X^+$ in two parts and thus getting a decreasing sequence $(X_n)_{n \in \omega}$ of closed subsets of $\mathbb{S}^n$ such that $\langle c \rangle$ is taken to $\langle c_n \rangle \neq 0$ by the homomorphism on homology induced by inclusion. Now each $\mathbb{S}^n \setminus X_n$ is open, and thus we get
	\begin{equation*}
		\textstyle\wtilde{H}_k(\mathbb{S}^n \setminus \bigcap_{n \in \omega} X_n) = \colim_j\wtilde{H}_k(\mathbb{S}^n\setminus X_j).
	\end{equation*}
	Since $\wtilde{H}_k(\mathbb{S}^n \setminus X_j) \to \wtilde{H}_k(\mathbb{S}^n \setminus X_{j + 1})$ sends $\langle c_j \rangle$ to $\langle c_{j + 1} \rangle$, we end up with a nonzero element $\langle c_\infty \rangle$ in $\wtilde{H}_k(\mathbb{S}^n \setminus \bigcap_{n \in \omega}X_n)$. But $\bigcap_{n \in \omega} X_n$ is homeomorphic to $I^{m - 1}$. Hence by hypothesis, $\wtilde{H}_k(\mathbb{S}^n \setminus \bigcap_{n \in \omega}X_n)$, contradicting the existence of a nonzero element $\langle c_\infty \rangle$. 
	ntradiction.
\end{proof}

\begin{corollary}
	\label{cor:S_m}
	Let $X \subseteq \mathbb{S}^n$ be homeomorphic to $\mathbb{S}^m$ for some $0 \leq m \leq n - 1$. Then
	\begin{equation*}
		\wtilde{H}_k(\mathbb{S}^n \setminus X) = \ccases{
			\mathbb{Z} & k = n - m - 1,\\
			0 & k \neq n - m - 1.
		}
	\end{equation*}
\end{corollary}

\begin{proof}
	Let $m = 0$. Then $X$ consists of two distinct points and thus $\mathbb{S}^n \setminus X \approx \mathbb{S}^{n - 1}$. Hence 
	\begin{equation*}
		\wtilde{H}_k(\mathbb{S}^n \setminus X) = \wtilde{H}_k(\mathbb{S}^{n - 1}) = \ccases{
			\mathbb{Z} & k = n - 1,\\
			0 & k \neq n - 1.
		}
	\end{equation*}
	Now suppose the claim holds for $m - 1$. Then if $X \approx \mathbb{S}^m$, write $X = X^+ \cup X^-$, where $X^{\pm}$ is homeomorphic to the upper an lower hemisphere of $\mathbb{S}^m$. Applying Mayer-Vietoris to the cover $\mathbb{S}^n \setminus X^+$ and $\mathbb{S}^n \setminus X^-$ of $\mathbb{S}^n \setminus Y$, where $Y := X^+ \cap X^-$, yields a long exact sequence
	\begin{equation*}
		\begin{tikzcd}
			\dots\wtilde{H}_{k + 1}(\mathbb{S}^n \setminus X^+) \oplus \wtilde{H}_{k + 1}(\mathbb{S}^n\setminus X^-) \arrow[d, phantom, ""{coordinate, name=Z}]\arrow[r] & \wtilde{H}_{k + 1}(\mathbb{S}^n\setminus Y) \arrow[dl,rounded corners,
				to path=
				{ -- ([xshift=2ex]\tikztostart.east)
				|- (Z) [near end]\tikztonodes
				-| ([xshift=-2ex]\tikztotarget.west)
				-- (\tikztotarget)}
			] \\
			\wtilde{H}_k(\mathbb{S}^n \setminus X) \arrow[r] & \wtilde{H}_k(\mathbb{S}^n \setminus X^+) \oplus \wtilde{H}_k(\mathbb{S}^n\setminus X^-)\dots
		\end{tikzcd}
	\end{equation*}
	The end terms of above sequence both vanish by proposition \ref{prop:I_m} and thus we get an isomorphism
	\begin{equation*}
		\wtilde{H}_k(\mathbb{S}^n \setminus X) \cong \wtilde{H}_{k - 1}(\mathbb{S}^n \setminus Y).
	\end{equation*}
	Since $Y \approx \mathbb{S}^{m - 1}$, we are done.
\end{proof}

\begin{theorem}[The Jordan-Brouwer Separation Theorem]
	Let $f \in \mathsf{Top}(\mathbb{S}^{n-1},\mathbb{S}^n)$ is an embedding. Then $\mathbb{S}^n \setminus f(\mathbb{S}^{n - 1})$ has two components, where $f(\mathbb{S}^{n - 1})$ is the boundary of each component.
\end{theorem}

\begin{proof}
	Define $X := f(\mathbb{S}^{n - 1})$. Hence corollary \ref{cor:S_m} yields
	\begin{equation*}
		\wtilde{H}_k(\mathbb{S}^n \setminus X) = \ccases{
			\mathbb{Z} & k = 0,\\
			0 & k > 0.
		}
	\end{equation*}
	Thus the dimension axiom together with additivity on path components implies that $\mathbb{S}^n \setminus X$ has two path-components. Since $X$ is closed by the closed map lemma, $\mathbb{S}^n \setminus X$ is open, and since $\mathbb{S}^n$ is a topological manifold, $\mathbb{S}^n$ is locally path-connected and thus by \cite[93]{lee:topological_manifolds:2011}, $\mathbb{S}^n \setminus X$ is locally path-connected and hence path-components and components coincide.\\
	Let $X_1$ and $X_2$ denote the two components of $\mathbb{S}^n \setminus X$. Since $X \cup X_1$ is closed, the boundary of $\partial X_1$ is contained in $X$. Conversly, suppose $p \in X$ and let $U$ be a neighbourhood of $p$ in $\mathbb{S}^n$. Now there exists some $V \subseteq X \cap U$, such that $p \in V$ and $X \setminus V \approx \mathbb{B}^{n - 1}$ ($X$ is a copy of $\mathbb{S}^{n - 1}$). Since $\mathbb{B}^{n - 1} \approx \intcc{0,1}^{n - 1}$, we get by proposition \ref{prop:I_m}, that $\wtilde{H}_0(\mathbb{S}^n \setminus (X \setminus V)) = 0$ and hence the dimension axiom yields that $\mathbb{S}^n \setminus (X \setminus V)$ has one path component. Now for $x_1 \in X_1$ and $x_2 \in X_2$ we find a path with image in $\mathbb{S}^n \setminus (X \setminus V)$, since $X_1$ and $X_2$ are distinct path-components, we must have that the path intersects $V$. Hence $V$ contains points belonging to $\wbar{X}_1$ and $\wbar{X}_2$, thus $p \in \partial X_1$.
\end{proof}
