\chapter{Foundations}
\section*{Basic Category Theory}
\subsection*{Categories}

We use the first order theory of Neumann-Bernays-G\"odel (BNG) as described in \cite[231]{mendelson:logic:2015}.

\begin{definition}[Category]
	A \bld{category $\mathcal{C}$} consists of 
	\begin{itemize}[leftmargin = *]
		\item A class $\ob(\mathcal{C})$, called the \bld{objects of $\mathcal{C}$}.
		\item A class $\mor(\mathcal{C})$, called the \bld{morphisms of $\mathcal{C}$}.
		\item Two functions $\dom : \mor(\mathcal{C}) \to \ob(\mathcal{C})$ and $\cod : \mor(\mathcal{C}) \to \ob(\mathcal{C})$, which assign to each morphism $f$ in $\mathcal{C}$ its \bld{domain} and \bld{codomain}, respectively.
		\item For each $X \in \ob(\mathcal{C})$ a function $\ob(\mathcal{C}) \to \mor(\mathcal{C})$ which assigns a morphism $\id_X$ such that $\dom \id_X = \cod \id_X = X$.
		\item A function 
			\begin{equation}
				\circ : \cbr{(g,f) \in \mor(\mathcal{C}) \times \mor(\mathcal{C}) : \dom g = \cod f} \to \mor(\mathcal{C}) 
			\end{equation}
			mapping $(g,f)$ to $g \circ f$, called \bld{composition}, such that $\dom(g \circ f) = \dom f$ and $\cod(g \circ f) = \cod g$.
	\end{itemize}
	Subject to the following axioms:
	\begin{itemize}[leftmargin = *]
		\item \bld{(Associativity Axiom)} For all $f,g,h \in \mor(\mathcal{C})$ with $\dom h = \cod g$ and $\dom g = \cod f$, we have that
			\begin{equation}
				(h \circ g) \circ f = h \circ (g \circ f).
			\end{equation}
		\item \bld{(Unit Axiom)} For all $f \in \mor(\mathcal{C})$ with $\dom f = X$ and $\cod f = Y$ we have that
			\begin{equation}
				f = f \circ \id_X = \id_Y \circ f.
			\end{equation}
	\end{itemize}
\end{definition}

\begin{remark}
	Let $\mathcal{C}$ be a category. For $X,Y \in \ob(\mathcal{C})$ we will abreviate
	\begin{equation*}
		\mathcal{C}(X,Y) := \cbr{f \in \mor(\mathcal{C}) : \dom f = X \text{ and } \cod f = Y}.
	\end{equation*}
	Moreover, $f \in \mathcal{C}(X,Y)$ is depicted as
	\begin{equation}
		f : X \to Y.
	\end{equation}
\end{remark}

\begin{example}
	Let $\ast$ be a single, not nearer specified object. Consider as morphisms the class of all cardinal numbers and as composition cardinal addition. By \cite[112--113]{halbeisen:set_theory:2012}, cardinal addition is associative and $\varnothing$ serves for the identity $\id_\ast$.  
\end{example}

\begin{definition}[Locally Small, Hom-Set]
	A category $\mathcal{C}$ is said to be \bld{locally small} if for all $X,Y \in \mathcal{C}$, $\mathcal{C}(X,Y)$ is a set. If $\mathcal{C}$ is locally small, $\mathcal{C}(X,Y)$ is called a \bld{hom-set} for all $X,Y \in \mathcal{C}$. 
\end{definition}

\begin{definition}[Isomorphism]
	Let $\mathcal{C}$ be a category. An \bld{isomorphism in $\mathcal{C}$} is a morphism $f : X \to Y$ in $\mathcal{C}$, such that there exists a morphism $g : Y \to X$ in $\mathcal{C}$ with 
	\begin{equation*}
		g \circ f = \id_X \qquad \text{and} \qquad f \circ g = \id_Y.
	\end{equation*}
\end{definition}

In algebraic topology, there is a very useful construction on categories.

\begin{definition}[Congruence]
	Let $\mathcal{C}$ be a category. A \bld{congruence on $\mathcal{C}$} is an equivalence relation $\sim$ on $\mor(\mathcal{C})$ such that 
	\begin{enumerate}[label = \textup{(}\alph*\textup{)},wide = 0pt]
		\item If $f \in \mathcal{C}(X,Y)$ and $f {\sim} g$, then $g \in \mathcal{C}(X,Y)$.
		\item If $f_0 : X \to Y$ and $g_0 : Y \to Z$ such that $f_0 {\sim} f_1$ and $g_0 {\sim} g_1$, then $g_0 \circ f_0 {\sim} g_1 \circ f_1$.
	\end{enumerate}
\end{definition}

\begin{exercise}
	Let $\mathcal{C}$ be a category. Show that for any congruence on $\mathcal{C}$, there exists a category $\mathcal{C}'$, called \bld{quotient category}, with $\ob(\mathcal{C}') = \ob(\mathcal{C})$, for any objects $X,Y \in \mathcal{C}'$
	\begin{equation*}
		\mathcal{C}'(X,Y) = \cbr[0]{\sbr[0]{f} : f \in \mathcal{C}(X,Y)},
	\end{equation*}
	\noindent and pointwise composition.
\end{exercise}

\subsection*{Functors}

\begin{definition}[Functor]
	Let $\mathcal{C}$ and $\mathcal{D}$ be categories. A \bld{functor $F : \mathcal{C} \to \mathcal{D}$} is a pair of functions $(F_1,F_2)$, $F_1 : \ob(\mathcal{C}) \to \ob(\mathcal{D})$, called the \bld{object function} and $F_2 : \mor(\mathcal{C}) \to \mor(\mathcal{D})$, called the \bld{morphism function}, such that for every morphism $f : X \to Y$ we have that $F_2(f) : F_1(X) \to F_1(Y)$ and $(F_1,F_2)$ is subject to the following \bld{compatibility conditions}:
	\begin{itemize}[leftmargin = *]
		\item For all $X \in \ob(\mathcal{C})$, $F_2(\id_X) = \id_{F_1(X)}$.
		\item For all $f \in \mathcal{C}(X,Y)$ and $g \in \mathcal{C}(Y,Z)$ we have that $F_2(g \circ f) = F_2(g) \circ F_2(f)$.
	\end{itemize}
\end{definition}

\begin{remark}
	Let $F : \mathcal{C} \to \mathcal{D}$ be a functor. It is convenient to denote the components $F_1$ and $F_2$ also with $F$.
\end{remark}

\subsection*{Subcategories}
\begin{definition}[Subcategory]
	Let $\mathcal{C}$ be a category. A \bld{subcategory $\mathcal{\altS}$ of $\mathcal{C}$} consists of
	\begin{itemize}[leftmargin = *]
		\item A subclass $\ob(\mathcal{\altS}) \subseteq \ob(\mathcal{C})$.
		\item A subclass $\mor(\mathcal{\altS}) \subseteq \mor(\mathcal{C})$.
	\end{itemize}
	Subject to the following conditions:
	\begin{itemize}[leftmargin = *]
		\item For all $X \in \mathcal{\altS}$, $\id_S \in \mor(\mathcal{\altS})$.
	\end{itemize}
\end{definition}

\begin{example}[$\mathsf{Top}^2$]
	Define the objects of $\mathsf{Top}^2$ to be the class of tuple $(X,A)$, where $X \in \ob(\mathsf{Top})$ and $A$ is a subspace of $X$. Moreover, given objects $(X,A)$ and $(Y,B)$ in $\mathsf{Top}^2$, a morphism between $(X,A)$ and $(Y,B)$ is a tuple $(f,g)$, where $f \in \mathsf{Top}(X,Y)$ and $g \in \mathsf{Top}(A,B)$, such that 
	\begin{equation*}
		\begin{tikzcd}
			A \arrow[r,hook]\arrow[d,"g"'] & X \arrow[d,"f"]\\
			B \arrow[r,hook] & Y
		\end{tikzcd}
	\end{equation*}
	\noindent commutes.
\end{example}

\begin{example}[$\mathsf{Top}_\ast$]
	Define the objects of $\mathsf{Top}_\ast$ to be the class of all tuple $(X,p)$, where $X$ is a topological space and $p \in X$. Moreover, given objects $(X,p)$ and $(Y,q)$ in $\mathsf{Top}_\ast$, define $\mathsf{Top}_\ast\del[1]{(X,p),(Y,q)} := \cbr[0]{f \in \mathsf{Top}(X,Y) : f(p) = q}$. It is easy to check that $\mathsf{Top}_\ast$ is a category, called the \bld{category of pointed topological spaces}.

\end{example}

\subsection*{Limits}

\begin{definition}[Diagram]
	Let $\mathcal{C}$ be a category and $\mathsf{A}$ a small category. A functor $\mathsf{A} \to \mathcal{C}$ is called a \bld{diagram in $\mathcal{C}$ of shape $\mathsf{A}$}.
\end{definition}

\begin{definition}[Cone and Limit]
	Let $\mathcal{C}$ be a category and $D : \mathsf{A} \to \mathcal{C}$ a diagram in $\mathcal{C}$ of shape $\mathsf{A}$. A \bld{cone on $D$} is a tuple $\del[1]{C,(f_\alpha)_{\alpha \in \mathsf{A}}}$, where $C \in \mathcal{C}$ is an object, called the \bld{vertex} of the cone, and a family of arrows in $\mathcal{C}$
	\begin{equation}
		\del[1]{C \xrightarrow{f_\alpha} D(\alpha)}_{\alpha \in \mathsf{A}}.
	\end{equation}
	\noindent such that for all morphisms $f \in \cat{A}$, $f : \alpha \to \beta$, the triangle
	\begin{equation*}
		\xymatrix{
			& D(\alpha) \ar[dd]^{D(f)}\\
			C \ar[ur]^{f_\alpha}\ar[dr]_{f_\beta}\\
			& D(\beta)}
	\end{equation*}
	\noindent commutes. A \bld{(small) limit of $D$} is a cone $\del[1]{L, (\pi_\alpha)_{\alpha \in \mathsf{A}}}$ with the property that for any other cone $\del[1]{C,(f_\alpha)_{\alpha \in \mathsf{A}}}$ there exists a unique morphism $\wbar{f} : C \to L$ such that $\pi_\alpha \circ \wbar{f} = f_\alpha$ holds for every $\alpha \in \mathsf{A}$.
	\label{def:limit}
\end{definition}

\begin{remark}
	In the setting of definition \ref{def:limit}, if $\del[1]{L, (\pi_\alpha)_{\alpha \in \cat{A}}}$ is a limit of $D$, we sometimes reffering to $L$ only as the limit of $D$ and we write
	\begin{equation}
		L = \lim_{\leftarrow \mathsf{A}} D.
	\end{equation}
\end{remark}

\subsection*{Filtered Colimits}
\begin{definition}[Filtered Category]
	A category $\mathcal{J}$ is \bld{filtered}, if $\mathcal{J}$ is not empty and
	\begin{enumerate}[label = \textup{(}\alph*\textup{)},wide = 0pt]
		\item To any two objects $j$ and $j'$ in $\mathcal{J}$ there exists $k \in \ob(\mathcal{J})$ and morphisms $j \to k$ and $j' \to k$.
		\item To any two parallel arrows $u,v : i \to j$ in $\mathcal{J}$, there exists $k \in \ob(\mathcal{J})$ and a morphism $w : j \to k$ in $\mathcal{J}$, such that $w \circ u = w \circ v$.
	\end{enumerate}
\end{definition}

\begin{definition}[Filtered Colimit]
	Let $\mathcal{C}$ be a category. A \bld{filtered diagram in $\mathcal{C}$ of shape $\mathcal{J}$} is a diagram in $\mathcal{C}$ of shape $\mathcal{J}$, where $\mathcal{J}$ is a small filtered category. A \bld{filtered colimit of $D$} is a colimit of a filtered diagram $D$ in $\mathcal{C}$, written $\colim D$.
\end{definition}

\begin{proposition}
	\label{prop:filtered_colimits_set}
	In $\mathsf{Set}$, all filtered limits exist.
\end{proposition}

\begin{proof}
	Let $D : \mathcal{J} \to \mathsf{Set}$ be a filtered diagram. Define
	\begin{equation*}
		X := \coprod_{j \in \ob(\mathcal{J})} D(j),
	\end{equation*}
	\noindent and define a relation $\sim$ on $X$ by
	\begin{equation*}
		x \in D(j)\sim y \in D(j') \> :\Leftrightarrow \> \exists f : j \to k,g : j' \to k \text{ such that } D(f)(x) = D(g)(y).
	\end{equation*} 
	Then it is easy to check that $\sim$ is an equivalence relation and that $\colim D \cong X/{\sim}$.
\end{proof}

\begin{proposition}
	In $\mathsf{Top}$, all filtered colimits exist.
\end{proposition}

\begin{proof}
	Let $D : \mathcal{J} \to \mathsf{Top}$ be a filtered diagram. If $U : \mathsf{Top} \to \mathsf{Set}$ denotes the forgetful functor, $U \circ D$ is a filtered diagram in $\mathsf{Set}$. Hence using proposition \ref{prop:filtered_colimits_set}, we know that $\colim (U \circ D)$ exists. Hence we get a limiting cocone $\del[1]{\colim (U \circ D),(q_j)_{j \in \ob(\mathcal{J})}}$ in $\mathsf{Set}$. Define a topology $\colim (U \circ D)$ by letting $U \subseteq \colim(U \circ D)$ to be open if and only if $q^{-1}_j(U)$ open in $U(D(j))$ for all $j \in \ob(\mathcal{J})$. Then it is easy to check that we have a limiting cocone in $\mathsf{Top}$.
\end{proof}

\begin{definition}[Preorder]
	Let $P$ be a set. A \bld{preorder on $P$} is a reflexive and transitive binary relation $\preccurlyeq$ on $P$. A set equipped with a preorder is called a \bld{preordered set}.
\end{definition}

\begin{definition}[Directed Preorder]
	A preorder $\preccurlyeq$ on a set $P$ is said to be \bld{directed} if for every two elements $p,q \in P$there exists $r \in P$, such that $p \preccurlyeq r$ and $q \preccurlyeq r$ holds. A set equipped with a directed preorder is called a \bld{directed set}. 
\end{definition}

\begin{lemma}
	\label{lem:directed_preorder_filtered_category}
	Let $(P,\preccurlyeq)$ be a directed set. Define 
	\begin{itemize}[wide = 0pt]
		\item $\ob(\mathcal{J}(P,\preccurlyeq)) := P$.
		\item For $p,q \in P$, if $p \not\preccurlyeq q$ let $\mathcal{J}(P,\preccurlyeq)(p,q) := \varnothing$ and if $p \preccurlyeq q$ let $\mathcal{J}(P,\preccurlyeq)(p,q)$ be the unique arrow $p \to q$.
		\item For $p,q,r \in P$, let $q \to r \circ p \to q := p \to r$.
	\end{itemize}
	Then $\mathcal{J}(P,\preccurlyeq)$ is a filtered category.
\end{lemma}

\begin{exercise}
	Prove lemma \ref{lem:directed_preorder_filtered_category}.
\end{exercise}

\begin{definition}[Sequential Colimit]
	A \bld{sequential colimit of $D$} is a colimit of a diagram $D$ of shape $(\omega,\leq)$. We will write $\colim_n D(n)$ for a sequential colimit of $D$.
\end{definition}

\section*{Basic Algebra}
\subsection*{The Isomorphism Theorems}

\section*{Basic Point-Set Topology}
\subsection*{The Lebesgue Number Lemma}

\begin{definition}[Lebesgue Number]
	Let $(M,d)$ be a metric space with an open cover $(U_\alpha)_{\alpha \in A}$. A number $\delta > 0$ is called a \bld{Lebesgue number} for the cover, if every subset of $M$ whose diameter is less than $\delta$ is contained in $U_\alpha$ for some $\alpha \in A$.
\end{definition}

\begin{lemma}[Lebesgue Number Lemma]
	\label{lem:Lebesgue_number_lemma}
	Every open cover of a compact metric space admits a Lebesgue number.	
\end{lemma}

\subsection*{The Closed Map Lemma}

\begin{lemma}[Closed Map Lemma]
	Let $X,Y \in \ob(\mathsf{Top})$ such that $X$ is compact and $Y$ is Hausdorff, and $f \in \mathsf{Top}(X,Y)$. Then:
	\begin{enumerate}[label = \textup{(}\alph*\textup{)},wide = 0pt]
		\item $f$ is a closed map.
		\item If $f$ is injective, it is a topological embedding.
		\item If $f$ is surjective, it is a quotient map.
		\item If $f$ is bijective, it is a homeomorphism.
	\end{enumerate}
	\label{lem:closed_map_lemma}
\end{lemma}

\section*{Homological Algebra}
\subsection*{Diagram Lemmas}
\begin{proposition}[Snake Lemma]
	Suppose we are given a commutative diagram in $\mathsf{AbGrp}$ with exact rows:
	\begin{equation*}
		\begin{tikzcd}
			& A \arrow[r,"i"]\arrow[d,"f"] & B \arrow[r,"j"]\arrow[d,"g"] & C \arrow[r]\arrow[d,"h"] & 0\\
			0 \arrow[r] & A' \arrow[r,"i'"'] & B' \arrow[r,"j'"'] & C'
		\end{tikzcd}
	\end{equation*}
	Then there exists $\delta \in \mathsf{AbGrp}(\ker h,\coker f)$ such that the sequence
	\begin{equation}
		\begin{tikzcd}
			 \ker f \arrow[r] & \ker g \arrow[r] & \ker h \arrow[r,"\delta"] & \coker f \arrow[r] & \coker g \arrow[r] & \coker h
		 \end{tikzcd}
	\end{equation}
	\noindent is exact.
	\label{prop:snake_lemma}
\end{proposition}

\begin{proof}
	Consider the augmented diagram in figure \ref{fig:snake_lemma}, where the morphisms $k$, $l$, $p$ and $q$ are induced by $i$, $j$, $i'$ and $j'$, respectively.
	\begin{figure}[h!tb]
		\centering
		\begin{tikzcd}
			& 0 \arrow[d] & 0 \arrow[d] & 0 \arrow[d]\\
			& \ker f \arrow[d]\arrow[r,"k"] & \ker g \arrow[d]\arrow[r,"l"] & \ker h \arrow[d]\\
			& A \arrow[r,"i"]\arrow[d,"f"] & B \arrow[r,"j"]\arrow[d,"g"] & C \arrow[r]\arrow[d,"h"] & 0\\
			0 \arrow[r] & A' \arrow[r,"i'"']\arrow[d] & B' \arrow[r,"j'"']\arrow[d] & C' \arrow[d]\\
			& \coker f \arrow[d]\arrow[r,"p"'] & \coker g \arrow[d]\arrow[r,"q"'] & \coker h \arrow[d]\\
			& 0 & 0 & 0
		\end{tikzcd}		
		\caption{Proof of the snake lemma.}
		\label{fig:snake_lemma}
	\end{figure}

	\begin{enumerate}[label = \textit{Step \arabic*:},wide = 0pt]
		\item \textit{Exactness at $\ker g$.} Let $a \in \ker f$. Then $l(k(a)) = j(i(a)) = 0$ by exactness at $B$ and thus $\im k \subseteq \ker l$. Conversly, let $b \in \ker l$. Then $j(b) = 0$ and by exactness at $B$, there exists $a \in A$ such that $i(a) = b$. Moreover $0 = g(b) = g(i(a)) = i'(f(a))$ since $b \in \ker g$ and thus $f(a) = 0$ by injectivity of $i'$. Hence $\ker j \subseteq \im k$.
		\item \textit{Exactness at $\coker g$.} Let $a' + \im f \in \coker f$. Then
			\begin{equation*}
				q(p(a' + \im f)) = j'(i'(a')) + \im h = \im h
			\end{equation*}
			\noindent by exactness at $B'$ implies $\im p \subseteq \ker q$. Conversly, let $b' + \im g \in \ker q$. Then 
			\begin{equation*}
				0 = q(b' + \im g) = j'(b') + \im h
			\end{equation*}
			\noindent and thus $j'(b') \in \im h$. Hence there exists $c \in C$, such that $j'(b') = h(c)$. Since $j$ is surjective, we find $b \in B$ such that $j(b) = c$. Therefore $j'(b') = h(j(b))$. By commutativity we get $j'(b') = j'(g(b))$ which is equivalent to $j'(b' - g(b)) = 0$. Thus $b' - g(b) \in \ker j'$ and exactness at $B'$ yields the existence of $a' \in A'$ such that $i'(a') = b' - g(b)$. Now
			\begin{equation*}
				p(a' + \im f) = i'(a') + \im g = b' - g(b) + \im g = b' + \im g
			\end{equation*}
			\noindent and thus $\ker q \subseteq \im p$.
		\item \textit{Definition of $\delta$.} Consider the snakelike path indicated in figure \ref{fig:snakelike_path}. Let $c \in \ker h$. Since $j$ is surjective, we find $b \in B$ such that $j(b) = c$. Since $c \in \ker h$, we get that $0 = h(c) = h(j(b)) = j'(g(b))$ and thus $g(b) \in \ker j'$ which implies $g(b) \in \im i'$ by exactness at $B'$. Hence there exists $a' \in A'$ such that $i'(a') = g(b)$. Actually this $a'$ is unique since $i'$ is injective. Define $\delta : \ker h \to \coker f$ by
			\begin{equation*}
				\delta(c) := a' + \im f.
			\end{equation*}
			\begin{figure}[h!tb]
				\centering
				\begin{subfigure}[b]{.5\textwidth}
					\begin{tikzcd}
						\ker f \arrow[r,"k"] & \ker g \arrow[r,"l"] & \ker h \arrow[d]\\
						& B \arrow[d,"g"]& C \arrow[l]\\
						A' \arrow[d] & B' \arrow[l]\\
						\coker f \arrow[r,"p"'] & \coker g \arrow[r,"q"'] & \coker h
					\end{tikzcd}		
					\caption{Snakelike path.}
					\label{fig:snakelike_path}
				\end{subfigure}
				~
				\begin{subfigure}[b]{.5\textwidth}
					\begin{tikzcd}
						& & c \arrow[d,mapsto]\\
						& b \arrow[d,mapsto]& c \arrow[l,mapsto]\\
						a' \arrow[d,mapsto] & g(b) \arrow[l,mapsto]\\
						a' + \im f
					\end{tikzcd}		
					\caption{Definition of $\delta$.}
					\label{fig:definition_delta}					
				\end{subfigure}
				\caption{}
			\end{figure}
		\item \textit{Checking that $\delta$ is a morphism of groups.} Since $j$ is only surjective, we have to show that $\delta$ is a function. So suppose we choose $b_0 \in B$ instead of $b \in B$ in figure \ref{fig:definition_delta} with $b_0 \neq b$. We want to show that $\delta(c) = a' + \im f = a_0' + \im f$, or equivalently $a' - a_0' \in \im f$. Since $c = j(b) = j(b_0)$, we have that $b - b_0 \in \ker j$. Hence by exactness at $B$ there exists $a \in A$ such that $i(a) = b - b_0$. Applying $g$ and invoking commutativity yields
			\begin{equation*}
				g(b) - g(b_0) = g(i(a)) = i'(f(a))
			\end{equation*}
			Hence $i'(a') - i'(a_0')= i'(f(a))$ and thus the injectivity of $i'$ yields $a' - a_0' = f(a)$. In the same manner one can show that $\delta$ is a morphism of groups.
		\item \textit{Exactness at $\ker h$.} Let $b \in \ker g$. Then $\im l \subseteq \ker \delta$ immeddiately follows from figure \ref{fig:im_l_subseteq_ker_delta}. Conversly, suppose $c \in \ker \delta$. From figure \ref{fig:ker_delta_subseteq_im_l} we get that
			\begin{equation*}
				g(b) = i'(a') = i'(f(a)) = g(i(a))
			\end{equation*}
			\noindent and thus $b - i(a) \in \ker g$. So $l(b - i(a)) = j(b) - j(i(a)) = j(b) = c$ by exactness at $B$ and thus $\ker \delta \subseteq \im l$.
			\begin{figure}[h!tb]
				\centering
				\begin{subfigure}[b]{.5\textwidth}
					\begin{tikzcd}
						& & j(b) \arrow[d,mapsto]\\
						& b \arrow[d,mapsto]& j(b) \arrow[l,mapsto]\\
						0 \arrow[d,mapsto] & g(b) = 0 \arrow[l,mapsto]\\
						\im f\\
					\end{tikzcd}		
					\caption{$\im l \subseteq \ker \delta$.}
					\label{fig:im_l_subseteq_ker_delta}
				\end{subfigure}
				~
				\begin{subfigure}[b]{.5\textwidth}
					\begin{tikzcd}
						& & c \arrow[d,mapsto]\\
						a \arrow[d,mapsto] & b \arrow[d,mapsto]& c \arrow[l,mapsto]\\
						a' = f(a) \arrow[d,mapsto] & g(b) \arrow[l,mapsto]\\
						a' + \im f\\
					\end{tikzcd}		
					\caption{$\ker \delta \subseteq \im l$.}
					\label{fig:ker_delta_subseteq_im_l}					
				\end{subfigure}
				\caption{}
			\end{figure}

		\item \textit{Exactness at $\coker f$.} Suppose that $a' + \im f \in \im \delta$. Then
			\begin{equation*}
				p(a' + \im f) = i'(a') + \im g = g(b) + \im g = \im g 
			\end{equation*}
			\noindent and thus $\im \delta \subseteq \ker p$. Conversly, suppose that $a' + \im f \in \ker p$. Hence $i'(a') \in \im g$ and we find $b \in B$ such that $g(b) = i'(a')$. Consider $j(b)$. By exactness at $B'$ follows
			\begin{equation*}
				h(j(b)) = j'(g(b)) = j'(i'(a')) = 0
			\end{equation*}
			So $j(b) \in \ker h$. Moreover, by construction $\delta(j(b)) = a' + \im f$ and thus $\ker p \subseteq \im \delta$.
	\end{enumerate}
\end{proof}

\begin{proposition}[Five Lemma]
	Suppose we are given a commutative diagram in $\mathsf{AbGrp}$ with exact rows and columns:
	\begin{equation*}
		\begin{tikzcd}
			& 0\arrow[d] & & 0\arrow[d] & 0\arrow[d]\\
			A \arrow[d,"f"]\arrow[r,"\varphi_1"] & B \arrow[r,"\varphi_2"]\arrow[d,"g"] & C \arrow[r,"\varphi_3"]\arrow[d,"h"] & D \arrow[r,"\varphi_4"]\arrow[d,"k"] & E \arrow[d,"l"]\\
			A' \arrow[r,"\psi_1"']\arrow[d] & B' \arrow[r,"\psi_2"']\arrow[d] & C' \arrow[r,"\psi_3"'] & D' \arrow[r,"\psi_4"']\arrow[d] & E'\\
			0 & 0 & & 0
		\end{tikzcd}
	\end{equation*}
	Then $h$ is an isomorphism.
\end{proposition}

\begin{proof} 
	We show that $h$ is bijective.
	\begin{enumerate}[label = \textit{Step \arabic*:}, wide = 0pt]
		\item \textit{$h$ is injective.} See figure \ref{fig:h_injective}. Let $c \in \ker h$. Hence $h(c) = 0$ and since $\varphi_3$ is a morphism of groups, we have that $(\varphi_3 \circ h)(c) = 0$. By commutativity, $(k \circ \varphi_3)(c) = 0$ and thus since $k$ is injective, $\varphi_3(c) = 0$. Exactness at $C$ implies that there exists some $b \in B$ such that $\varphi_2(b) = c$. Moreover, by commutativity $\psi_2(g(b)) = 0$ and thus we find $a' \in A'$ such that $\psi_1(a') = g(b)$. Sujectivity of $f$ implies the existence of $a \in A$ such that $f(a) = a'$. Commutativity yields $g(b) = g(\varphi_1(a))$ and thus $b - \varphi_1(a) \in \ker g$. Since $g$ is injective, $b = \varphi_1(a)$ and thus $c = \varphi_2(\varphi_1(a)) = 0$.  
			\begin{figure}[h!tb]
				\centering
				\begin{tikzcd}
					a \arrow[d,mapsto]\arrow[r,mapsto] & b = \varphi_1(a) \arrow[r,mapsto]\arrow[d,mapsto] & c \arrow[r,mapsto]\arrow[d,mapsto] & 0 \arrow[d,mapsto]\\
					a' \arrow[r,mapsto] & g(b) \arrow[r,mapsto] & h(c) = 0 \arrow[r,mapsto] & 0
				\end{tikzcd}
				\caption{Proof of injectivity of $h$.}
				\label{fig:h_injective}
			\end{figure}

		\item \textit{$h$ is surjective.} See figure \ref{fig:h_surjective}. Let $c' \in C'$. Since $k$ is surjective, we find $d \in D$ such that $k(d) = \psi_3(c')$. Hence exactness at $D'$ together with commutativity yields $(l \circ \varphi_4)(d) = 0$. Since $l$ is injective, we get that $\varphi_4(d) = 0$. Thus by exactness at $D$ we find $c \in C$ such that $\varphi_3(c) = d$. Hence by commutativity, $(\psi_3 \circ h)(c) = \psi_3(c')$ or equivalently, $c' - h(c) \in \ker \psi_3$. By exactness at $C'$ we find $b' \in B'$ such that $\psi_2(b') = c' - h(c)$. Moreover, since $g$ is surjective, we find $b \in B$ such that $g(b) = b'$. Finally, commutativity yields $(h \circ \varphi_2)(b) = c' - h(c)$ or equivalently $c' = h(c + \varphi_2(b))$.
			\begin{figure}[h!tb]
				\centering
				\begin{tikzcd}
					c \arrow[r,mapsto] & d \arrow[r,mapsto]\arrow[d,mapsto] & 0 \arrow[d,mapsto]\\
					c' \arrow[r,mapsto] & \psi_3(c') \arrow[r,mapsto] & 0
				\end{tikzcd}
				\caption{Proof of surjectivity of $h$.}
				\label{fig:h_surjective}
			\end{figure}
	\end{enumerate}
\end{proof}
