\chapter{Foundations}
\section*{Basic Category Theory}
\subsection*{Categories}

We use the first order theory of Neumann-Bernays-G\"odel (BNG) as described in \cite[231]{mendelson:logic:2015}.

\begin{definition}[Category]
	A \bld{category $\mathcal{C}$} consists of 
	\begin{itemize}[leftmargin = *]
		\item A class $\ob(\mathcal{C})$, called the \bld{objects of $\mathcal{C}$}.
		\item A class $\mor(\mathcal{C})$, called the \bld{morphisms of $\mathcal{C}$}.
		\item Two functions $\dom : \mor(\mathcal{C}) \to \ob(\mathcal{C})$ and $\cod : \mor(\mathcal{C}) \to \ob(\mathcal{C})$, which assign to each morphism $f$ in $\mathcal{C}$ its \bld{domain} and \bld{codomain}, respectively.
		\item For each $X \in \ob(\mathcal{C})$ a function $\ob(\mathcal{C}) \to \mor(\mathcal{C})$ which assigns a morphism $\id_X$ such that $\dom \id_X = \cod \id_X = X$.
		\item A function 
			\begin{equation}
				\circ : \cbr{(g,f) \in \mor(\mathcal{C}) \times \mor(\mathcal{C}) : \dom g = \cod f} \to \mor(\mathcal{C}) 
			\end{equation}
			mapping $(g,f)$ to $g \circ f$, called \bld{composition}, such that $\dom(g \circ f) = \dom f$ and $\cod(g \circ f) = \cod g$.
	\end{itemize}
	Subject to the following axioms:
	\begin{itemize}[leftmargin = *]
		\item \bld{(Associativity Axiom)} For all $f,g,h \in \mor(\mathcal{C})$ with $\dom h = \cod g$ and $\dom g = \cod f$, we have that
			\begin{equation}
				(h \circ g) \circ f = h \circ (g \circ f).
			\end{equation}
		\item \bld{(Unit Axiom)} For all $f \in \mor(\mathcal{C})$ with $\dom f = X$ and $\cod f = Y$ we have that
			\begin{equation}
				f = f \circ \id_X = \id_Y \circ f.
			\end{equation}
	\end{itemize}
\end{definition}

\begin{remark}
	Let $\mathcal{C}$ be a category. For $X,Y \in \ob(\mathcal{C})$ we will abreviate
	\begin{equation*}
		\mathcal{C}(X,Y) := \cbr{f \in \mor(\mathcal{C}) : \dom f = X \text{ and } \cod f = Y}.
	\end{equation*}
	Moreover, $f \in \mathcal{C}(X,Y)$ is depicted as
	\begin{equation}
		f : X \to Y.
	\end{equation}
\end{remark}

\begin{example}
	Let $\ast$ be a single, not nearer specified object. Consider as morphisms the class of all cardinal numbers and as composition cardinal addition. By \cite[112--113]{halbeisen:set_theory:2012}, cardinal addition is associative and $\varnothing$ serves for the identity $\id_\ast$.  
\end{example}

\begin{definition}[Locally Small, Hom-Set]
	A category $\mathcal{C}$ is said to be \bld{locally small} if for all $X,Y \in \mathcal{C}$, $\mathcal{C}(X,Y)$ is a set. If $\mathcal{C}$ is locally small, $\mathcal{C}(X,Y)$ is called a \bld{hom-set} for all $X,Y \in \mathcal{C}$. 
\end{definition}

\subsection*{Functors}

\begin{definition}[Functor]
	Let $\mathcal{C}$ and $\mathcal{D}$ be categories. A \bld{functor $F : \mathcal{C} \to \mathcal{D}$} is a pair of functions $(F_1,F_2)$, $F_1 : \ob(\mathcal{C}) \to \ob(\mathcal{D})$, called the \bld{object function} and $F_2 : \mor(\mathcal{C}) \to \mor(\mathcal{D})$, called the \bld{morphism function}, such that for every morphism $f : X \to Y$ we have that $F_2(f) : F_1(X) \to F_1(Y)$ and $(F_1,F_2)$ is subject to the following \bld{compatibility conditions}:
	\begin{itemize}[leftmargin = *]
		\item For all $X \in \ob(\mathcal{C})$, $F_2(\id_X) = \id_{F_1(X)}$.
		\item For all $f \in \mathcal{C}(X,Y)$ and $g \in \mathcal{C}(Y,Z)$ we have that $F_2(g \circ f) = F_2(g) \circ F_2(f)$.
	\end{itemize}
\end{definition}

\begin{remark}
	Let $F : \mathcal{C} \to \mathcal{D}$ be a functor. It is convenient to denote the components $F_1$ and $F_2$ also with $F$.
\end{remark}

\subsection*{Subcategories}
\begin{definition}[Subcategory]
	Let $\mathcal{C}$ be a category. A \bld{subcategory $\mathcal{\altS}$ of $\mathcal{C}$} consists of
	\begin{itemize}[leftmargin = *]
		\item A subclass $\ob(\mathcal{\altS}) \subseteq \ob(\mathcal{C})$.
		\item A subclass $\mor(\mathcal{\altS}) \subseteq \mor(\mathcal{C})$.
	\end{itemize}
	Subject to the following conditions:
	\begin{itemize}[leftmargin = *]
		\item For all $X \in \mathcal{\altS}$, $\id_S \in \mor(\mathcal{\altS})$.
	\end{itemize}
\end{definition}

\begin{example}[$\mathsf{Top}_\ast$]
	Define the objects of $\mathsf{Top}_\ast$ to be the class of all tuple $(X,p)$, where $X$ is a topological space and $p \in X$. Moreover, given objects $(X,p)$ and $(Y,q)$ in $\mathsf{Top}_\ast$, define $\mathsf{Top}_\ast\del[1]{(X,p),(Y,q)} := \cbr[0]{f \in \mathsf{Top}(X,Y) : f(p) = q}$. It is easy to check that $\mathsf{Top}_\ast$ is a category, called the \bld{category of pointed topological spaces}.

\end{example}

\subsection*{Limits}

\begin{definition}[Diagram]
	Let $\mathcal{C}$ be a category and $\mathsf{A}$ a small category. A functor $\mathsf{A} \to \mathcal{C}$ is called a \bld{diagram in $\mathcal{C}$ of shape $\mathsf{A}$}.
\end{definition}

\begin{definition}[Cone and Limit]
	Let $\mathcal{C}$ be a category and $D : \mathsf{A} \to \mathcal{C}$ a diagram in $\mathcal{C}$ of shape $\mathsf{A}$. A \bld{cone on $D$} is a tuple $\del[1]{C,(f_\alpha)_{\alpha \in \mathsf{A}}}$, where $C \in \mathcal{C}$ is an object, called the \bld{vertex} of the cone, and a family of arrows in $\mathcal{C}$
	\begin{equation}
		\del[1]{C \xrightarrow{f_\alpha} D(\alpha)}_{\alpha \in \mathsf{A}}.
	\end{equation}
	\noindent such that for all morphisms $f \in \cat{A}$, $f : \alpha \to \beta$, the triangle
	\begin{equation*}
		\xymatrix{
			& D(\alpha) \ar[dd]^{D(f)}\\
			C \ar[ur]^{f_\alpha}\ar[dr]_{f_\beta}\\
			& D(\beta)}
	\end{equation*}
	\noindent commutes. A \bld{(small) limit of $D$} is a cone $\del[1]{L, (\pi_\alpha)_{\alpha \in \mathsf{A}}}$ with the property that for any other cone $\del[1]{C,(f_\alpha)_{\alpha \in \mathsf{A}}}$ there exists a unique morphism $\wbar{f} : A \to L$ such that $\pi_\alpha \circ \wbar{f} = f_\alpha$ holds for every $\alpha \in \mathsf{A}$.
	\label{def:limit}
\end{definition}

\begin{remark}
	In the setting of definition \ref{def:limit}, if $\del[1]{L, (\pi_\alpha)_{\alpha \in \cat{A}}}$ is a limit of $D$, we sometimes reffering to $L$ only as the limit of $D$ and we write
	\begin{equation}
		L = \lim_{\leftarrow \mathsf{A}} D.
	\end{equation}
\end{remark}
