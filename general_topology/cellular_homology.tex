\chapter{Cellular Homology}
\section*{Cell Complexes}
\subsection*{Adjunction Spaces}

\begin{definition}[Adjunction Space]
	Let $X$ and $Y$ be topologial spaces and let $A \subseteq X$ be a closed subspace. Moreover, let $f \in \mathsf{Top}(A,Y)$. Define the \bld{adjunction space of $X$ and $Y$ along $f$}, written $X \cup_f Y$, to be
	\begin{equation*}
		X \cup_f Y := \textstyle\del[1]{X \coprod Y}/{\sim},
	\end{equation*}
	\noindent where $\sim$ is the smallest equivalence relation on $X \coprod Y$ generated by $a {\sim} f(a)$, for $a \in A$.
\end{definition}

\begin{lemma}
	Let $X$ and $Y$ be topological spaces, $A \subseteq X$ a closed subspace and $f \in \mathsf{Top}(A,Y)$. Then:
	\begin{enumerate}[label = \textup{(}\alph*\textup{)}]
		\item $X \cup_f Y$ with obvious inclusions is the pushout of the diagram
			\begin{equation*}
				\begin{tikzcd}
					A \arrow[d,hook,"\iota"']\arrow[r,"f"] & Y\\
					X
				\end{tikzcd}
			\end{equation*}
			\noindent in $\mathsf{Top}$.
		\item The inclusion $q \circ \iota_Y : Y \to X \cup_f Y$ is a closed embedding.
		\item $q \circ \iota_X\vert_{X \setminus A}$ is an open embedding.
		\item $X \cup_f Y$ is the disjoint union of $(q \circ \iota_X)(X \setminus A)$ and $(q \circ \iota_Y)(Y)$.
	\end{enumerate}
\end{lemma}

\begin{proof}
	To prove (a), simply use that $X \coprod Y$ is a coproduct in $\mathsf{Top}$. Indeed, if we have another cocone for the diagram, we have also a cocone for the coproduct diagram of $X$ and $Y$. Hence there exists a unique continuous map from $X \coprod Y$ to the other vertex, and it is easy to check that this map passes to the quotient.\\
	To prove (b), observe that $q \circ \iota_Y$ with restricted codomain has an obvious inverse defined by $\sbr[0]{y} \mapsto y$. This is well defined since if $y {\sim}y'$, we must have $y = y'$ by definition of the equivalence relation generated by $a{\sim}f(a)$. Let $B \subseteq Y$ closed. Then $q^{-1}\del[1]{q(\iota_Y(B))} = f^{-1}(B) \coprod B$, and thus since $f^{-1}(B)$ is closed in $A$ and $A$ is closed in $X$, $f^{-1}(B)$ is closed in $X$. Hence $f^{-1}(B) \coprod B$ is closed in $X \coprod Y$ by definition of the disjoint union space topology. From this also follows that $q(\iota_Y(Y))$ is closed in $X \cup_f Y$.\\
	Note that since $A$ is closed in $X$, $X \setminus A$ is open in $X$. Similar to part (b), we see that an inverse is given by $\sbr{x} \mapsto x$. Let $U \subseteq X \setminus A$ be open. Then $q^{-1}\del[1]{q(\iota_X(U))} = U$, which is open in $X \setminus A$ and hence in $X$.\\
	
\end{proof}

\begin{definition}[Deformation Retract]
	Let $X$ be a topologial space and $A \subseteq X$ a subspace. We say that \bld{$A$ is a deformation retract of $X$}, if there exists a retract $r$ of $\iota : A \hookrightarrow X$ in $\mathsf{Top}$, such that $\iota \circ r \simeq \id_X$.
\end{definition}

\begin{definition}[Cells]
	Let $n \in \omega$, $n \geq 1$. Then $\mathbb{E}^n := \mathbb{B}^n \setminus \mathbb{S}^{n-1}$ is called the \bld{standard $n$-cell}. If $X$ is a topological space and $E \subseteq X$ is homeomorphic to $\mathbb{E}^n$, then $E$ is called an \bld{$n$-cell in $X$}. Moreover, if $f \in \mathsf{Top}(\mathbb{S}^{n-1},Y)$, the adjunction space $\mathbb{B}^n \cup_f Y$ is said to be obtained from $Y$ by \bld{attaching an $n$-cell}. 
\end{definition}

\begin{proposition}
	Let $Y$ be a Hausdorff space, $n \in \omega$, $n > 0$, and $f \in \mathsf{Top}(\mathbb{S}^{n - 1},Y)$. Then if $\iota : Y \hookrightarrow  \mathbb{B}^n \cup_f Y$ denotes inclusion, there is a long exact sequence
	\begin{equation*}
		\begin{tikzcd}
			\dots H_k(\mathbb{S}^{n - 1}) \arrow[r,"H_k(f)"] & H_k(Y) \arrow[r,"H_k(\iota)"] & H_k(\mathbb{B}^n \cup_f Y) \arrow[r] & H_{k - 1}(\mathbb{S}^{n - 1}) \dots
		\end{tikzcd}
	\end{equation*}
\end{proposition}
