%%%%%%%%%%%%%%%%%%%%%%%%%%%%%%%%%%%%%%%%%%%%%%%%%%%%%%%%%%%%%%%%%%%%%%%%%%
%Author:																 %
%-------																 %
%Yannis Baehni at University of Zurich									 %
%baehni.yannis@uzh.ch													 %
%																		 %
%Version log:															 %
%------------															 %
%06/02/16 . Basic structure												 %
%04/08/16 . Layout changes including section, contents, abstract.		 %
%%%%%%%%%%%%%%%%%%%%%%%%%%%%%%%%%%%%%%%%%%%%%%%%%%%%%%%%%%%%%%%%%%%%%%%%%%

%Page Setup
\documentclass[
	12pt, 
	oneside, 
	a4paper,
	reqno,
	final
]{amsart}

\usepackage[
	left = 3cm, 
	right = 3cm, 
	top = 3cm, 
	bottom = 3cm
]{geometry}

%Headers and footers
\usepackage{fancyhdr}
	\pagestyle{fancy}
	%Clear fields
	\fancyhf{}
	%Header right
	\fancyhead[R]{
		\footnotesize
		Yannis B\"{a}hni\\
		\href{mailto:yannis.baehni@uzh.ch}{yannis.baehni@uzh.ch}
	}
	%Header left
	\fancyhead[L]{
		\footnotesize
		401-3581-67L Symplectic Geometry\\
		Autumn 2017
	}
	%Page numbering in footer
	\fancyfoot[C]{\thepage}
	%Separation line header and footer
	\renewcommand{\headrulewidth}{0.4pt}
	%\renewcommand{\footrulewidth}{0.4pt}
	
	\setlength{\headheight}{19pt} 

%Title
\usepackage[foot]{amsaddr}
\usepackage{upref}
\usepackage{newtxtext}
\usepackage[subscriptcorrection,nofontinfo,mtpcal,mtphrb]{mtpro2}
\usepackage{bm}
\usepackage{xspace}
\makeatletter
\usepackage{etoolbox}
\patchcmd{\abstract}{\scshape\abstractname}{\textbf{\abstractname}}{}{}

\usepackage[all,cmtip]{xy}

%Section, subsection and subsubsection font
%------------------------------------------
\makeatletter
	\renewcommand{\@secnumfont}{\bfseries}
	\renewcommand\section{\@startsection{section}{1}%
  	\z@{.7\linespacing\@plus\linespacing}{.5\linespacing}%
  	{\normalfont\bfseries\centering}}
	\renewcommand\subsection{\@startsection{subsection}{2}%
    	\z@{.5\linespacing\@plus.7\linespacing}{-.5em}%
    	{\normalfont\bfseries}}%
	\renewcommand\subsubsection{\@startsection{subsubsection}{3}%
    	\z@{.5\linespacing\@plus.7\linespacing}{-.5em}%
    	{\normalfont\bfseries}}%
%Formatting title of TOC
\renewcommand{\contentsnamefont}{\bfseries}
%Table of Contents
\setcounter{tocdepth}{3}

% Add bold to \section titles in ToC and remove . after numbers
\renewcommand{\tocsection}[3]{%
  \indentlabel{\@ifnotempty{#2}{\bfseries\ignorespaces#1 #2\quad}}\bfseries#3}
% Remove . after numbers in \subsection
\renewcommand{\tocsubsection}[3]{%
  \indentlabel{\@ifnotempty{#2}{\ignorespaces#1 #2\quad}}#3}
\let\tocsubsubsection\tocsubsection% Update for \subsubsection
%...

\newcommand\@dotsep{4.5}
\def\@tocline#1#2#3#4#5#6#7{\relax
  \ifnum #1>\c@tocdepth % then omit
  \else
    \par \addpenalty\@secpenalty\addvspace{#2}%
    \begingroup \hyphenpenalty\@M
    \@ifempty{#4}{%
      \@tempdima\csname r@tocindent\number#1\endcsname\relax
    }{%
      \@tempdima#4\relax
    }%
    \parindent\z@ \leftskip#3\relax \advance\leftskip\@tempdima\relax
    \rightskip\@pnumwidth plus1em \parfillskip-\@pnumwidth
    #5\leavevmode\hskip-\@tempdima{#6}\nobreak
    \leaders\hbox{$\m@th\mkern \@dotsep mu\hbox{.}\mkern \@dotsep mu$}\hfill
    \nobreak
    \hbox to\@pnumwidth{\@tocpagenum{\ifnum#1=1\bfseries\fi#7}}\par% <-- \bfseries for \section page
    \nobreak
    \endgroup
  \fi}
\AtBeginDocument{%
\expandafter\renewcommand\csname r@tocindent0\endcsname{0pt}
}
\def\l@subsection{\@tocline{2}{0pt}{2.5pc}{5pc}{}}
\def\l@subsubsection{\@tocline{2}{0pt}{4.5pc}{5pc}{}}
\makeatother

\advance\footskip0.4cm
\textheight=54pc    %a4paper
\textheight=50.5pc %letterpaper
\advance\textheight-0.4cm
\calclayout

%Font settings
%\usepackage{anyfontsize}
%Footnote settings
%\usepackage{mathptmx}
\usepackage{footmisc}
%	\renewcommand*{\thefootnote}{\fnsymbol{footnote}}
\usepackage{commath}
%Further math environments
%Further math fonts (loads amsfonts implicitely)
%Redefinition of \text
%\usepackage{amstext}
\usepackage{upref}
%Graphics
%\usepackage{graphicx}
%\usepackage{caption}
%\usepackage{subcaption}
%Frames
\usepackage{mdframed}
\allowdisplaybreaks
%\usepackage{interval}
\newcommand{\toup}{%
  \mathrel{\nonscript\mkern-1.2mu\mkern1.2mu{\uparrow}}%
}
\newcommand{\todown}{%
  \mathrel{\nonscript\mkern-1.2mu\mkern1.2mu{\downarrow}}%
}
\AtBeginDocument{\renewcommand*\d{\mathop{}\!\mathrm{d}}}
\renewcommand{\Re}{\operatorname{Re}}
\renewcommand{\Im}{\operatorname{Im}}
\DeclareMathOperator\Log{Log}
\DeclareMathOperator\Arg{Arg}
\DeclareMathOperator\sech{sech}
\DeclareMathOperator*\esssup{ess.sup}
\DeclareMathOperator\id{id}
\DeclareMathOperator\im{im}
\DeclareMathOperator\Vol{Vol}
\DeclareMathOperator\dist{dist}
%\usepackage{hhline}
%\usepackage{booktabs} 
%\usepackage{array}
%\usepackage{xfrac} 
%\everymath{\displaystyle}
%Enumerate
\usepackage{tikz}
\usetikzlibrary{external}
\tikzexternalize % activate!
\usetikzlibrary{patterns}
\pgfdeclarepatternformonly{adjusted lines}{\pgfqpoint{-1pt}{-1pt}}{\pgfqpoint{40pt}{40pt}}{\pgfqpoint{39pt}{39pt}}%
{
  \pgfsetlinewidth{.8pt}
  \pgfpathmoveto{\pgfqpoint{0pt}{0pt}}
  \pgfpathlineto{\pgfqpoint{39.1pt}{39.1pt}}
  \pgfusepath{stroke}
}
\usepackage{enumitem} 
%\renewcommand{\labelitemi}{$\bullet$}
%\renewcommand{\labelitemii}{$\ast$}
%\renewcommand{\labelitemiii}{$\cdot$}
%\renewcommand{\labelitemiv}{$\circ$}
%Colors
%\usepackage{color}
%\usepackage[cmtip, all]{xy}
%Theorems
\newtheoremstyle{main} 		             	 		%Stylename
  	{}	                                     		%Space above
  	{}	                                    		%Space below
  	{\itshape}			                     		%Body font
  	{}        	                             		%Indent
  	{\bfseries}   	                         		%Head font
  	{.}            	                        		%Head punctuation
  	{ }           	                         		%Head space 
  	{\thmname{#1}\thmnumber{ #2}\thmnote{ (#3)}}	%Head specification
\theoremstyle{main}
\newtheorem{definition}{Definition}[section]
\newtheorem{proposition}{Proposition}[section]
\newtheorem{corollary}{Corollary}[section]
\newtheorem{theorem}{Theorem}[section]
\newtheorem{lemma}{Lemma}[section]
%Roman style theorems
\newtheoremstyle{roman}
	{}
	{}
  	{}
  	{}
	{\bfseries}
	{.}
  	{ }
	{\thmname{#1}\thmnumber{ #2}\thmnote{ (#3)}}
\theoremstyle{roman}
\newtheorem{example}{Example}[section]
\newtheorem{solution}{Solution}[section]
\newtheorem{remark}{Remark}[section]
%Exercise style theorems
\newtheoremstyle{exercise}
  	{}
  	{}
  	{\small}
  	{}
  	{\bfseries}
  	{.}
 	{ }
  	{\thmname{#1}\thmnumber{ #2}\thmnote{ (#3)}}
\theoremstyle{exercise}
\newtheorem{exercise}{Exercise}[section]
%Changing default style of proof environment
\renewcommand*{\proofname}{\itshape Proof}
%German non-ASCII-Characters
%Graphics-Tool
%\usepackage{tikz}
%\usepackage{tikzscale}
%\usepackage{bbm}
%\usepackage{bera}
%Listing-Setup
%Bibliographie
\usepackage[backend=bibtex, style=alphabetic]{biblatex}
%\usepackage[babel, german = swiss]{csquotes}
\bibliography{../latex/bibliography}
%PDF-Linking
%\usepackage[hyphens]{url}
\usepackage[bookmarksopen=true,bookmarksnumbered=true]{hyperref}
%\PassOptionsToPackage{hyphens}{url}\usepackage{hyperref}
\hypersetup{
  colorlinks   = true, %Colours links instead of ugly boxes
  urlcolor     = blue, %Colour for external hyperlinks
  linkcolor    = blue, %Colour of internal links
  citecolor    = blue %Colour of citations
}
%Weierstrass-P symbol for power set
\newcommand{\powerset}{\raisebox{.15\baselineskip}{\Large\ensuremath{\wp}}}
\newcommand{\bld}[1]{\boldmath\textit{\textbf{#1}}\unboldmath}
\usepackage{pict2e}
\makeatletter
\DeclareRobustCommand{\intprod}{%
	\mathbin{\mathpalette\int@prod{(0.1,0)(0.9,0)(0.9,0.8)}}%
}
\newcommand{\int@prod}[2]{%
	\begingroup
	\sbox\z@{$\m@th#1+$}%
	\setlength\unitlength{\wd\z@}%
	\begin{picture}(1,1)
	\roundcap
	\polyline#2
	\end{picture}%
	\endgroup
}
\makeatother
\newcommand{\Sbb}{\mathbb{S}}
\newcommand{\dRrm}{\mathrm{dR}}
\newcommand{\Rbb}{\mathbb{R}}
\newcommand{\Lcal}{\mathcal{L}}
\newcommand{\Zbb}{\mathbb{Z}}
\newcommand{\Nbb}{\mathbb{N}}
\newcommand{\Xfrak}{\mathfrak{X}}


\title{The Tubular Neighbourhood Theorem}
\author{Yannis B\"ahni}
\address[Yannis B\"ahni]{University of Zurich, R\"amistrasse 71, 8006 Zurich}
\email[Yannis B\"ahni]{\href{mailto:yannis.baehni@uzh.ch}{yannis.baehni@uzh.ch}}

\begin{document}

\maketitle
\thispagestyle{fancy}

\section{Prerequisites}

\begin{definition}
	Let $(X,d)$ be a metric space and $A \subseteq X$. For $x \in X$, define the \bld{distance from $x$ to $A$}, written $\dist(x,A)$, by
	\begin{equation*}
		\dist(x,A) := \inf_{a \in A} d(x,a).
	\end{equation*}
\end{definition}

\begin{lemma}
	Let $(X,d)$ be a metric space and $A \subseteq X$ nonempty. Then $\dist(\cdot,A): X \to \Rbb$ is a continuous function.
	\label{lem:distance_continuous}
\end{lemma}

\begin{proof}
	We show that $\dist(\cdot,A)$ is in fact Lipschitz continuous. Let $x,y \in X$. Then for any $a \in A$ we have that
	\begin{equation*}
		\dist(x,A) \leq d(x,a) \leq d(x,y) + d(y,a).
	\end{equation*}
	Hence $\dist(x,A) - d(x,y)$ is a lower bound for $d(y,a)$ for any $a \in A$. But this means
	\begin{equation*}
		\dist(x,A) - d(x,y) \leq \dist(y,A).
	\end{equation*}
	Reversing the roles of $x$ and $y$ in the previous argument and applying the symmetry of the metric, we get that 
	\begin{equation*}
		\abs[0]{\dist(x,A) - \dist(y,A)} \leq d(x,y).
	\end{equation*}
\end{proof}

\begin{lemma}
	Let $(X,d)$ be a metric space and $K \subseteq X$ be compact and nonempty. If $\dist(x,K) = 0$ for some $x \in X$, then $x \in K$.
	\label{lem:distance_compact}
\end{lemma}

\begin{proof}
	For any $\varepsilon > 0$, we find $y \in K$ such that 
	\begin{equation*}
		\dist(x,K) \leq d(x,y) < \dist(x,K) + \varepsilon.
	\end{equation*}
	Thus we find a sequence $(y_n)_{n \in \Nbb}$, such that $\dist(x,y_n) \to 0$. Since $K$ is compact, there exists a subsequence $y_{n_k}$ in $K$ such that $y_{n_k} \to y$, where $y \in K$. But then 
	\begin{equation*}
		d(x,y) = \lim_{k \to \infty}d(x,y_{n_k}) = 0
	\end{equation*}
	\noindent which implies $x = y$ and so $x \in K$.
\end{proof}

\begin{theorem}[Inverse Function Theorem Generalization, Compact Case]
	Let $M$ and $N$ be smooth manifolds, $K$ a compact subspace of $M$ and $F : M \to N$ a smooth mapping, such that $F \vert_K$ is injective and $dF_p$ is nonsingular for any $p \in K$. Then there exists a neighbourhood $U$ of $K$ in $M$ and a neighbourhood $V$ of $F(K)$ in $N$ such that $F\vert_U : U \to V$ is a diffeomorphism.
	\label{thm:generalization_ift_compact}
\end{theorem}

\begin{proof}
	By corollary 13.30 \cite[341]{lee:smooth_manifolds:2013}, every smooth manifold is metrizable. Hence we can equip $M$ with a metric $d$. Moreover, the metric topology on $M$ induced by $d$ is the same as the original manifold topology. By proposition 1.12 \cite[9]{lee:smooth_manifolds:2013}, every topological manifold is locally compact, hence by proposition 4.63 \cite[104--105]{lee:topological_manifolds:2011}, each point of $M$ has a precompact neighbourhood. Since $K \subseteq M$, we find for any $p \in K$ a precompact neighbourhood $V_p$ of $p$. Thus $(V_p)_{p \in K}$ is an open cover of $K$ and the compactness of $K$ implies that there exists a finite subcover $V_{p_1},\dots,V_{p_n}$ of $K$.\\
	For any $\varepsilon > 0$, define
	\begin{equation*}
		U_\varepsilon := \cbr[0]{ p \in M : \dist(p,K) < \varepsilon}.
	\end{equation*}
	By lemma \ref{lem:distance_continuous}, $U_\varepsilon$ is open since $U_\varepsilon = \dist(\cdot,A)^{-1}(\intoo{-\infty,\varepsilon})$. Thus 
	\begin{equation*}
		W_\varepsilon := \bigcup_{i = 1}^n (V_{p_i} \cap U_\varepsilon)
	\end{equation*}
	\noindent is open and clearly $K \subseteq W_\varepsilon$ for any $\varepsilon > 0$. Hence $W_\varepsilon$ is a neighbourhood of $K$.\\
	Assume now that $F$ is not injective on any neighbourhood of $K$. For any $n \in \Nbb$ we thus find $x_n,y_n \in W_{1/n}$ such that $x_n \neq y_n$ but $F(x_n) = F(y_n)$. Hence we have constructed two sequences $(x_n)_{n \in \Nbb}$ and $(y_n)_{n \in \Nbb}$ in $W_1$. Now by 
	\begin{equation*}
		W_\varepsilon = \bigcup_{i = 1}^n (V_{p_i} \cap U_\varepsilon) \subseteq \bigcup_{i = 1}^n V_{p_i} \subseteq \bigcup_{i = 1}^n \wbar{V}_{p_i} 
	\end{equation*}
	\noindent we get that $W_\varepsilon$ is contained in a compact set. Thus we find $p_1,p_2 \in \cup_{i = 1}^n \wbar{V}_{p_i}$ such that $x_{n_k} \to p$ and $y_{n_k} \to q$. But 
	\begin{equation*}
		\dist(p,K) = \lim_{k \to \infty}\dist(x_{n_k},A) \leq \lim_{k \to \infty}\frac{1}{n_k} = 0.
	\end{equation*}
	\noindent by the continuity of the distance function and so $\dist(p,K) = \dist(q,K) = 0$. But this implies $p,q \in K$ by lemma \ref{lem:distance_compact}. Moreover, since $F$ is continuous by \cite[34]{lee:smooth_manifolds:2013} we have that
	\begin{equation*}
		F(p) = \lim_{k \to \infty} F(x_{n_k}) = \lim_{k \to \infty}F(y_{n_k}) = F(q)
	\end{equation*}
	\noindent and so by injectivity of $F\vert_K$ we get that $p = q$.\\
	Finally, since $dF_p$ is nonsingular, the inverse function theorem for manifolds \cite[79]{lee:smooth_manifolds:2013} guarantees the existence of neighbourhoods $U_0$ of $p$ and $V_0$ of $F(p)$ such that $F\vert_{U_0} : U_0 \to V_0$ is a diffeomorphism. Since $x_{n_k}$ and $y_{n_k}$ both converge to $p$ and $x_{n_k} \neq y_{n_k}$ for all $k \in \Nbb$ but $F(x_{n_k}) = F(y_{n_k})$, we get that $F$ cannot be injective, hence no diffeomorphism, which is a contradiction. Hence there exists a neighbourhood $W$ of $K$ such that $F\vert_W$ is injective.\\
	Since $dF_p$ is nonsingular for any $p \in K$, there exist neighbourhoods $U_{0,p}$ of $p$ and $V_{0,p}$ of $F(p)$ such that $F\vert_{U_{0,p}} : U_{0,p} \to V_{0,p}$ is a diffeomorphism by the inverse function theorem. Moreover, for any $p \in K$ there exists $r_p$ such that $B_{r_p}(p) \subseteq U_{0,p}$ and by shrinking $r_p$, if necessary, we may assume that $B_{r_p}(p) \subseteq W$. Set
	\begin{equation*}
		U := \bigcup_{p \in K} B_{r_p}(p) \qquad \text{and} \qquad V := F(U) = \bigcup_{p \in K} F(B_{r_p}(p)).
	\end{equation*}
	\noindent Then $K \subseteq U$, $U$ is open and $F(K) \subseteq F(U) = V$. Also $F(B_{r_p}(p))$ is open in $N$. Indeed, $F\vert_{U_{0,p}} : U_{0,p} \to V_{0,p}$ is a diffeomorphism and thus an open map. Since $B_{r_p}(p) = U_{0,p} \cap B_{r_p}(p)$, $B_{r_p}(p)$ is also open in $U_{0,p}$. So $F(B_{r_p}(p))$ is open in $V_{0,p}$. But this means that there exists an open set $B$ in $N$ such that $F(B_{r_p}(p)) = V_{0,p} \cap B$, the right hand side is open in $N$ and so is $F(B_{r_p}(p))$. Hence $V$ is open in $N$ as a union of open sets. Moreover, $F$ is bijective and a local diffeomorphism. Thus by proposition 4.6 (f) \cite[80]{lee:smooth_manifolds:2013} $F\vert_U : U \to V$ is a diffeomorphism.
\end{proof}

\begin{definition}[Normal Bundle]
	Let $(M,g)$ be a Riemannian manifold and $S \subseteq M$ a Riemannian submanifold. Let $p \in S$. The \bld{normal space to $S$ at $p$} is defined by
	\begin{equation}
		N_pS := \cbr{v \in T_pM : \forall w \in T_pS(\langle v,w \rangle_g = 0) }
	\end{equation}
	\noindent and the \bld{normal bundle of $S$} as
	\begin{equation}
		NS := \coprod_{p \in S} N_pS.
	\end{equation}
\end{definition}

\end{document}
