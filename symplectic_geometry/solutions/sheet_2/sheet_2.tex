%%%%%%%%%%%%%%%%%%%%%%%%%%%%%%%%%%%%%%%%%%%%%%%%%%%%%%%%%%%%%%%%%%%%%%%%%%
%Author:																 %
%-------																 %
%Yannis Baehni at University of Zurich									 %
%baehni.yannis@uzh.ch													 %
%																		 %
%Version log:															 %
%------------															 %
%06/02/16 . Basic structure												 %
%04/08/16 . Layout changes including section, contents, abstract.		 %
%%%%%%%%%%%%%%%%%%%%%%%%%%%%%%%%%%%%%%%%%%%%%%%%%%%%%%%%%%%%%%%%%%%%%%%%%%

%Page Setup
\documentclass[
	12pt, 
	oneside, 
	a4paper,
	reqno,
	final
]{amsbook}

\usepackage[
	left = 3cm, 
	right = 3cm, 
	top = 3cm, 
	bottom = 3cm
]{geometry}

%Headers and footers
\usepackage{fancyhdr}
	\pagestyle{fancy}
	%Clear fields
	\fancyhf{}
	%Header right
	\fancyhead[R]{
		\footnotesize
		Yannis B\"{a}hni\\
		\href{mailto:yannis.baehni@uzh.ch}{yannis.baehni@uzh.ch}
	}
	%Header left
	\fancyhead[L]{
		\footnotesize
		401-3001-61L Algebraic Topology I\\
		Autumn Semester 2017
	}
	%Page numbering in footer
	\fancyfoot[C]{\thepage}
	%Separation line header and footer
	\renewcommand{\headrulewidth}{0.4pt}
	%\renewcommand{\footrulewidth}{0.4pt}
	
	\setlength{\headheight}{19pt} 

%Title
\usepackage[foot]{amsaddr}
\usepackage{newtxtext}
\usepackage[subscriptcorrection,nofontinfo,mtpcal,mtphrb]{mtpro2}
\usepackage{mathtools}
\usepackage{bm}
\usepackage{xspace}
\usepackage[all]{xy}
\usepackage{tikz-cd}
\makeatletter
\def\@textbottom{\vskip \z@ \@plus 1pt}
\let\@texttop\relax
\usepackage{etoolbox}
\patchcmd{\abstract}{\scshape\abstractname}{\textbf{\abstractname}}{}{}
\usepackage{chngcntr}
\counterwithout{figure}{chapter}
%Section, subsection and subsubsection font
%------------------------------------------
	\renewcommand{\@secnumfont}{\bfseries}
	\renewcommand\section{\@startsection{section}{1}%
  	\z@{.7\linespacing\@plus\linespacing}{.5\linespacing}%
  	{\normalfont\bfseries\boldmath\centering}}
	\renewcommand\subsection{\@startsection{subsection}{2}%
    	\z@{.5\linespacing\@plus.7\linespacing}{-.5em}%
    	{\normalfont\bfseries\boldmath}}%
	\renewcommand\subsubsection{\@startsection{subsubsection}{3}%
    	\z@{.5\linespacing\@plus.7\linespacing}{-.5em}%
    	{\normalfont\bfseries\boldmath}}%

		\renewenvironment{proof}{\textit{Proof}.}{\hfill\qedsymbol}

%ToC
%---
\makeatletter
\setcounter{tocdepth}{3}
% Add bold to \chapter titles in ToC and remove . after numbers
\renewcommand{\tocchapter}[3]{%
  	\indentlabel{\@ifnotempty{#2}{\bfseries\ignorespaces#1 #2: }}\bfseries#3}
\renewcommand{\tocappendix}[3]{%
  	\indentlabel{\@ifnotempty{#2}{\bfseries\ignorespaces#1 #2: }}\bfseries#3}
% Remove . after numbers in \section and \subsection
\renewcommand{\tocsection}[3]{%
  	\indentlabel{\@ifnotempty{#2}{\ignorespaces#1 #2\quad}}#3}
\renewcommand{\tocsubsection}[3]{%
  	\indentlabel{\@ifnotempty{#2}{\ignorespaces#1 #2\quad}}#3}
\let\tocsubsubsection\tocsubsection% Update for \subsubsection
%...
\newcommand\@dotsep{4.5}
\def\@tocline#1#2#3#4#5#6#7{\relax
  \ifnum #1>\c@tocdepth % then omit
  \else
    \par \addpenalty\@secpenalty\addvspace{#2}%
    \begingroup \hyphenpenalty\@M
    \@ifempty{#4}{%
      \@tempdima\csname r@tocindent\number#1\endcsname\relax
    }{%
      \@tempdima#4\relax
    }%
    \parindent\z@ \leftskip#3\relax \advance\leftskip\@tempdima\relax
    \rightskip\@pnumwidth plus1em \parfillskip-\@pnumwidth
    #5\leavevmode\hskip-\@tempdima{#6}\nobreak
    \leaders\hbox{$\m@th\mkern \@dotsep mu\hbox{.}\mkern \@dotsep mu$}\hfill
    \nobreak
    \hbox to\@pnumwidth{\@tocpagenum{\ifnum#1=0\bfseries\fi#7}}\par% <-- \bfseries for \chapter page
    \nobreak
    \endgroup
  \fi}
\AtBeginDocument{%
\expandafter\renewcommand\csname r@tocindent0\endcsname{0pt}
}
\def\l@subsection{\@tocline{2}{0pt}{2.5pc}{5pc}{}}
\def\l@subsubsection{\@tocline{2}{0pt}{4.5pc}{5pc}{}}
\makeatother

\advance\footskip0.4cm
\textheight=54pc    %a4paper
\textheight=50.5pc %letterpaper
\advance\textheight-0.4cm
\calclayout

%Font settings
%\usepackage{anyfontsize}
%Footnote settings
\usepackage{footmisc}
%	\renewcommand*{\thefootnote}{\fnsymbol{footnote}}
\usepackage{commath}
%Further math environments
%Further math fonts (loads amsfonts implicitely)
%Redefinition of \text
%\usepackage{amstext}
\usepackage{upref}
%Graphics
%\usepackage{graphicx}
%\usepackage{caption}
%\usepackage{subcaption}
%Frames
\usepackage{mdframed}
\allowdisplaybreaks
%\usepackage{interval}
\newcommand{\toup}{%
  \mathrel{\nonscript\mkern-1.2mu\mkern1.2mu{\uparrow}}%
}
\newcommand{\todown}{%
  \mathrel{\nonscript\mkern-1.2mu\mkern1.2mu{\downarrow}}%
}
\AtBeginDocument{\renewcommand*\d{\mathop{}\!\mathrm{d}}}
\renewcommand{\Re}{\operatorname{Re}}
\renewcommand{\Im}{\operatorname{Im}}
\DeclareMathOperator\Log{Log}
\DeclareMathOperator\Arg{Arg}
\DeclareMathOperator\id{id}
\DeclareMathOperator\sech{sech}
\DeclareMathOperator\Aut{Aut}
\DeclareMathOperator\h{h}
\DeclareMathOperator\sgn{sgn}
\DeclareMathOperator\arctanh{arctanh}
\DeclareMathOperator\supp{supp}
\DeclareMathOperator\ob{ob}
\DeclareMathOperator\mor{mor}
\DeclareMathOperator\M{M}
\DeclareMathOperator\dom{dom}
\DeclareMathOperator\cod{cod}
\DeclareMathOperator\im{im}
\DeclareMathOperator\Ab{Ab}
\DeclareMathOperator\coker{coker}
%\usepackage{hhline}
%\usepackage{booktabs} 
%\usepackage{array}
%\usepackage{xfrac} 
%\everymath{\displaystyle}
%Enumerate
\usepackage{tikz}
%\usepackgae{graphicx}
\usepackage{subcaption}
\usepackage{enumitem} 
%\renewcommand{\labelitemi}{$\bullet$}
%\renewcommand{\labelitemii}{$\ast$}
%\renewcommand{\labelitemiii}{$\cdot$}
%\renewcommand{\labelitemiv}{$\circ$}
%Colors
%\usepackage{color}
%\usepackage[cmtip, all]{xy}
%Main style theorem environment
\newtheoremstyle{main} 		             	 		%Stylename
  	{}	                                     		%Space above
  	{}	                                    		%Space below
  	{\itshape}			                     		%Body font
  	{}        	                             		%Indent
  	{\bfseries\boldmath}   	                         		%Head font
  	{.}            	                        		%Head punctuation
  	{ }           	                         		%Head space 
  	{\thmname{#1}\thmnumber{ #2}\thmnote{ (#3)}}	%Head specification
\theoremstyle{main}
\newtheorem{definition}{Definition}[chapter]
\newtheorem{proposition}{Proposition}[chapter]
\newtheorem{corollary}{Corollary}[chapter]
\newtheorem{theorem}{Theorem}[chapter]
\newtheorem{lemma}{Lemma}[chapter]
\newtheoremstyle{nonit} 		             	 		%Stylename
  	{}	                                     		%Space above
  	{}	                                    		%Space below
  	{}			                     		%Body font
  	{}        	                             		%Indent
  	{\bfseries\boldmath}   	                   		%Head font
  	{.}            	                        		%Head punctuation
  	{ }           	                         		%Head space 
  	{\thmname{#1}\thmnumber{ #2}\thmnote{ (#3)}}	%Head specification
\theoremstyle{nonit}
\newtheorem{remark}{Remark}[chapter]
\newtheorem{examples}{Examples}[chapter]
\newtheorem{example}{Example}[chapter]
\newtheorem{problem}{Problem}[chapter]
\newtheoremstyle{ex} 		             	 		%Stylename
  	{}	                                     		%Space above
  	{}	                                    		%Space below
  	{\small}			                     		%Body font
  	{}        	                             		%Indent
  	{\bfseries\boldmath}   	                         		%Head font
  	{.}            	                        		%Head punctuation
  	{ }           	                         		%Head space 
  	{\thmname{#1}\thmnumber{ #2}\thmnote{ (#3)}}	%Head specification
\theoremstyle{ex}
\newtheorem{exercise}{Exercise}[chapter]
%German non-ASCII-Characters
%Graphics-Tool
%\usepackage{tikz}
%\usepackage{tikzscale}
%\usepackage{bbm}
%\usepackage{bera}
%Listing-Setup
%Bibliographie
\usepackage[backend=bibtex, style=alphabetic]{biblatex}
%\usepackage[babel, german = swiss]{csquotes}
\bibliography{bibliography}
%PDF-Linking
%\usepackage[hyphens]{url}
\usepackage[bookmarksopen=true,bookmarksnumbered=true]{hyperref}
%\PassOptionsToPackage{hyphens}{url}\usepackage{hyperref}
\urlstyle{rm}
\hypersetup{
  colorlinks   = true, %Colours links instead of ugly boxes
  urlcolor     = blue, %Colour for external hyperlinks
  linkcolor    = blue, %Colour of internal links
  citecolor    = blue %Colour of citations
}
\newcommand{\bld}[1]{\boldmath\textit{\textbf{#1}}\unboldmath}
\newcommand{\eqclass}[1]{\sbr[0]{#1}}
\newcommand{\cat}[1]{\mathsf{#1}}
\newcommand{\Sbb}{\mathbb{S}}
\newcommand{\Zbb}{\mathbb{Z}}
\newcommand{\Nbb}{\mathbb{N}}
\newcommand{\Rbb}{\mathbb{R}}
\newcommand{\Hbb}{\mathbb{H}}
\newcommand{\Cbb}{\mathbb{C}}
\newcommand{\Tcal}{\mathcal{T}}
\newcommand{\SLrm}{\mathrm{SL}}
\newcommand{\PSLrm}{\mathrm{PSL}}
\newcommand{\SLrmstar}{\mathrm{S^*L}}
\newcommand{\PSLrmstar}{\mathrm{PS^*L}}
\newcommand{\GLrm}{\mathrm{GL}}
\newcommand{\Mrm}{\mathrm{M}}
\newcommand{\Isom}{\mathrm{Isom}}
\newcommand{\Mob}{\mathrm{M\ddot{o}b}}
\newcommand{\Ebb}{\mathbb{E}}
\newcommand{\Cscr}{\mathscr{C}}
\newcommand{\pwrm}{\mathrm{pw}}
\newcommand{\clos}[1]{\overline{#1}}
\newcommand{\Hcal}{\mathcal{H}}
\newcommand{\Hbcal}{\bm{\mathcal{H}}}
\newcommand{\Ucal}{\mathcal{U}}
\newcommand{\Ubcal}{\bm{\mathcal{U}}}
\renewcommand{\det}{\mathrm{det}}
\newcommand{\ab}{\mathrm{ab}}


\title{Homework 2: Symplectic Forms vs. Area and Volume}
\author{Yannis B\"ahni}
\address[Yannis B\"ahni]{University of Zurich, R\"amistrasse 71, 8006 Zurich}
\email[Yannis B\"ahni]{\href{mailto:yannis.baehni@uzh.ch}{yannis.baehni@uzh.ch}}

\begin{document}

\maketitle
\thispagestyle{fancy}
\setcounter{section}{1}

\begin{exercise}
	\label{ex:1}
Let $(M,\omega)$ be a $2n$-dimensional symplectic manifold. 
\begin{enumerate}[label = \textup{(}\alph*\textup{)}]
\item $\omega^n$ is a volume form.
\item Show that if $M$ is compact, then $\sbr[0]{\omega^n} \in H_{\dRrm}^{2n}(M)$ is nonzero.
\item Conclude that $\sbr[0]{\omega} \neq 0$.
\item $\Sbb^{2n}$ does not admit a symplectic structure for $n > 1$.
\end{enumerate}
\end{exercise}

\begin{solution}
	Part (a) immediately follows from the fact that for each $p \in M$ we have that $\omega_p^n \neq 0$. Thus $\omega^n$ is a nonvanishing form of top degree, hence a volume form.\\
	For proving (b), assume that $\sbr[0]{\omega^n} = 0$. Hence $\omega^n$ is exact. Thus there exists $\mu \in \Omega^{2n - 1}(M)$ such that $\omega^n = d\mu$. But then Stoke's theorem \cite[411]{lee:smooth_manifolds:2013} together with positivity \cite[407]{lee:smooth_manifolds:2013} yields
	\begin{equation*}
		0 < \int_{M} \omega^n = \int_M d\mu = \int_{\partial M} \mu	= \int_\varnothing \mu = 0
	\end{equation*}
	\noindent since $M$ is oriented by part (a) and $\omega^n$ is a positively oriented orientation form (see \cite[381]{lee:smooth_manifolds:2013}).\\
	For proving (c), observe that $\sbr[0]{\omega^n} = \sbr[0]{\omega} \cupprod \dots \cupprod \sbr[0]{\omega}$, where $\cupprod$ is the so-called cup product (see \cite[464]{lee:smooth_manifolds:2013}). So if $\sbr[0]{\omega} = 0$, we have by bilinearity also $\sbr[0]{\omega^n} = 0$, which contradicts part (b).\\
For proving (d), by \cite[450]{lee:smooth_manifolds:2013} we have that 
\begin{equation*}
	H_{\dRrm}^p(\Sbb^n) \cong \ccases{
		\Rbb & p = 0 \text{ or } p = n,\\
		0 & 0 < p < n,
	}
\end{equation*}
\noindent for $n \geq 1$. Let $n > 1$. Assume that $(\Sbb^n, \omega)$ is a symplectic manifold. Since $\Sbb^n$ is compact, part (c) implies that $\sbr[0]{\omega} \neq 0$. But $\sbr[0]{\omega} \in H_{\dRrm}^2(\Sbb^{2n}) \cong 0$.
\end{solution}

\begin{example}
	Consider the symplectic manifold $(\Rbb^{2n},\omega_0)$, where $\omega_0$ is the standard symplectic structure on $\Rbb^{2n}$. Clearly, $\Rbb^{2n}$ is not compact and $\omega_0$ is exact since
	\begin{equation*}
		d \PARENS{\sum_{i = 1}^n x^i dy^i} = \sum_{i = 1}^n dx^i \wedge dy^i = \omega_0.
	\end{equation*}
\end{example}

\begin{example}
	Let $M$ be a smooth manifold. Then $(T^*M, \omega)$ is a symplectic manfiold, where $\omega$ is the canonical symplectic form on $T^*M$. It is an exact form, since $\omega = - d\alpha$, where $\alpha$ is the tautological $1$-form. Moreover, $T^*M$ is not compact by problem 10-19 \cite[271]{lee:smooth_manifolds:2013}.	
\end{example}

\begin{exercise}
Let $(M,\omega)$ be a $2n$-dimensional symplectic manifold. 
\begin{enumerate}[label = \textup{(}\alph*\textup{)}]
\item 
\end{enumerate}
\end{exercise}

\begin{solution}
	For proving (a), we have using \cite[117]{lee:smooth_manifolds:2013}
	\begin{equation*}
		T_p\Sbb^n = \cbr[0]{v \in \Rbb^{n + 1} : \langle v,p \rangle = 0}
	\end{equation*}
	\noindent for each $p \in \Sbb^n$. Consider the \bld{Euler vector field $V$} defined by
	\begin{equation*}
		V := x^i \frac{\partial}{\partial x^i}.
	\end{equation*}
	Then $V$ is a unit normal vector field along $\Sbb^n$. Indeed, if $p \in \Sbb^n$ and $v \in T_p\Sbb^n$ we have that 
	\begin{equation*}
		\langle p,v \rangle_{\wbar{g}} = \langle p,v \rangle = 0
	\end{equation*}
	\noindent and
	\begin{equation*}
		\abs[0]{p}_{\wbar{g}} = \abs[0]{p} = 1.
	\end{equation*}
	Hence by \cite[390]{lee:smooth_manifolds:2013}, the volume form $\omega_{\mathring{g}}$ on $(\Sbb^n,\mathring{g})$ is given by
	\begin{equation*}
		\omega_{\mathring{g}} = \iota_{\Sbb^n}^*\PARENS{i_V\omega_{\wbar{g}}}.
	\end{equation*}
	More precisely, in the case $n = 2$ we have
	\begin{align*}
		i_V\omega_{\wbar{g}} &= i_V(dx \wedge dy \wedge dz)\\
		&= (i_V dx) \wedge dy \wedge dz - dx \wedge i_V(dy \wedge dz)\\
		&= (i_V dx) \wedge dy \wedge dz - dx \wedge (i_V dy) \wedge dz + dx \wedge dy \wedge (i_V dz)\\
		&= x dy \wedge dz + y dz \wedge dx+ z dx \wedge dy.
	\end{align*}
	For $v,w \in T_p\Sbb^2$, $p \in \Sbb^2$, we have that
	\begin{equation*}
		\omega_{\mathring{g}}\vert_p(v,w) = (i_V\omega_{\wbar{g}})_{\iota(p)}\PARENS{d\iota_p(v),d\iota_p(w)} = (i_V\omega_{\wbar{g}})\vert_p(v,w) 
	\end{equation*}
	\noindent under the usual identification of $T_p\Sbb^n$ as a linear subspace of $T_p\Rbb^{n+1}$. Finally
	\begin{align*}
		\omega_{\mathring{g}}(v,w) &= (x dy \wedge dz + y dz \wedge dx+ z dx \wedge dy)(v,w)\\
		&= x \det\begin{pmatrix}
			dy(v) & dz(v)\\
			dy(w) & dz(w)
		\end{pmatrix}
		+ y\det\begin{pmatrix}
			dz(v) & dx(v)\\
			dz(w) & dx(w)
		\end{pmatrix}
		+ z \det\begin{pmatrix}
			dx(v) & dy(v)\\
			dx(w) & dy(w)
		\end{pmatrix}\\
		&= x(v^2w^3 - w^2v^3) + y(v^3w^1 - w^3v^1) + z(v^1w^2 - w^1v^2)\\
		&= \langle p, v \times w \rangle
	\end{align*}
	\noindent for $p := (x,y,z) \in \Sbb^2$ using \cite[356]{lee:smooth_manifolds:2013}.\\
	For proving (b), consider cylindrical polar coordinates $(\theta,z)$ on $\Sbb^2$ given by
	\begin{equation*}
		(x,y,z) = (\sqrt{1 - z^2}\cos \theta,\sqrt{1 - z^2}\sin\theta,z).
	\end{equation*}
	Then we get
	\begin{align*}
		i_V \omega_{\wbar{g}} =& \id^*(i_V\omega_{\wbar{g}})\\
		=& \id^*(x dy \wedge dz + y dz \wedge dx+ z dx \wedge dy)\\
		=& \sqrt{1 - z^2}\cos \theta d(\sqrt{1 - z^2}\sin \theta) \wedge dz + \sqrt{1 - z^2}\sin \theta dz \wedge d(\sqrt{1 - z^2}\cos \theta)\\
		&+ z d(\sqrt{1 - z^2}\cos \theta) \wedge d(\sqrt{1 - z^2}\sin \theta)\\
		=& \sqrt{1 - z^2}\cos \theta \del[3]{\sqrt{1 - z^2}\cos \theta d\theta - \frac{z}{\sqrt{1 - z^2}}\sin \theta dz} \wedge dz\\ 
		&- \sqrt{1 - z^2}\sin \theta dz \wedge \del[3]{\sqrt{1 - z^2}\sin \theta d\theta + \frac{z}{\sqrt{1 - z^2}}\cos\theta dz}\\
		&- z  \del[3]{\sqrt{1 - z^2}\sin \theta d\theta + \frac{z}{\sqrt{1 - z^2}}\cos\theta dz}\\
		&\wedge \del[3]{\sqrt{1 - z^2}\cos \theta d\theta - \frac{z}{\sqrt{1 - z^2}}\sin \theta dz}\\
		=& (1 - z^2)\cos^2 \theta d\theta \wedge dz - (1 - z^2)\sin^2\theta dz \wedge d\theta + z^2 \sin^2 \theta d\theta \wedge dz \\
		&- z^2 \cos^2 \theta dz \wedge d\theta\\
		=& d\theta \wedge dz.
	\end{align*}
	For proving (c), just observe that
	\begin{equation*}
		\Vol(\Sbb^2) = \int_{\Sbb^2}\omega_{\mathring{g}} = \int_{\intoo{0,2\pi} \times \intoo{-1,1}} d\theta \wedge dz = 4\pi.	
	\end{equation*}
\end{solution}

\begin{exercise}
Let $(M,\omega)$ be a $2n$-dimensional symplectic manifold. 
\begin{enumerate}[label = \textup{(}\alph*\textup{)}]
\item 
\end{enumerate}
\end{exercise}

\begin{solution}
	For (a), observe that by exercise \ref{ex:1}, the manifold must be orientable. Thus surfaces like the sphere, torus and connected sums of these (classification theorem) do possess a symplectic structure, whereas the non-orientable surfaces like the real projective plane and the M\"obius strip do not possess any symplectic structures.
\end{solution}

\end{document}
