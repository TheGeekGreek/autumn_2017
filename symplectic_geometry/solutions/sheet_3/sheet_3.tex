%%%%%%%%%%%%%%%%%%%%%%%%%%%%%%%%%%%%%%%%%%%%%%%%%%%%%%%%%%%%%%%%%%%%%%%%%%
%Author:																 %
%-------																 %
%Yannis Baehni at University of Zurich									 %
%baehni.yannis@uzh.ch													 %
%																		 %
%Version log:															 %
%------------															 %
%06/02/16 . Basic structure												 %
%04/08/16 . Layout changes including section, contents, abstract.		 %
%%%%%%%%%%%%%%%%%%%%%%%%%%%%%%%%%%%%%%%%%%%%%%%%%%%%%%%%%%%%%%%%%%%%%%%%%%

%Page Setup
\documentclass[
	12pt, 
	oneside, 
	a4paper,
	reqno,
	final
]{amsbook}

\usepackage[
	left = 3cm, 
	right = 3cm, 
	top = 3cm, 
	bottom = 3cm
]{geometry}

%Headers and footers
\usepackage{fancyhdr}
	\pagestyle{fancy}
	%Clear fields
	\fancyhf{}
	%Header right
	\fancyhead[R]{
		\footnotesize
		Yannis B\"{a}hni\\
		\href{mailto:yannis.baehni@uzh.ch}{yannis.baehni@uzh.ch}
	}
	%Header left
	\fancyhead[L]{
		\footnotesize
		401-3001-61L Algebraic Topology I\\
		Autumn Semester 2017
	}
	%Page numbering in footer
	\fancyfoot[C]{\thepage}
	%Separation line header and footer
	\renewcommand{\headrulewidth}{0.4pt}
	%\renewcommand{\footrulewidth}{0.4pt}
	
	\setlength{\headheight}{19pt} 

%Title
\usepackage[foot]{amsaddr}
\usepackage{newtxtext}
\usepackage[subscriptcorrection,nofontinfo,mtpcal,mtphrb]{mtpro2}
\usepackage{mathtools}
\usepackage{bm}
\usepackage{xspace}
\usepackage[all]{xy}
\usepackage{tikz-cd}
\makeatletter
\def\@textbottom{\vskip \z@ \@plus 1pt}
\let\@texttop\relax
\usepackage{etoolbox}
\patchcmd{\abstract}{\scshape\abstractname}{\textbf{\abstractname}}{}{}
\usepackage{chngcntr}
\counterwithout{figure}{chapter}
%Section, subsection and subsubsection font
%------------------------------------------
	\renewcommand{\@secnumfont}{\bfseries}
	\renewcommand\section{\@startsection{section}{1}%
  	\z@{.7\linespacing\@plus\linespacing}{.5\linespacing}%
  	{\normalfont\bfseries\boldmath\centering}}
	\renewcommand\subsection{\@startsection{subsection}{2}%
    	\z@{.5\linespacing\@plus.7\linespacing}{-.5em}%
    	{\normalfont\bfseries\boldmath}}%
	\renewcommand\subsubsection{\@startsection{subsubsection}{3}%
    	\z@{.5\linespacing\@plus.7\linespacing}{-.5em}%
    	{\normalfont\bfseries\boldmath}}%

		\renewenvironment{proof}{\textit{Proof}.}{\hfill\qedsymbol}

%ToC
%---
\makeatletter
\setcounter{tocdepth}{3}
% Add bold to \chapter titles in ToC and remove . after numbers
\renewcommand{\tocchapter}[3]{%
  	\indentlabel{\@ifnotempty{#2}{\bfseries\ignorespaces#1 #2: }}\bfseries#3}
\renewcommand{\tocappendix}[3]{%
  	\indentlabel{\@ifnotempty{#2}{\bfseries\ignorespaces#1 #2: }}\bfseries#3}
% Remove . after numbers in \section and \subsection
\renewcommand{\tocsection}[3]{%
  	\indentlabel{\@ifnotempty{#2}{\ignorespaces#1 #2\quad}}#3}
\renewcommand{\tocsubsection}[3]{%
  	\indentlabel{\@ifnotempty{#2}{\ignorespaces#1 #2\quad}}#3}
\let\tocsubsubsection\tocsubsection% Update for \subsubsection
%...
\newcommand\@dotsep{4.5}
\def\@tocline#1#2#3#4#5#6#7{\relax
  \ifnum #1>\c@tocdepth % then omit
  \else
    \par \addpenalty\@secpenalty\addvspace{#2}%
    \begingroup \hyphenpenalty\@M
    \@ifempty{#4}{%
      \@tempdima\csname r@tocindent\number#1\endcsname\relax
    }{%
      \@tempdima#4\relax
    }%
    \parindent\z@ \leftskip#3\relax \advance\leftskip\@tempdima\relax
    \rightskip\@pnumwidth plus1em \parfillskip-\@pnumwidth
    #5\leavevmode\hskip-\@tempdima{#6}\nobreak
    \leaders\hbox{$\m@th\mkern \@dotsep mu\hbox{.}\mkern \@dotsep mu$}\hfill
    \nobreak
    \hbox to\@pnumwidth{\@tocpagenum{\ifnum#1=0\bfseries\fi#7}}\par% <-- \bfseries for \chapter page
    \nobreak
    \endgroup
  \fi}
\AtBeginDocument{%
\expandafter\renewcommand\csname r@tocindent0\endcsname{0pt}
}
\def\l@subsection{\@tocline{2}{0pt}{2.5pc}{5pc}{}}
\def\l@subsubsection{\@tocline{2}{0pt}{4.5pc}{5pc}{}}
\makeatother

\advance\footskip0.4cm
\textheight=54pc    %a4paper
\textheight=50.5pc %letterpaper
\advance\textheight-0.4cm
\calclayout

%Font settings
%\usepackage{anyfontsize}
%Footnote settings
\usepackage{footmisc}
%	\renewcommand*{\thefootnote}{\fnsymbol{footnote}}
\usepackage{commath}
%Further math environments
%Further math fonts (loads amsfonts implicitely)
%Redefinition of \text
%\usepackage{amstext}
\usepackage{upref}
%Graphics
%\usepackage{graphicx}
%\usepackage{caption}
%\usepackage{subcaption}
%Frames
\usepackage{mdframed}
\allowdisplaybreaks
%\usepackage{interval}
\newcommand{\toup}{%
  \mathrel{\nonscript\mkern-1.2mu\mkern1.2mu{\uparrow}}%
}
\newcommand{\todown}{%
  \mathrel{\nonscript\mkern-1.2mu\mkern1.2mu{\downarrow}}%
}
\AtBeginDocument{\renewcommand*\d{\mathop{}\!\mathrm{d}}}
\renewcommand{\Re}{\operatorname{Re}}
\renewcommand{\Im}{\operatorname{Im}}
\DeclareMathOperator\Log{Log}
\DeclareMathOperator\Arg{Arg}
\DeclareMathOperator\id{id}
\DeclareMathOperator\sech{sech}
\DeclareMathOperator\Aut{Aut}
\DeclareMathOperator\h{h}
\DeclareMathOperator\sgn{sgn}
\DeclareMathOperator\arctanh{arctanh}
\DeclareMathOperator\supp{supp}
\DeclareMathOperator\ob{ob}
\DeclareMathOperator\mor{mor}
\DeclareMathOperator\M{M}
\DeclareMathOperator\dom{dom}
\DeclareMathOperator\cod{cod}
\DeclareMathOperator\im{im}
\DeclareMathOperator\Ab{Ab}
\DeclareMathOperator\coker{coker}
%\usepackage{hhline}
%\usepackage{booktabs} 
%\usepackage{array}
%\usepackage{xfrac} 
%\everymath{\displaystyle}
%Enumerate
\usepackage{tikz}
%\usepackgae{graphicx}
\usepackage{subcaption}
\usepackage{enumitem} 
%\renewcommand{\labelitemi}{$\bullet$}
%\renewcommand{\labelitemii}{$\ast$}
%\renewcommand{\labelitemiii}{$\cdot$}
%\renewcommand{\labelitemiv}{$\circ$}
%Colors
%\usepackage{color}
%\usepackage[cmtip, all]{xy}
%Main style theorem environment
\newtheoremstyle{main} 		             	 		%Stylename
  	{}	                                     		%Space above
  	{}	                                    		%Space below
  	{\itshape}			                     		%Body font
  	{}        	                             		%Indent
  	{\bfseries\boldmath}   	                         		%Head font
  	{.}            	                        		%Head punctuation
  	{ }           	                         		%Head space 
  	{\thmname{#1}\thmnumber{ #2}\thmnote{ (#3)}}	%Head specification
\theoremstyle{main}
\newtheorem{definition}{Definition}[chapter]
\newtheorem{proposition}{Proposition}[chapter]
\newtheorem{corollary}{Corollary}[chapter]
\newtheorem{theorem}{Theorem}[chapter]
\newtheorem{lemma}{Lemma}[chapter]
\newtheoremstyle{nonit} 		             	 		%Stylename
  	{}	                                     		%Space above
  	{}	                                    		%Space below
  	{}			                     		%Body font
  	{}        	                             		%Indent
  	{\bfseries\boldmath}   	                   		%Head font
  	{.}            	                        		%Head punctuation
  	{ }           	                         		%Head space 
  	{\thmname{#1}\thmnumber{ #2}\thmnote{ (#3)}}	%Head specification
\theoremstyle{nonit}
\newtheorem{remark}{Remark}[chapter]
\newtheorem{examples}{Examples}[chapter]
\newtheorem{example}{Example}[chapter]
\newtheorem{problem}{Problem}[chapter]
\newtheoremstyle{ex} 		             	 		%Stylename
  	{}	                                     		%Space above
  	{}	                                    		%Space below
  	{\small}			                     		%Body font
  	{}        	                             		%Indent
  	{\bfseries\boldmath}   	                         		%Head font
  	{.}            	                        		%Head punctuation
  	{ }           	                         		%Head space 
  	{\thmname{#1}\thmnumber{ #2}\thmnote{ (#3)}}	%Head specification
\theoremstyle{ex}
\newtheorem{exercise}{Exercise}[chapter]
%German non-ASCII-Characters
%Graphics-Tool
%\usepackage{tikz}
%\usepackage{tikzscale}
%\usepackage{bbm}
%\usepackage{bera}
%Listing-Setup
%Bibliographie
\usepackage[backend=bibtex, style=alphabetic]{biblatex}
%\usepackage[babel, german = swiss]{csquotes}
\bibliography{bibliography}
%PDF-Linking
%\usepackage[hyphens]{url}
\usepackage[bookmarksopen=true,bookmarksnumbered=true]{hyperref}
%\PassOptionsToPackage{hyphens}{url}\usepackage{hyperref}
\urlstyle{rm}
\hypersetup{
  colorlinks   = true, %Colours links instead of ugly boxes
  urlcolor     = blue, %Colour for external hyperlinks
  linkcolor    = blue, %Colour of internal links
  citecolor    = blue %Colour of citations
}
\newcommand{\bld}[1]{\boldmath\textit{\textbf{#1}}\unboldmath}
\newcommand{\eqclass}[1]{\sbr[0]{#1}}
\newcommand{\cat}[1]{\mathsf{#1}}
\newcommand{\Sbb}{\mathbb{S}}
\newcommand{\Zbb}{\mathbb{Z}}
\newcommand{\Nbb}{\mathbb{N}}
\newcommand{\Rbb}{\mathbb{R}}
\newcommand{\Hbb}{\mathbb{H}}
\newcommand{\Cbb}{\mathbb{C}}
\newcommand{\Tcal}{\mathcal{T}}
\newcommand{\SLrm}{\mathrm{SL}}
\newcommand{\PSLrm}{\mathrm{PSL}}
\newcommand{\SLrmstar}{\mathrm{S^*L}}
\newcommand{\PSLrmstar}{\mathrm{PS^*L}}
\newcommand{\GLrm}{\mathrm{GL}}
\newcommand{\Mrm}{\mathrm{M}}
\newcommand{\Isom}{\mathrm{Isom}}
\newcommand{\Mob}{\mathrm{M\ddot{o}b}}
\newcommand{\Ebb}{\mathbb{E}}
\newcommand{\Cscr}{\mathscr{C}}
\newcommand{\pwrm}{\mathrm{pw}}
\newcommand{\clos}[1]{\overline{#1}}
\newcommand{\Hcal}{\mathcal{H}}
\newcommand{\Hbcal}{\bm{\mathcal{H}}}
\newcommand{\Ucal}{\mathcal{U}}
\newcommand{\Ubcal}{\bm{\mathcal{U}}}
\renewcommand{\det}{\mathrm{det}}
\newcommand{\ab}{\mathrm{ab}}


\title{Homework 3: Exact Symplectic Manifolds}
\author{Yannis B\"ahni}
\address[Yannis B\"ahni]{University of Zurich, R\"amistrasse 71, 8006 Zurich}
\email[Yannis B\"ahni]{\href{mailto:yannis.baehni@uzh.ch}{yannis.baehni@uzh.ch}}

\begin{document}

\maketitle
\thispagestyle{fancy}
\setcounter{section}{1}

\begin{exercise}
Let $M$ and $N$ be smooth manifolds, $F : M \to N$ a diffeomorphism and $A \in \Gamma\del[1]{T^{(0,k)}TN}$, $k \in \Zbb$, $k \geq 1$. Then 
\begin{equation}
F^*A(X_1,\dots,X_k) = A(F_*X_1,\dots,F_*X_k) \circ F
\end{equation}
\noindent holds for all $X_1,\dots,X_k \in \Xfrak(M)$.
\label{ex_properties_pullback}
\end{exercise}

\begin{solution}
Let $p \in M$. Then
\begin{align*}
F^*A(X_1,\dots,X_k)(p) &= (F^*A)_p(X_1\vert_p,\dots,X_k\vert_p)\\
&= A_{F(p)}\del[1]{d F_p(X_1\vert_p),\dots,d F_p(X_k\vert_p)}\\
&= A_{F(p)}\del[1]{(F_*X_1)_{F(p)},\dots,(F_*X_k)_{F(p)}}\\
&= A\del[1]{F_*X_1,\dots,F_*X_k}\del[1]{F(p)}.
\end{align*}
\end{solution}

\begin{exercise} 
\begin{enumerate}[label = \textup{(}\alph*\textup{)}]
\item  
\end{enumerate}
\end{exercise}

\begin{solution}
	For (a), consider the tangent-cotangent isomorphism $\wtilde{\omega} : TM \to T^*M$. Set $X := \wtilde{\omega}^{-1}(-\alpha)$. As a composition of smooth functions, $X : M \to TM$ is smooth. Moreover, $X_p = \wtilde{\omega}^{-1}(-\alpha_p) \in T_pM$. Thus $X \in \Xfrak(M)$. Moreover 
	\begin{equation*}
		i_X \omega = \wtilde{\omega}\del[1]{\wtilde{\omega}^{-1}(-\alpha)} = - \alpha.  
	\end{equation*}
	Since $\wtilde{\omega}$ is an isomorphism, $X$ is unique. Cartan's magic formula together with the assumption $\omega = -d\alpha$ yields
	\begin{equation*}
		\Lcal_X \omega = di_X\omega + i_Xd\omega = di_X \omega = -d\alpha = \omega.
	\end{equation*}
	For proving (b), assume that $L_X \omega = \omega$ for some $X \in \Xfrak(M)$. Again, Cartan's magic formula yields
	\begin{equation*}
		\Lcal_X\omega = di_X\omega + i_Xd\omega = di_X\omega = \omega.
	\end{equation*}
	Now $i_X\omega \in \Omega^1(M)$ and thus $\omega$ is exact.\\
	For proving (c), an application of the Fisherman's formula yields 
	\begin{equation*}
		\frac{d}{dt} (\exp tX)^* \omega = (\exp tX)^* \Lcal_X \omega = (\exp tX)^* \omega
	\end{equation*}
	\noindent and the property of the flow $\exp tX$ yields
	\begin{equation*}
		(\exp tX\vert_0)^*\omega = \id_M^* \omega = \omega.
	\end{equation*}
	Also we have that $\frac{d}{dt}e^t\omega = e^t\omega$ and $e^0\omega = \omega$. Hence $(\exp tX)^* \omega$ and $e^t\omega$ solve the same locally uniquely solvable initial value problem and are therefore locally equal.\\
	For proving (d), observe that for $Y \in \Xfrak(M)$ we have that
	\begin{align*}
		\omega(g_*X,Y) \circ g &= (g^*\omega)(X,g_*^{-1}Y)\\
		&= \omega\del[1]{X,g_*^{-1}Y}\\
		&= i_X\omega\del[1]{g_*^{-1}Y}\\
		&= -\alpha\del[1]{g_*^{-1}Y} \circ g^{-1}\\
		&= -(g^*\alpha)\del[1]{g_*^{-1}Y}\\
		&= -\alpha(Y) \circ g\\
		&= i_X\omega(Y) \circ g\\
		&= \omega(X,Y) \circ g.
	\end{align*}
	Thus $\wtilde{\omega}(g_*X) = \wtilde{\omega}(X)$ and since $\wtilde{\omega}$ is an isomorphism, we have that $g_*X = X$.\\
	For proving (e), let us define $\rho_t := g \circ \exp tX \circ g^{-1}$. Then 
	\begin{equation*}
		\rho_0 = g \circ \exp tX \vert_{t = 0} \circ g^{-1} = \id_M
	\end{equation*}
	\noindent and
	\begin{align*}
		\frac{d}{dt} \rho_t(p) &= \frac{d}{dt}g\del[1]{\exp tX\del[1]{g^{-1}(p)}}\\
		&= dg_{\exp tX\del[0]{g^{-1}(p)}}\frac{d}{dt}\exp tX\del[1]{g^{-1}(p)}\\
		&= dg_{\exp tX\del[0]{g^{-1}(p)}}X\del[1]{\exp tX\del[1]{g^{-1}(p)}}\\
		&= (g_*X)_{(g \circ \exp tX \circ g^{-1})(p)}.
	\end{align*}
	Thus $\rho_t$ is the flow of the vector field $g_* X$. By part (d) $g_* X$ coincides with $X$ and thus $\rho_t$ is also the flow of $X$. But flows are unique and thus 
	\begin{equation*}
		g \circ \exp tX \circ g^{-1} = \exp tX
	\end{equation*}
	\noindent from which the claim follows.\\
	For (f), let us compute $\Lcal_X \omega_0$. We have
	\begin{align*}
		\Lcal_X\omega_0 &= di_X\omega_0\\
		&= d \sum_{i = 1}^n \del[1]{(i_X dx_i) \wedge dy_i - dx_i \wedge (i_Xdy_i)}\\
		&= \frac{1}{2}d \sum_{i = 1}^n (x_i dy_i - y_idx_i)\\
		&= \frac{1}{2}\sum_{i = 1}^n (dx_i \wedge dy_i - dy_i \wedge dx_i)\\
		&= \omega_0.
	\end{align*}
\end{solution}

\begin{exercise} 
\begin{enumerate}[label = \textup{(}\alph*\textup{)}]
\item  
\end{enumerate}
\end{exercise}

\begin{solution}
	For showing (a), let us consider $X := f^i \frac{\partial}{\partial x^i} + g^i \frac{\partial}{\partial \xi^i}$. Then
	\begin{align*}
		i_X\omega &= i_X\sum_{i = 1}^n dx_i \wedge d\xi_i\\
		&= \sum_{i = 1}^n \del[1]{(i_Xdx_i) \wedge d\xi_i - dx_i \wedge (i_X d\xi_i)}\\
		&= \sum_{i = 1}^n \del[1]{f^i d\xi_i - g^i dx_i}. 
	\end{align*}
	Comparing this with $-\alpha$ yields $f^i = 0$ and $g^i = \xi^i$ for all $i = 1,\dots,n$. Hence
	\begin{equation*}
		X = \xi^i \frac{\partial}{\partial \xi^i}.
	\end{equation*}
\end{solution}

\appendix
\begin{appendix}
	\section{The Tubular Neighbourhood Theorem}
	\begin{theorem}[Generalization of the Inverse Function Theorem]
		Let $M$ and $N$ be smooth manifolds and $S$ a compact immersed submanifold of $M$. Moreover, let $f : M \to N$ be smooth such that $f\vert_S$ is injective. Suppose that $df_p$ is an isomorphism for all $x \in S$.
	\end{theorem}
\end{appendix}

\end{document}
