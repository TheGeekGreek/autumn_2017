%%%%%%%%%%%%%%%%%%%%%%%%%%%%%%%%%%%%%%%%%%%%%%%%%%%%%%%%%%%%%%%%%%%%%%%%%%
%Author:																 %
%-------																 %
%Yannis Baehni at University of Zurich									 %
%baehni.yannis@uzh.ch													 %
%																		 %
%Version log:															 %
%------------															 %
%06/02/16 . Basic structure												 %
%04/08/16 . Layout changes including section, contents, abstract.		 %
%%%%%%%%%%%%%%%%%%%%%%%%%%%%%%%%%%%%%%%%%%%%%%%%%%%%%%%%%%%%%%%%%%%%%%%%%%

%Page Setup
\documentclass[
	12pt, 
	oneside, 
	a4paper,
	reqno,
	final
]{amsart}

\usepackage[
	left = 3cm, 
	right = 3cm, 
	top = 3cm, 
	bottom = 3cm
]{geometry}

%Headers and footers
\usepackage{fancyhdr}
	\pagestyle{fancy}
	%Clear fields
	\fancyhf{}
	%Header right
	\fancyhead[R]{
		\footnotesize
		Yannis B\"{a}hni\\
		\href{mailto:yannis.baehni@uzh.ch}{yannis.baehni@uzh.ch}
	}
	%Header left
	\fancyhead[L]{
		\footnotesize
		401-3581-67L Symplectic Geometry\\
		Autumn 2017
	}
	%Page numbering in footer
	\fancyfoot[C]{\thepage}
	%Separation line header and footer
	\renewcommand{\headrulewidth}{0.4pt}
	%\renewcommand{\footrulewidth}{0.4pt}
	
	\setlength{\headheight}{19pt} 

%Title
\usepackage[foot]{amsaddr}
\usepackage{upref}
\usepackage{newtxtext}
\usepackage[subscriptcorrection,nofontinfo,mtpcal,mtphrb]{mtpro2}
\usepackage{bm}
\usepackage{xspace}
\makeatletter
\usepackage{etoolbox}
\patchcmd{\abstract}{\scshape\abstractname}{\textbf{\abstractname}}{}{}

\usepackage[all,cmtip]{xy}

%Section, subsection and subsubsection font
%------------------------------------------
\makeatletter
	\renewcommand{\@secnumfont}{\bfseries}
	\renewcommand\section{\@startsection{section}{1}%
  	\z@{.7\linespacing\@plus\linespacing}{.5\linespacing}%
  	{\normalfont\bfseries\centering}}
	\renewcommand\subsection{\@startsection{subsection}{2}%
    	\z@{.5\linespacing\@plus.7\linespacing}{-.5em}%
    	{\normalfont\bfseries}}%
	\renewcommand\subsubsection{\@startsection{subsubsection}{3}%
    	\z@{.5\linespacing\@plus.7\linespacing}{-.5em}%
    	{\normalfont\bfseries}}%
%Formatting title of TOC
\renewcommand{\contentsnamefont}{\bfseries}
%Table of Contents
\setcounter{tocdepth}{3}

% Add bold to \section titles in ToC and remove . after numbers
\renewcommand{\tocsection}[3]{%
  \indentlabel{\@ifnotempty{#2}{\bfseries\ignorespaces#1 #2\quad}}\bfseries#3}
% Remove . after numbers in \subsection
\renewcommand{\tocsubsection}[3]{%
  \indentlabel{\@ifnotempty{#2}{\ignorespaces#1 #2\quad}}#3}
\let\tocsubsubsection\tocsubsection% Update for \subsubsection
%...

\newcommand\@dotsep{4.5}
\def\@tocline#1#2#3#4#5#6#7{\relax
  \ifnum #1>\c@tocdepth % then omit
  \else
    \par \addpenalty\@secpenalty\addvspace{#2}%
    \begingroup \hyphenpenalty\@M
    \@ifempty{#4}{%
      \@tempdima\csname r@tocindent\number#1\endcsname\relax
    }{%
      \@tempdima#4\relax
    }%
    \parindent\z@ \leftskip#3\relax \advance\leftskip\@tempdima\relax
    \rightskip\@pnumwidth plus1em \parfillskip-\@pnumwidth
    #5\leavevmode\hskip-\@tempdima{#6}\nobreak
    \leaders\hbox{$\m@th\mkern \@dotsep mu\hbox{.}\mkern \@dotsep mu$}\hfill
    \nobreak
    \hbox to\@pnumwidth{\@tocpagenum{\ifnum#1=1\bfseries\fi#7}}\par% <-- \bfseries for \section page
    \nobreak
    \endgroup
  \fi}
\AtBeginDocument{%
\expandafter\renewcommand\csname r@tocindent0\endcsname{0pt}
}
\def\l@subsection{\@tocline{2}{0pt}{2.5pc}{5pc}{}}
\def\l@subsubsection{\@tocline{2}{0pt}{4.5pc}{5pc}{}}
\makeatother

\advance\footskip0.4cm
\textheight=54pc    %a4paper
\textheight=50.5pc %letterpaper
\advance\textheight-0.4cm
\calclayout

%Font settings
%\usepackage{anyfontsize}
%Footnote settings
%\usepackage{mathptmx}
\usepackage{footmisc}
%	\renewcommand*{\thefootnote}{\fnsymbol{footnote}}
\usepackage{commath}
%Further math environments
%Further math fonts (loads amsfonts implicitely)
%Redefinition of \text
%\usepackage{amstext}
\usepackage{upref}
%Graphics
%\usepackage{graphicx}
%\usepackage{caption}
%\usepackage{subcaption}
%Frames
\usepackage{mdframed}
\allowdisplaybreaks
%\usepackage{interval}
\newcommand{\toup}{%
  \mathrel{\nonscript\mkern-1.2mu\mkern1.2mu{\uparrow}}%
}
\newcommand{\todown}{%
  \mathrel{\nonscript\mkern-1.2mu\mkern1.2mu{\downarrow}}%
}
\AtBeginDocument{\renewcommand*\d{\mathop{}\!\mathrm{d}}}
\renewcommand{\Re}{\operatorname{Re}}
\renewcommand{\Im}{\operatorname{Im}}
\DeclareMathOperator\Log{Log}
\DeclareMathOperator\Arg{Arg}
\DeclareMathOperator\sech{sech}
\DeclareMathOperator*\esssup{ess.sup}
\DeclareMathOperator\id{id}
\DeclareMathOperator\im{im}
\DeclareMathOperator\Vol{Vol}
\DeclareMathOperator\dist{dist}
%\usepackage{hhline}
%\usepackage{booktabs} 
%\usepackage{array}
%\usepackage{xfrac} 
%\everymath{\displaystyle}
%Enumerate
\usepackage{tikz}
\usetikzlibrary{external}
\tikzexternalize % activate!
\usetikzlibrary{patterns}
\pgfdeclarepatternformonly{adjusted lines}{\pgfqpoint{-1pt}{-1pt}}{\pgfqpoint{40pt}{40pt}}{\pgfqpoint{39pt}{39pt}}%
{
  \pgfsetlinewidth{.8pt}
  \pgfpathmoveto{\pgfqpoint{0pt}{0pt}}
  \pgfpathlineto{\pgfqpoint{39.1pt}{39.1pt}}
  \pgfusepath{stroke}
}
\usepackage{enumitem} 
%\renewcommand{\labelitemi}{$\bullet$}
%\renewcommand{\labelitemii}{$\ast$}
%\renewcommand{\labelitemiii}{$\cdot$}
%\renewcommand{\labelitemiv}{$\circ$}
%Colors
%\usepackage{color}
%\usepackage[cmtip, all]{xy}
%Theorems
\newtheoremstyle{main} 		             	 		%Stylename
  	{}	                                     		%Space above
  	{}	                                    		%Space below
  	{\itshape}			                     		%Body font
  	{}        	                             		%Indent
  	{\bfseries}   	                         		%Head font
  	{.}            	                        		%Head punctuation
  	{ }           	                         		%Head space 
  	{\thmname{#1}\thmnumber{ #2}\thmnote{ (#3)}}	%Head specification
\theoremstyle{main}
\newtheorem{definition}{Definition}[section]
\newtheorem{proposition}{Proposition}[section]
\newtheorem{corollary}{Corollary}[section]
\newtheorem{theorem}{Theorem}[section]
\newtheorem{lemma}{Lemma}[section]
%Roman style theorems
\newtheoremstyle{roman}
	{}
	{}
  	{}
  	{}
	{\bfseries}
	{.}
  	{ }
	{\thmname{#1}\thmnumber{ #2}\thmnote{ (#3)}}
\theoremstyle{roman}
\newtheorem{example}{Example}[section]
\newtheorem{solution}{Solution}[section]
\newtheorem{remark}{Remark}[section]
%Exercise style theorems
\newtheoremstyle{exercise}
  	{}
  	{}
  	{\small}
  	{}
  	{\bfseries}
  	{.}
 	{ }
  	{\thmname{#1}\thmnumber{ #2}\thmnote{ (#3)}}
\theoremstyle{exercise}
\newtheorem{exercise}{Exercise}[section]
%Changing default style of proof environment
\renewcommand*{\proofname}{\itshape Proof}
%German non-ASCII-Characters
%Graphics-Tool
%\usepackage{tikz}
%\usepackage{tikzscale}
%\usepackage{bbm}
%\usepackage{bera}
%Listing-Setup
%Bibliographie
\usepackage[backend=bibtex, style=alphabetic]{biblatex}
%\usepackage[babel, german = swiss]{csquotes}
\bibliography{../latex/bibliography}
%PDF-Linking
%\usepackage[hyphens]{url}
\usepackage[bookmarksopen=true,bookmarksnumbered=true]{hyperref}
%\PassOptionsToPackage{hyphens}{url}\usepackage{hyperref}
\hypersetup{
  colorlinks   = true, %Colours links instead of ugly boxes
  urlcolor     = blue, %Colour for external hyperlinks
  linkcolor    = blue, %Colour of internal links
  citecolor    = blue %Colour of citations
}
%Weierstrass-P symbol for power set
\newcommand{\powerset}{\raisebox{.15\baselineskip}{\Large\ensuremath{\wp}}}
\newcommand{\bld}[1]{\boldmath\textit{\textbf{#1}}\unboldmath}
\usepackage{pict2e}
\makeatletter
\DeclareRobustCommand{\intprod}{%
	\mathbin{\mathpalette\int@prod{(0.1,0)(0.9,0)(0.9,0.8)}}%
}
\newcommand{\int@prod}[2]{%
	\begingroup
	\sbox\z@{$\m@th#1+$}%
	\setlength\unitlength{\wd\z@}%
	\begin{picture}(1,1)
	\roundcap
	\polyline#2
	\end{picture}%
	\endgroup
}
\makeatother
\newcommand{\Sbb}{\mathbb{S}}
\newcommand{\dRrm}{\mathrm{dR}}
\newcommand{\Rbb}{\mathbb{R}}
\newcommand{\Lcal}{\mathcal{L}}
\newcommand{\Zbb}{\mathbb{Z}}
\newcommand{\Nbb}{\mathbb{N}}
\newcommand{\Xfrak}{\mathfrak{X}}


\title{Homework 1: Symplectic Linear Algebra}
\author{Yannis B\"ahni}
\address[Yannis B\"ahni]{University of Zurich, R\"amistrasse 71, 8006 Zurich}
\email[Yannis B\"ahni]{\href{mailto:yannis.baehni@uzh.ch}{yannis.baehni@uzh.ch}}

\begin{document}

\maketitle
\thispagestyle{fancy}
\setcounter{section}{1}

\begin{exercise}
Let $(V,\Omega)$ be a symplectic vector space and $Y,W \subseteq V$ be linear subspaces.
\begin{enumerate}[label = \textup{(}\alph*\textup{)}]
\item $\dim Y + \dim Y^\Omega = \dim V$.
\item $\del[1]{Y^\Omega}^\Omega = Y$.
\item $Y \subseteq W \Leftrightarrow W^\Omega \subseteq Y^\Omega$.
\item $\Omega\vert_{Y}$ nondegenerate $\Leftrightarrow Y \cap Y^\Omega = \cbr[0]{0} \Leftrightarrow V = Y \oplus Y^\Omega$.
\item If $Y \subseteq Y^\Omega$, then $\dim Y \leq \frac{1}{2}\dim V$.
\item If $\dim Y = 1$, then $Y$ is isotropic.
\item If $Y$ is of codimension $1$, then $Y$ is coisotropic. 
\item $Y$ lagrangian $\Leftrightarrow$ $Y$ isotropic and coisotropic $\Leftrightarrow$ $Y = Y^\Omega$.
\end{enumerate}
\end{exercise}

\begin{solution}
	For proving (a), consider the mapping $\Phi : V \to Y^*$ defined by $\Phi(v) := \Omega(v,\cdot)\vert_Y$. Clearly, $\ker \Phi = Y^\Omega$. Let $\varphi \in Y^*$. By exercise B.13 \cite[623]{lee:smooth_manifolds:2013}, there exists an extension $\wtilde{\varphi} \in V^*$ of $\varphi$, i.e. $\wtilde{\varphi}\vert_Y = \varphi$. Since $\wtilde{\Omega}$ is an isomorphism, there exists $v \in V$ such that $\wtilde{\varphi} = \Omega(v,\cdot)$. Which implies $\wtilde{\varphi}\vert_Y = \Omega(v,\cdot)\vert_Y$. Hence we get that $\Phi$ is surjective and thus the rank-nullity law \cite[627]{lee:smooth_manifolds:2013} implies that 
\begin{equation*}
	\dim V = \dim(\Phi(V)) + \dim(\ker\Phi) = \dim Y^* + \dim Y^\Omega = \dim Y + \dim Y^\Omega
\end{equation*}
\noindent since $V$ is finite dimensional.\\
For proving (b), let $v \in Y$. Then for any $u \in Y^\Omega$ we have that $\Omega(v,u) = - \Omega(u,v) = 0$ and thus $Y \subseteq \del[1]{Y^\Omega}^\Omega$. Hence $Y$ is a linear subspace of $\del[1]{Y^\Omega}^\Omega$. Furthermore part (a) yields
\begin{equation*}
\dim Y = \dim V - \dim Y^\Omega = \dim \del[1]{Y^\Omega}^\Omega.
\end{equation*}
\noindent Thus exercise B.4. (b) \cite[620]{lee:smooth_manifolds:2013} implies that $\del[1]{Y^\Omega}^\Omega = Y$.\\
For proving (c), suppose that $Y \subseteq W$ and let $v \in W^\Omega$. Then for any $u \in Y$ we have that $\Omega(v,u) = 0$ and thus $W^\Omega \subseteq Y^\Omega$. Conversly, suppose that $W^\Omega \subseteq Y^\Omega$. By part (b) we can also show that $\del[1]{Y^\Omega}^\Omega \subseteq \del[1]{W^\Omega}^\Omega$ holds. But this is easily seen.\\
For proving (d), we show the two equivalences separately. The first equivalence follows immediately from the observation that
\begin{equation*}
	\ker \widetilde{\Omega\vert_{Y}} = \cbr[0]{v \in Y : \forall u \in Y\del[0]{\Omega\vert_Y(v,u) = 0}} = Y \cap Y^\Omega.
\end{equation*}
For showing the second equivalence, assume that $Y \cap Y^\Omega = \cbr[0]{0}$. Part (a) and exercise B.8 \cite[621]{lee:smooth_manifolds:2013} yield
\begin{equation*}
\dim(Y + Y^\Omega) = \dim Y + \dim Y^\Omega - \dim(Y \cap Y^\Omega) = \dim Y + \dim Y^\Omega = \dim V.
\end{equation*}
Thus again exercise B.4. (b) \cite[620]{lee:smooth_manifolds:2013} implies that $Y + Y^\Omega = V$. Therefore $V = Y \oplus Y^\Omega$. The other implication simply holds by the definition of the direct sum.\\
(e) directly follows from (a) and exercise B.4. \cite[620]{lee:smooth_manifolds:2013} since
\begin{equation*}
\dim V = \dim Y + \dim Y^\Omega \geq 2\dim Y.
\end{equation*}
For proving (f), let $v \in Y \setminus \cbr[0]{0}$. Then every element of $Y$ can be written uniquely as $\lambda v$ for some $\lambda \in \Rbb$. Thus by the alternating property of $\Omega$ we get that
\begin{equation*}
	\Omega(\lambda v,\mu v) = \lambda \mu \Omega(v,v) = 0
\end{equation*}
\noindent for all $\lambda,\mu \in \Rbb$ and so $Y \subseteq Y^\Omega$.\\
For proving (g), we observe that part (a) yields $\dim Y^\Omega = 1$. Thus $Y^\Omega$ is isotropic by part (f) and therefore $Y^\Omega \subseteq (Y^\Omega)^\Omega = Y$ by part (b).\\
For proving (h), we first observe that the second equivalence is trivially true. We show that $Y$ is lagrangian if and only if $Y = Y^\Omega$. Assume that $Y$ is lagrangian. From part (a) immediately follows that $\dim Y = \dim Y^\Omega$. Since $Y \subseteq Y^\Omega$ we get that $Y = Y^\Omega$. Conversly, assume that $Y = Y^\Omega$. Using again part (a) we get that $2\dim Y = \dim V$.
\end{solution}

\begin{exercise}
	Let $(V,\Omega)$ be a symplectic vector space and $E$ be a finite dimenisonal real vector space.
\begin{enumerate}[label = \textup{(}\alph*\textup{)}]
\item $(E \oplus E^*,\Omega_0)$ is a symplectic vector space where 
	\begin{equation}
		\Omega_0(u \oplus \alpha, v \oplus \beta) := \beta(u) - \alpha(v).
	\end{equation}
	Moreover, if $(e_i)$ is a symplectic basis for $E$, then $(e_i \oplus 0, 0 \oplus e_i^*)$ is a symplectic basis for $(E \oplus E^*,\Omega_0)$.
\item If $Y$ is a lagrangian subspace of $V$, then $V$ is symplectomorphic to $(Y \oplus Y^*,\Omega_0)$. 
\item If $Y$ is a lagrangian subspace of $V$, then any basis $(e_i)$ of $Y$ can be extended to a symplectic basis for $V$.
\end{enumerate}
\end{exercise}

\begin{remark}
	We intentionally switched the order of exercises which does feel in our view more natural. Clearly, the exericse was not mentioned to solve that way since we will use a more advanced concept from chapter 12. But we think that this solution adds some more generality to the theory.
\end{remark}

\begin{solution}
	For proving (a), we observe that bilinearity and skew-symmetry is immediate from the definition of $\Omega_0$. Hence we have to show that $\Omega_0$ is nondegenerate. Assume that $u \oplus \alpha \in \ker \Omega_0$. Hence we have that $\beta(u) = \alpha(v)$ for all $v \oplus \beta \in E \oplus E^*$. Assume that $u \neq 0$. Then we find $\beta \in E^*$ such that $\beta(u) \neq 0$. Setting $v = 0$ yields a contradiction and thus $u = 0$. Assume that $\alpha \neq 0$. Hence we find $v \in E$ such that $\alpha(v) \neq 0$. Hence setting $\beta = 0$ again yields a contradiction. Thus $\alpha = 0$ and so $\Omega_0$ is nondegenerate. The symplectic form is canonical in the sense that its definition does not depend on a choice of a basis for $E \oplus E^*$. That $(e_i \oplus 0, 0 \oplus e_i^*)$ is a symplectic basis follows directly from 
	\begin{equation*}
		\Omega_0(e_i \oplus 0, e_j \oplus 0) = 0 = \Omega_0(0 \oplus e_i^*, 0 \oplus e_j^*)
	\end{equation*}
	\noindent and
	\begin{equation*}
		\Omega_0(e_i \oplus 0, 0 \oplus e_j^*) = e_j^*(e_i) = \delta_{ij}.
	\end{equation*}
	To prove (b), let $J$ be a compatible complex structure on $(V,\Omega)$ (the existence is assured by \cite[84]{dasilva:symplectic:2008}). Thus lemma \ref{lem:decomposition} implies that $V = Y \oplus J(Y)$. Define a mapping $\varphi: Y \oplus J(Y) \to Y \oplus Y^*$ by 
\begin{equation*}
\varphi\del[1]{x + J(y)} := x \oplus -\Phi\del[1]{J(y)}
\end{equation*}
\noindent where $\Phi : J(y) \to Y^*$ is the isomorphism constructed in lemma \ref{lem:decomposition}. $\Phi$ is easily seen to be an isomorphism. Moreover
\begin{align*}
(\varphi^*\Omega_0)\del[1]{x + J(y),x' + J(y')} &= \Omega_0\del[1]{\varphi(x + J(y)), \varphi(x' + J(y'))}\\
&= \Omega_0\del[1]{x \oplus -\Omega\del[1]{J(y),\cdot}\big\vert_Y, x' \oplus -\Omega\del[1]{J(y'),\cdot}\big\vert_Y}\\
&= -\Omega\del[1]{J(y'),x} + \Omega\del[1]{J(y),x'}\\
&= \Omega\del[1]{x + J(y),x' + J(y')}.
\end{align*}
Hence $\varphi$ is a symplectomorphism.\\
To prove (c), we observe that by part (a) and (b) and lemma \ref{lem:symplectic_basis}, $\del[1]{\varphi^{-1}(e_i \oplus 0), \varphi^{-1}(0 \oplus e_i^*)}$ is a symplectic basis for $V$, but $\varphi^{-1}\del[1]{u \oplus \alpha} = u - \Phi^{-1}(\alpha)$ and thus $(e_i)$ is part of the symplectic basis.
\end{solution}

\begin{exercise}
	Let $V$ be a finite dimensional real vector space.
\begin{enumerate}[label = \textup{(}\alph*\textup{)}]
	\item Any $\Omega \in \Lambda^2(V^*)$ is of the form
		\begin{equation}
			\Omega = \sum_{i = 1}^n e_i^* \wedge f_i^*
		\end{equation}
		\noindent where $(u_i,e_i,f_i)$ is a basis for $V$ provided by the standard form theorem \cite[3]{dasilva:symplectic:2008}.
	\item Assume $\dim V = 2n$ and $\Omega \in \Lambda^2(V^*)$. Then $\Omega$ is symplectic if and only if $\Omega^n \neq 0$.
\end{enumerate}
\end{exercise}

\begin{solution}
	For proving (a), we have that if $(v_i)$ is any basis for $V$, then
	\begin{equation*}
		\Omega = \sum_{i < j} \Omega(v_i,v_j) v_i^* \wedge v_j^*
	\end{equation*}
	\noindent by \cite[353]{lee:smooth_manifolds:2013}. Thus the statement directly follows from the standard form theorem.\\
	For proving (b), let $(e_i,f_i)$ be a symplectic basis for $V$. By part (a) we have that 
	\begin{equation*}
		\Omega = \sum_{i = 1}^n e_i^* \wedge f_i^*.
	\end{equation*}
	We claim that
	\begin{equation}
		\PARENS{\sum_{i = 1}^n e_i^* \wedge f_i^*}^n = n! \PARENS{e_1^* \wedge f_1^* \wedge \dots \wedge e_n^* \wedge f_n^*}.
		\label{eq:power}
	\end{equation}
	We do a proof by induction over $n$. The formula trivially holds for $n = 1$. So assume that it holds for some $n \geq 1$. The binomial theorem yields
	\begin{align*}
		\PARENS{\sum_{i = 1}^{n + 1} e_i^* \wedge f_i^*}^{n + 1} =& \PARENS{\sum_{i = 1}^{n + 1} e_i^* \wedge f_i^*}^n \wedge \PARENS{\sum_{i = 1}^{n + 1} e_i^* \wedge f_i^*}\\
		=& \PARENS{\sum_{i = 1}^n e_i^* \wedge f_i^* + e_{n + 1}^* \wedge f_{n + 1}^*}^n \wedge \PARENS{\sum_{i = 1}^{n + 1} e_i^* \wedge f_i^*}\\
		=& \PARENS{\sum_{k = 0}^n {n \choose k}  \PARENS{\sum_{i = 1}^n e_i^* \wedge f_i^*}^k \wedge \PARENS{ e_{n + 1}^* \wedge f_{n + 1}^*}^{n - k}} \wedge \PARENS{\sum_{i = 1}^{n + 1} e_i^* \wedge f_i^*}\\
		=& \PARENS{\sum_{i = 1}^n e_i^* \wedge f_i^*}^n \wedge \PARENS{\sum_{i = 1}^{n + 1} e_i^* \wedge f_i^*}\\
		& + n\PARENS{\sum_{i = 1}^n e_i^* \wedge f_i^*}^{n - 1} \wedge \PARENS{ e_{n + 1}^* \wedge f_{n + 1}^*} \wedge \PARENS{\sum_{i = 1}^{n + 1} e_i^* \wedge f_i^*}\\
		=& n! \PARENS{e_1^* \wedge f_1^* \wedge \dots \wedge e_{n + 1}^* \wedge f_{n + 1}^*}\\
		&+ n\PARENS{\sum_{i = 1}^n e_i^* \wedge f_i^*}^n \wedge \PARENS{ e_{n + 1}^* \wedge f_{n + 1}^*}\\
		=&  (n + 1)! \PARENS{e_1^* \wedge f_1^* \wedge \dots \wedge e_{n + 1}^* \wedge f_{n + 1}^*}.
	\end{align*}
	The use of the binomial theorem is justified since elements of $\Lambda^2(V^*)$ commute under the wedge product by \cite[356]{lee:smooth_manifolds:2013}. Hence
	\begin{equation*}
		\Omega^n = n! \PARENS{e_1^* \wedge f_1^* \wedge \dots \wedge e_n^* \wedge f_n^*} \neq 0.
	\end{equation*}
	Conversly, assume that $\Omega^n \neq 0$. Assume that $\Omega$ is degenerate. Hence there exists a basis
	\begin{equation*}
		u_1,\dots,u_k,e_1,\dots,e_m,f_1,\dots,f_m
	\end{equation*}
	\noindent of $f$ such that $m < n$. But then part (a) together with (\ref{eq:power}) yield
	\begin{equation*}
		\Omega^n = m! \PARENS{e_1^* \wedge f_1^* \wedge \dots \wedge e_m^* \wedge f_m^*} \wedge \PARENS{\sum_{i = 1}^m e_i^* \wedge f_i^*}^{n - m} = 0.
	\end{equation*}
\end{solution}

\appendix
\begin{appendix}
	\section{Lagrangian Subspaces, Compatible Structures and Symplectic Bases}
	The next lemma is adapted from exercise 3. (a), homework 8 \cite[88]{dasilva:symplectic:2008}.
	\begin{lemma}
		Let $(V,\Omega)$ be a symplectic vector space, $J$ be a compatible complex structure on $V$ and $Y$ a lagrangian subspace of $V$. Then $J(Y)$ is a lagrangian subspace, $V = Y \oplus J(Y)$ and $J(Y) \cong Y^*$.
		\label{lem:decomposition}
	\end{lemma}

	\begin{proof}
		Clearly $\dim J(Y) = \dim Y = \frac{1}{2}\dim V$ since $J$ is invertible and $J(Y) \subseteq J(Y)^\Omega$ since $\Omega(J\cdot,J\cdot) = \Omega(\cdot,\cdot)$ and $Y = Y^\Omega$. Hence $J(Y)$ is lagrangian and thus 
\begin{equation*}
V = Y \oplus Y^\perp = Y \oplus J(Y)^\Omega = Y \oplus J(Y)
\end{equation*}
\noindent by exercise B.45. \cite[637]{lee:smooth_manifolds:2013}. Consider $\Phi : J(Y) \to Y^*$ defined by 
\begin{equation*}
\Phi\del[1]{J(y)} := \Omega\del[1]{J(y),\cdot}\big\vert_Y. 
\end{equation*}
Let $J(y) \in \ker \Phi$. Then $\Omega\del[1]{J(y),w} = 0$ for all $w \in Y$. Especially $\Omega\del[1]{J(y),y} = 0$. But this is only possible if $y = 0$. Thus $\Phi$ is injective and due to dimensional reasons surjective, hence an isomorphism. 
	\end{proof}

	\begin{lemma}
		Let $(V,\Omega)$ and $(V',\Omega')$ be symplectic vector spaces and $\varphi : V \to V'$ a symplectomorphism. If $(e_i,f_i)$ is a symplectic basis for $V$, then $\del[1]{\varphi(e_i),\varphi(f_i)}$ is a symplectic basis for $V'$.
		\label{lem:symplectic_basis}
	\end{lemma}

	\begin{proof}
		We have that
\begin{align*}
	&\Omega'\PARENS{\varphi(e_i),\varphi(e_j)} = (\varphi^*\Omega')(e_i,e_j) = \Omega(e_i,e_j) = 0\\
&\Omega'\PARENS{\varphi(f_i),\varphi(f_j)} = (\varphi^*\Omega')(f_i,f_j) = \Omega(f_i,f_j) = 0\\
&\Omega'\PARENS{\varphi(e_i),\varphi(f_j)} = (\varphi^*\Omega')(e_i,f_j) = \Omega(e_i,f_j) = \delta_{ij}.
\end{align*}
	\end{proof}
\end{appendix}

\printbibliography
\end{document}
