%%%%%%%%%%%%%%%%%%%%%%%%%%%%%%%%%%%%%%%%%%%%%%%%%%%%%%%%%%%%%%%%%%%%%%%%%%
%Author:																 %
%-------																 %
%Yannis Baehni at University of Zurich									 %
%baehni.yannis@uzh.ch													 %
%																		 %
%Version log:															 %
%------------															 %
%06/02/16 . Basic structure												 %
%04/08/16 . Layout changes including section, contents, abstract.		 %
%%%%%%%%%%%%%%%%%%%%%%%%%%%%%%%%%%%%%%%%%%%%%%%%%%%%%%%%%%%%%%%%%%%%%%%%%%

%Page Setup
\documentclass[
	12pt, 
	oneside, 
	a4paper,
	reqno,
	final
]{amsart}

\usepackage[
	left = 3cm, 
	right = 3cm, 
	top = 3cm, 
	bottom = 3cm
]{geometry}

%Headers and footers
\usepackage{fancyhdr}
	\pagestyle{fancy}
	%Clear fields
	\fancyhf{}
	%Header right
	\fancyhead[R]{
		\footnotesize
		Yannis B\"{a}hni\\
		\href{mailto:yannis.baehni@uzh.ch}{yannis.baehni@uzh.ch}
	}
	%Header left
	\fancyhead[L]{
		\footnotesize
		401-3581-67L Symplectic Geometry\\
		Autumn 2017
	}
	%Page numbering in footer
	\fancyfoot[C]{\thepage}
	%Separation line header and footer
	\renewcommand{\headrulewidth}{0.4pt}
	%\renewcommand{\footrulewidth}{0.4pt}
	
	\setlength{\headheight}{19pt} 

%Title
\usepackage[foot]{amsaddr}
\usepackage{upref}
\usepackage{newtxtext}
\usepackage[subscriptcorrection,nofontinfo,mtpcal,mtphrb]{mtpro2}
\usepackage{bm}
\usepackage{xspace}
\makeatletter
\usepackage{etoolbox}
\patchcmd{\abstract}{\scshape\abstractname}{\textbf{\abstractname}}{}{}

\usepackage[all,cmtip]{xy}

%Section, subsection and subsubsection font
%------------------------------------------
\makeatletter
	\renewcommand{\@secnumfont}{\bfseries}
	\renewcommand\section{\@startsection{section}{1}%
  	\z@{.7\linespacing\@plus\linespacing}{.5\linespacing}%
  	{\normalfont\bfseries\centering}}
	\renewcommand\subsection{\@startsection{subsection}{2}%
    	\z@{.5\linespacing\@plus.7\linespacing}{-.5em}%
    	{\normalfont\bfseries}}%
	\renewcommand\subsubsection{\@startsection{subsubsection}{3}%
    	\z@{.5\linespacing\@plus.7\linespacing}{-.5em}%
    	{\normalfont\bfseries}}%
%Formatting title of TOC
\renewcommand{\contentsnamefont}{\bfseries}
%Table of Contents
\setcounter{tocdepth}{3}

% Add bold to \section titles in ToC and remove . after numbers
\renewcommand{\tocsection}[3]{%
  \indentlabel{\@ifnotempty{#2}{\bfseries\ignorespaces#1 #2\quad}}\bfseries#3}
% Remove . after numbers in \subsection
\renewcommand{\tocsubsection}[3]{%
  \indentlabel{\@ifnotempty{#2}{\ignorespaces#1 #2\quad}}#3}
\let\tocsubsubsection\tocsubsection% Update for \subsubsection
%...

\newcommand\@dotsep{4.5}
\def\@tocline#1#2#3#4#5#6#7{\relax
  \ifnum #1>\c@tocdepth % then omit
  \else
    \par \addpenalty\@secpenalty\addvspace{#2}%
    \begingroup \hyphenpenalty\@M
    \@ifempty{#4}{%
      \@tempdima\csname r@tocindent\number#1\endcsname\relax
    }{%
      \@tempdima#4\relax
    }%
    \parindent\z@ \leftskip#3\relax \advance\leftskip\@tempdima\relax
    \rightskip\@pnumwidth plus1em \parfillskip-\@pnumwidth
    #5\leavevmode\hskip-\@tempdima{#6}\nobreak
    \leaders\hbox{$\m@th\mkern \@dotsep mu\hbox{.}\mkern \@dotsep mu$}\hfill
    \nobreak
    \hbox to\@pnumwidth{\@tocpagenum{\ifnum#1=1\bfseries\fi#7}}\par% <-- \bfseries for \section page
    \nobreak
    \endgroup
  \fi}
\AtBeginDocument{%
\expandafter\renewcommand\csname r@tocindent0\endcsname{0pt}
}
\def\l@subsection{\@tocline{2}{0pt}{2.5pc}{5pc}{}}
\def\l@subsubsection{\@tocline{2}{0pt}{4.5pc}{5pc}{}}
\makeatother

\advance\footskip0.4cm
\textheight=54pc    %a4paper
\textheight=50.5pc %letterpaper
\advance\textheight-0.4cm
\calclayout

%Font settings
%\usepackage{anyfontsize}
%Footnote settings
%\usepackage{mathptmx}
\usepackage{footmisc}
%	\renewcommand*{\thefootnote}{\fnsymbol{footnote}}
\usepackage{commath}
%Further math environments
%Further math fonts (loads amsfonts implicitely)
%Redefinition of \text
%\usepackage{amstext}
\usepackage{upref}
%Graphics
%\usepackage{graphicx}
%\usepackage{caption}
%\usepackage{subcaption}
%Frames
\usepackage{mdframed}
\allowdisplaybreaks
%\usepackage{interval}
\newcommand{\toup}{%
  \mathrel{\nonscript\mkern-1.2mu\mkern1.2mu{\uparrow}}%
}
\newcommand{\todown}{%
  \mathrel{\nonscript\mkern-1.2mu\mkern1.2mu{\downarrow}}%
}
\AtBeginDocument{\renewcommand*\d{\mathop{}\!\mathrm{d}}}
\renewcommand{\Re}{\operatorname{Re}}
\renewcommand{\Im}{\operatorname{Im}}
\DeclareMathOperator\Log{Log}
\DeclareMathOperator\Arg{Arg}
\DeclareMathOperator\sech{sech}
\DeclareMathOperator*\esssup{ess.sup}
\DeclareMathOperator\id{id}
\DeclareMathOperator\im{im}
\DeclareMathOperator\Vol{Vol}
\DeclareMathOperator\dist{dist}
%\usepackage{hhline}
%\usepackage{booktabs} 
%\usepackage{array}
%\usepackage{xfrac} 
%\everymath{\displaystyle}
%Enumerate
\usepackage{tikz}
\usetikzlibrary{external}
\tikzexternalize % activate!
\usetikzlibrary{patterns}
\pgfdeclarepatternformonly{adjusted lines}{\pgfqpoint{-1pt}{-1pt}}{\pgfqpoint{40pt}{40pt}}{\pgfqpoint{39pt}{39pt}}%
{
  \pgfsetlinewidth{.8pt}
  \pgfpathmoveto{\pgfqpoint{0pt}{0pt}}
  \pgfpathlineto{\pgfqpoint{39.1pt}{39.1pt}}
  \pgfusepath{stroke}
}
\usepackage{enumitem} 
%\renewcommand{\labelitemi}{$\bullet$}
%\renewcommand{\labelitemii}{$\ast$}
%\renewcommand{\labelitemiii}{$\cdot$}
%\renewcommand{\labelitemiv}{$\circ$}
%Colors
%\usepackage{color}
%\usepackage[cmtip, all]{xy}
%Theorems
\newtheoremstyle{main} 		             	 		%Stylename
  	{}	                                     		%Space above
  	{}	                                    		%Space below
  	{\itshape}			                     		%Body font
  	{}        	                             		%Indent
  	{\bfseries}   	                         		%Head font
  	{.}            	                        		%Head punctuation
  	{ }           	                         		%Head space 
  	{\thmname{#1}\thmnumber{ #2}\thmnote{ (#3)}}	%Head specification
\theoremstyle{main}
\newtheorem{definition}{Definition}[section]
\newtheorem{proposition}{Proposition}[section]
\newtheorem{corollary}{Corollary}[section]
\newtheorem{theorem}{Theorem}[section]
\newtheorem{lemma}{Lemma}[section]
%Roman style theorems
\newtheoremstyle{roman}
	{}
	{}
  	{}
  	{}
	{\bfseries}
	{.}
  	{ }
	{\thmname{#1}\thmnumber{ #2}\thmnote{ (#3)}}
\theoremstyle{roman}
\newtheorem{example}{Example}[section]
\newtheorem{solution}{Solution}[section]
\newtheorem{remark}{Remark}[section]
%Exercise style theorems
\newtheoremstyle{exercise}
  	{}
  	{}
  	{\small}
  	{}
  	{\bfseries}
  	{.}
 	{ }
  	{\thmname{#1}\thmnumber{ #2}\thmnote{ (#3)}}
\theoremstyle{exercise}
\newtheorem{exercise}{Exercise}[section]
%Changing default style of proof environment
\renewcommand*{\proofname}{\itshape Proof}
%German non-ASCII-Characters
%Graphics-Tool
%\usepackage{tikz}
%\usepackage{tikzscale}
%\usepackage{bbm}
%\usepackage{bera}
%Listing-Setup
%Bibliographie
\usepackage[backend=bibtex, style=alphabetic]{biblatex}
%\usepackage[babel, german = swiss]{csquotes}
\bibliography{../latex/bibliography}
%PDF-Linking
%\usepackage[hyphens]{url}
\usepackage[bookmarksopen=true,bookmarksnumbered=true]{hyperref}
%\PassOptionsToPackage{hyphens}{url}\usepackage{hyperref}
\hypersetup{
  colorlinks   = true, %Colours links instead of ugly boxes
  urlcolor     = blue, %Colour for external hyperlinks
  linkcolor    = blue, %Colour of internal links
  citecolor    = blue %Colour of citations
}
%Weierstrass-P symbol for power set
\newcommand{\powerset}{\raisebox{.15\baselineskip}{\Large\ensuremath{\wp}}}
\newcommand{\bld}[1]{\boldmath\textit{\textbf{#1}}\unboldmath}
\usepackage{pict2e}
\makeatletter
\DeclareRobustCommand{\intprod}{%
	\mathbin{\mathpalette\int@prod{(0.1,0)(0.9,0)(0.9,0.8)}}%
}
\newcommand{\int@prod}[2]{%
	\begingroup
	\sbox\z@{$\m@th#1+$}%
	\setlength\unitlength{\wd\z@}%
	\begin{picture}(1,1)
	\roundcap
	\polyline#2
	\end{picture}%
	\endgroup
}
\makeatother
\newcommand{\Sbb}{\mathbb{S}}
\newcommand{\dRrm}{\mathrm{dR}}
\newcommand{\Rbb}{\mathbb{R}}
\newcommand{\Lcal}{\mathcal{L}}
\newcommand{\Zbb}{\mathbb{Z}}
\newcommand{\Nbb}{\mathbb{N}}
\newcommand{\Xfrak}{\mathfrak{X}}


\title{Additive and Abelian Categories}
\author{Yannis B\"{a}hni}
\address[Yannis B\"{a}hni]{University of Zurich, R\"{a}mistrasse 71, 8006 Zurich}
\email[Yannis B\"{a}hni]{\href{mailto:yannis.baehni@uzh.ch}{\nolinkurl{yannis.baehni@uzh.ch}}}

\begin{document}

\begin{abstract}
	We define preadditive, additive and abelian categories, where the latter is the natural generalization of $\mathsf{AbGrp}$ to study the basic results of homological algebra in. Moreover, we state and comment on two foundational results in the theory of abelian categories, namely the \emph{Mitchell Embedding Theorem} and the \emph{Eilenberg-Watts Theorem}, which roughly speaking, connect the category of modules with abelian categories.
\end{abstract}

\maketitle

\tableofcontents

\section{Introduction}
Given a diagram
\begin{equation*}
	\begin{tikzcd}
		A \arrow[r,"f"] & B \arrow[r,"g"] & C
	\end{tikzcd}
\end{equation*}
\noindent in $\mathsf{AbGrp}$, we say that the diagram is \emph{exact at $B$}, if $\im f = \ker g$. Recall that the \emph{cokernel of $f$} is defined to be $\coker f := B/\im f$. Consider the following commutative diagram with exact rows in $\mathsf{AbGrp}$:
\begin{equation*}
	\begin{tikzcd}
		& \bullet \arrow[r]\arrow[d,"f"] & \bullet \arrow[r]\arrow[d,"g"] & \bullet \arrow[r]\arrow[d,"h"] & 0\\
		0 \arrow[r] & \bullet \arrow[r] & \bullet \arrow[r] & \bullet
	\end{tikzcd}
\end{equation*}
Those who are familiar with algebraic topology recognise it as the setting of the \emph{snake lemma}. This basic result in \emph{homological algebra} yields the existence of a morphism of groups $\delta \in \mathsf{AbGrp}(\ker h,\coker f)$ such that the sequence
\begin{equation*}
	\begin{tikzcd}
			\ker f \arrow[r] & \ker g \arrow[r] & \ker h \arrow[r,"\delta"] & \coker f \arrow[r] & \coker g \arrow[r] & \coker h
	 \end{tikzcd}
\end{equation*}
\noindent is exact. The basic proof technique used to establish this result is called \emph{diagram chasing}. It turns out, that abelian categories are the right generalization for this type of proof.

\section{Preadditive Catgeories}
Let $G,H \in \ob(\mathsf{AbGrp})$ and $\varphi,\psi \in \mathsf{AbGrp}(G,H)$. Define $\varphi + \psi$ pointwise. Since $H$ is abelian, it follows that $\varphi + \psi \in \mathsf{AbGrp}(G,H)$. Moreover, it is easy to check, that with this operation defined above, $\mathsf{AbGrp}(G,H)$ is an abelian group and 
\begin{equation*}
	\circ : \mathsf{AbGrp}(H,K) \times \mathsf{AbGrp}(G,H) \to \mathsf{AbGrp}(G,K)
\end{equation*}
\noindent is bilinear for each $K \in \ob(\mathsf{AbGrp})$. This motivates the following definition. 

\begin{definition}[Preadditive Category \cite{maclane:categories:1978}]
	A \bld{preadditive category} is a locally small category $\mathcal{C}$ in which all hom-sets $\mathcal{C}(X,Y)$ can be equipped with the structure of an abelian group and composition is bilinear, i.e. for all mophisms $f,f' : X \to Y$ and $g,g' : Y \to Z$ in $\mathcal{C}$ we have that
	\begin{equation}
		(g + g') \circ (f + f') = g \circ f + g \circ f' + g' \circ f + g' \circ f'.
	\end{equation}
\end{definition}

\begin{remark}
	In a preadditive category $\mathcal{C}$, we have that $\mathcal{C}(X,Y) \neq \varnothing$ for all $X,Y \in \ob(\mathcal{C})$.
\end{remark}

\begin{lemma}
	\label{lem:composition_zero}
	Let $\mathcal{C}$ be a preadditive category. Then compositions with the zero elements are again zero elements of the correpsonding abelian groups.	
\end{lemma}

\begin{proof}
	This simply follows from $0 \circ f = (0 + 0) \circ f = 0 \circ f + 0 \circ f$. The other case is similar.
\end{proof}

\section{Additive Categories}
As in $\mathsf{Grp}$, the trivial group in $\mathsf{AbGrp}$ is both an initial and a terminal object. Objects with this property have a special name.

\begin{definition}[Null Object \cite{maclane:categories:1978}]
	Let $\mathcal{C}$ be a category. A \bld{null object in $\mathcal{C}$} is a an object of $\mathcal{C}$ which is both initial and terminal.
\end{definition}

\begin{definition}[Zero Arrow \cite{maclane:categories:1978}]
	Let $\mathcal{C}$ be a category with a null object $0$. For $X,Y \in \ob(\mathcal{C})$, the unique composition $X \to 0 \to Y$ is called the \bld{zero arrow from $X$ to $Y$}, denoted by $0 : X \to Y$.
\end{definition}

\begin{lemma}
	\label{lem:zero_arrow}
	Let $\mathcal{C}$ be a preadditive category with null object and $X,Y \in \ob(\mathcal{C})$. Then the zero arrow $0 : X \to Y$ is the zero element of the group $\mathcal{C}(X,Y)$.
\end{lemma}

\begin{proof}
	The zero arrow $0 : X \to Y$ is the unique composition
	\begin{equation*}
		\begin{tikzcd}
			X \arrow[r] & 0 \arrow[r] & Y.
		\end{tikzcd}
	\end{equation*}
	However, since $0$ is a null object, we have that the two morphisms are the two zero objects in the corresponding abelian group structures. Hence lemma \ref{lem:composition_zero} yields the result.
\end{proof}

Let $A,B \in \ob(\mathsf{AbGrp})$. Then we have seen that $A \coprod B \cong A \prod B$. This can be generalized to preadditive categories.

\begin{proposition}
	\label{prop:products_and_coproducts_coincide}
	Let $\mathcal{C}$ be a preadditive category admitting all finite coproducts. Then $\mathcal{C}$ admits all finite products which coincide with the finite coproducts. In particular, $\mathcal{C}$ has a null object.
\end{proposition}

\begin{proof}
	\begin{enumerate}[label = \textit{Step \arabic*:},wide = 0pt] 
		\item \textit{Zero-ary case \textup{\cite[194]{maclane:categories:1978}}.} Since $\mathcal{C}$ has the empty coproduct, $\mathcal{C}$ has an initial object $\varnothing$. Since $\varnothing$ is initial, there exists a unique map $\varnothing \to \varnothing$, namely $\id_\varnothing$. But $\mathcal{C}(\varnothing,\varnothing)$ is a group and thus $\id_\varnothing = 0$. Hence for any morphism $f : X \to \varnothing$, lemma \ref{lem:composition_zero} yields $f = \id_\varnothing \circ f = 0 \circ f = 0$.
		\item \textit{Binary case.} By the zero-ary case we know that $\mathcal{C}$ admits a null object $0$. Let $X,Y \in \ob(\mathcal{C})$. We want to show that $X \coprod Y$ is also a product of $X$ and $Y$. By the universal property of the coproduct we have a commutative diagram
			\begin{equation*}
				\begin{tikzcd}[column sep=10ex,row sep=5ex]
					& X\\
					X \arrow[ur,"\id_X"]\arrow[dr,"0"']\arrow[r,"\iota_X"] & X \coprod Y \arrow[u,"{(\id_X,0)}"']\arrow[d,"{(0,\id_Y)}"] & Y \arrow[ul,"0"']\arrow[dl,"\id_Y"]\arrow[l,"\iota_Y"']\\
					& Y.
				\end{tikzcd}
			\end{equation*}
			Suppose $(Z,p_X,p_Y)$ is another product cone. Define $f : Z \to X \coprod Y$ by
			\begin{equation*}
				f := \iota_X \circ p_X + \iota_Y \circ p_Y.
			\end{equation*}
			Using lemma \ref{lem:zero_arrow} and \ref{lem:composition_zero}, we compute
			\begin{align*}
				(\id_X,0) \circ f &= (\id_X,0) \circ \iota_X \circ p_X + (\id_X,0) \circ \iota_Y \circ p_Y\\
				&= \id_X \circ p_X + 0 \circ p_Y\\
				&= p_X + 0\\
				&= p_X,
			\end{align*}
			\noindent and simiarly $(0,\id_Y) \circ f = p_Y$. Now we have to check uniqueness. This is the hardest part of the proof and involves the \emph{Yoneda embedding} $\mathcal{\altY} : \mathcal{C} \hookrightarrow \sbr[0]{\mathcal{C}^{\mathrm{op}},\mathsf{Set}}$. We want to show that
			\begin{equation*}
				\iota_X \circ (\id_X,0) + \iota_Y \circ (0,\id_Y) = \id_{X \coprod Y}.
			\end{equation*}
			Applying the Yoneda embedding to the category $\mathcal{C}^\mathrm{op}$, we get that it is enough to show that
			\begin{equation*}
				f \circ \iota_X \circ (\id_X,0) + \iota_Y \circ (0,\id_Y) = f \circ \id_{X \coprod Y}
			\end{equation*}
			\noindent holds for all morphisms $f \in \mathcal{C}(X\coprod Y,C)$. Let $(\alpha,\beta) : X \coprod Y \to C$ be any morphism (where $(\alpha,\beta) \circ \iota_X) = \alpha$ and $(\alpha,\beta) \circ \iota_Y = \beta$). Using the universal property of the corpoduct it is easy to show that
			\begin{equation*}
				(\alpha,\beta) \circ \iota_X \circ (\id_X,0) = (\alpha,0) \quad \text{and} \quad (\alpha,\beta) \circ \iota_Y \circ (0,\id_Y) = (0,\beta),
			\end{equation*} 
			\noindent and moreover one can show that for any other morphism $(\alpha',\beta') : X \coprod Y \to C$ we have
			\begin{equation*}
				(\alpha,\beta) + (\alpha',\beta') = (\alpha + \alpha',\beta + \beta').
			\end{equation*}
			Thus
			\begin{equation*}
				(\alpha,\beta) \circ \id_{X \coprod Y} = (\alpha,\beta) = (\alpha,0) + (0,\beta) =(\alpha,\beta) \circ \del[1]{\iota_X \circ (\id_X,0) + \iota_Y \circ (0,\id_Y)}.
			\end{equation*}
			Now if $f' : Z \to X \coprod Y$ is another morphism making the diagram commute, we have that
			\begin{align*}
				f - f' &= \id_{X \coprod Y} \circ (f - f')\\
				&= \del[1]{\iota_X \circ (\id_X,0) + \iota_Y \circ (0,\id_Y)} \circ (f - f')\\
				&= \iota_X \circ (p_X - p_X) + \iota_Y \circ (p_Y - p_Y)\\
				&= \iota_X \circ 0 + \iota_Y \circ 0\\
				&= 0,
			\end{align*}
			\noindent by lemma \ref{lem:composition_zero}.
		\item \textit{$n$-ary case.} Induction over $n \in \omega$.
	\end{enumerate}
\end{proof}

\begin{definition}[Additive Category]
	An \bld{additive category} is a preadditive category which admits all finite coproducts.	
\end{definition}

\begin{remark}
	Let $\mathcal{C}$ be an additive category. Then by proposition \ref{prop:products_and_coproducts_coincide}, $\mathcal{C}$ admits all finite products which coincide with the coproducts.
\end{remark}

\section{Abelian Categories}

\begin{definition}[Kernel and Cokernel \cite{maclane:categories:1978}]
	Let $\mathcal{C}$ be a category with a null object $0$. A \bld{kernel of a morphism $f : X \to Y$} is defined to be an equalizer of
	\begin{equation*}
		\begin{tikzcd}
			X \arrow[r, shift left, "f"]\arrow[r,shift right,"0"'] & Y.
		\end{tikzcd}
	\end{equation*}
	Dually, a \bld{cokernel of a morphism $f : X \to Y$} is a coequalizer of the above diagram. 
\end{definition}

\begin{lemma}
	In $\mathsf{Grp}$, every monic is a kernel and every epic is a cokernel.
\end{lemma}

\begin{proof}
	Let $m : G \to H$ be a monic in $\mathsf{Grp}$. Consider the fork
	\begin{equation*}
		\begin{tikzcd}
			G \arrow[r,"m"] & H \arrow[r,shift left,"\pi"]\arrow[r,shift right,"0"'] & \coker m.
		\end{tikzcd}
	\end{equation*}
	\noindent Then one can check that this is in fact a universal fork. Similarly, one can check that
	\begin{equation*}
		\begin{tikzcd}
			\ker e \arrow[r,shift left,"\iota"]\arrow[r,shift right,"0"'] & G \arrow[r,"e"] & H
		\end{tikzcd}
	\end{equation*}
	\noindent is a universal cofork for any epic $e : G \to H$ in $\mathsf{Grp}$.
\end{proof}

\begin{definition}[Abelian Category \cite{maclane:categories:1978}]
	An \bld{abelian category} is an additive category satisfying the following additional conditions:
	\begin{enumerate}[label = \textup{(}\alph*\textup{)}, wide = 0pt]
		\item Every morphism admits a kernel.
		\item Every morphism admits a cokernel.
		\item Every monic is a kernel.
		\item Every epic is a cokernel.
	\end{enumerate}
\end{definition}

\begin{examples}
	$\mathsf{AbGrp}$, $\mathsf{Vect}_K$, $_{R}\mathsf{Mod}$ and $\mathsf{Mod}_R$.	
\end{examples}

\section{Exact Sequences}
\begin{lemma}
	\label{lem:exactness_at_B}
	Given a diagram
	\begin{equation*}
		\begin{tikzcd}
			A \arrow[r,"f"] & B \arrow[r,"g"] & C
		\end{tikzcd}
	\end{equation*}
	\noindent in $\mathsf{AbGrp}$, we have that above sequence is exact at $B$ if and only if $g \circ f = 0$ and
	\begin{equation*}
		\begin{tikzcd}
			\ker g \arrow[r,hook,"\iota"] & B \arrow[r,"\pi"] & \coker f 
		\end{tikzcd} = 0.
	\end{equation*}
\end{lemma}

\begin{proof}
	Trivial.
\end{proof}

In lemma \ref{lem:exactness_at_B}, the second condition involves statements about the kernel and the cokernel in the categorical sense. Indeed, in $\mathsf{Grp}$ we have that 
\begin{equation*}
	\begin{tikzcd}
		\ker g \arrow[r,hook,"\iota"] & B \arrow[r, shift left, "g"]\arrow[r,shift right,"0"'] & C
	\end{tikzcd}
	\qquad \text{and} \qquad
	\begin{tikzcd}
		A \arrow[r, shift left, "f"]\arrow[r,shift right,"0"'] & B \arrow[r,"\pi"] & \coker f
	\end{tikzcd}
\end{equation*}
\noindent are an equalizer and a coequalizer, respectively. Hence
\begin{equation*}
	\ker g = \begin{tikzcd}
		\ker g \arrow[r,hook,"\iota"] & B
	\end{tikzcd} \qquad \text{and} \qquad \coker f = \begin{tikzcd}
		B \arrow[r,"\pi"] & \coker f.
	\end{tikzcd}
\end{equation*}

\begin{definition}[Exactness]
	Let $\mathcal{C}$ be an abelian category. A sequence
	\begin{equation*}
		\begin{tikzcd}
			X \arrow[r] & Y \arrow[r] & Z
		\end{tikzcd}
	\end{equation*}
	\noindent is said to be \bld{exact at $Y$}, if 
	\begin{equation*}
		\begin{tikzcd}
			X \arrow[r] & Y \arrow[r] & Z 
		\end{tikzcd} = 0 \qquad \text{and} \qquad
		\begin{tikzcd}
			K \arrow[r] & Y \arrow[r] & C
		\end{tikzcd} = 0,
	\end{equation*}
	\noindent where
	\begin{equation*}
		K \to Y = \ker(Y \to Z)
		\qquad \text{and} \qquad
		Y \to C = \coker(X \to Y).
	\end{equation*}
\end{definition}


\section{The Mitchell Embedding Theorem and its Consequences}

\begin{definition}[Exact Functor \cite{maclane:categories:1978}]
	A functor $F : \mathcal{C} \to \mathcal{D}$ between two abelian categories $\mathcal{C}$ and $\mathcal{D}$ is called \bld{exact}, if $F$ preserves all finite limits and all finite colimits.
\end{definition}

\begin{theorem}[Mitchell Embedding \cite{freyd:abelian_categories:1964}]
	For every small abelian category there is an exact, full and faithful functor into $_{R}\mathsf{Mod}$ for some ring $R$.
	\label{thm:mitchell_embedding}
\end{theorem}

A usefull application of the Mitchell embedding is that one can do proofs of basic homological algebra results in a familiar environment like $_{R}\mathsf{Mod}$ by diagram chasing. However, as \cite[202--208]{maclane:categories:1978} shows, this can also be done without using the Mitchell embedding.

\section{The Eilenberg-Watts Theorem and its Consequences}

\begin{definition}[Bimodule]
	Let $R$ and $S$ be two unital rings. An \bld{$R$-$S$-bimodule} is an abelian group $B$ such that:
	\begin{enumerate}[label = \textup{(}\alph*\textup{)},wide = 0pt]
		\item $B$ is a left $R$-module and a right $S$-module.
		\item We have that $(rb)s = r(bs)$ for all $r \in R$, $s \in S$ and $b \in B$.
	\end{enumerate}
\end{definition}

\begin{theorem}[Eilenberg-Watts]
	Let $R$ and $S$ be unital rings and $B$ a $R$-$S$-bimodule. Then the tensor product functor
	\begin{equation*}
		(-)\otimes_R B : \mathsf{Mod}_R \to \mathsf{Mod}_S
	\end{equation*}
	\noindent is right exact and preserves small coproducts. Conversely, if $F : \mathsf{Mod}_R \to \mathsf{Mod}_S$ is right-exact and preserves small coproducts, then it is naturally isomorphic to tensoring with a bimodule.
\end{theorem}

\printbibliography
\end{document}
