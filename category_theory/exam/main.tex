%%%%%%%%%%%%%%%%%%%%%%%%%%%%%%%%%%%%%%%%%%%%%%%%%%%%%%%%%%%%%%%%%%%%%%%%%%
%Author:																 %
%-------																 %
%Yannis Baehni at University of Zurich									 %
%baehni.yannis@uzh.ch													 %
%																		 %
%Version log:															 %
%------------															 %
%06/02/16 . Basic structure												 %
%04/08/16 . Layout changes including section, contents, abstract.		 %
%%%%%%%%%%%%%%%%%%%%%%%%%%%%%%%%%%%%%%%%%%%%%%%%%%%%%%%%%%%%%%%%%%%%%%%%%%

%Page Setup
\documentclass[
	12pt, 
	oneside, 
	a4paper,
	reqno,
	final
]{amsbook}

\usepackage[
	left = 3cm, 
	right = 3cm, 
	top = 3cm, 
	bottom = 3cm
]{geometry}

%Headers and footers
\usepackage{fancyhdr}
	\pagestyle{fancy}
	%Clear fields
	\fancyhf{}
	%Header right
	\fancyhead[R]{
		\footnotesize
		Yannis B\"{a}hni\\
		\href{mailto:yannis.baehni@uzh.ch}{yannis.baehni@uzh.ch}
	}
	%Header left
	\fancyhead[L]{
		\footnotesize
		401-3001-61L Algebraic Topology I\\
		Autumn Semester 2017
	}
	%Page numbering in footer
	\fancyfoot[C]{\thepage}
	%Separation line header and footer
	\renewcommand{\headrulewidth}{0.4pt}
	%\renewcommand{\footrulewidth}{0.4pt}
	
	\setlength{\headheight}{19pt} 

%Title
\usepackage[foot]{amsaddr}
\usepackage{newtxtext}
\usepackage[subscriptcorrection,nofontinfo,mtpcal,mtphrb]{mtpro2}
\usepackage{mathtools}
\usepackage{bm}
\usepackage{xspace}
\usepackage[all]{xy}
\usepackage{tikz-cd}
\makeatletter
\def\@textbottom{\vskip \z@ \@plus 1pt}
\let\@texttop\relax
\usepackage{etoolbox}
\patchcmd{\abstract}{\scshape\abstractname}{\textbf{\abstractname}}{}{}
\usepackage{chngcntr}
\counterwithout{figure}{chapter}
%Section, subsection and subsubsection font
%------------------------------------------
	\renewcommand{\@secnumfont}{\bfseries}
	\renewcommand\section{\@startsection{section}{1}%
  	\z@{.7\linespacing\@plus\linespacing}{.5\linespacing}%
  	{\normalfont\bfseries\boldmath\centering}}
	\renewcommand\subsection{\@startsection{subsection}{2}%
    	\z@{.5\linespacing\@plus.7\linespacing}{-.5em}%
    	{\normalfont\bfseries\boldmath}}%
	\renewcommand\subsubsection{\@startsection{subsubsection}{3}%
    	\z@{.5\linespacing\@plus.7\linespacing}{-.5em}%
    	{\normalfont\bfseries\boldmath}}%

		\renewenvironment{proof}{\textit{Proof}.}{\hfill\qedsymbol}

%ToC
%---
\makeatletter
\setcounter{tocdepth}{3}
% Add bold to \chapter titles in ToC and remove . after numbers
\renewcommand{\tocchapter}[3]{%
  	\indentlabel{\@ifnotempty{#2}{\bfseries\ignorespaces#1 #2: }}\bfseries#3}
\renewcommand{\tocappendix}[3]{%
  	\indentlabel{\@ifnotempty{#2}{\bfseries\ignorespaces#1 #2: }}\bfseries#3}
% Remove . after numbers in \section and \subsection
\renewcommand{\tocsection}[3]{%
  	\indentlabel{\@ifnotempty{#2}{\ignorespaces#1 #2\quad}}#3}
\renewcommand{\tocsubsection}[3]{%
  	\indentlabel{\@ifnotempty{#2}{\ignorespaces#1 #2\quad}}#3}
\let\tocsubsubsection\tocsubsection% Update for \subsubsection
%...
\newcommand\@dotsep{4.5}
\def\@tocline#1#2#3#4#5#6#7{\relax
  \ifnum #1>\c@tocdepth % then omit
  \else
    \par \addpenalty\@secpenalty\addvspace{#2}%
    \begingroup \hyphenpenalty\@M
    \@ifempty{#4}{%
      \@tempdima\csname r@tocindent\number#1\endcsname\relax
    }{%
      \@tempdima#4\relax
    }%
    \parindent\z@ \leftskip#3\relax \advance\leftskip\@tempdima\relax
    \rightskip\@pnumwidth plus1em \parfillskip-\@pnumwidth
    #5\leavevmode\hskip-\@tempdima{#6}\nobreak
    \leaders\hbox{$\m@th\mkern \@dotsep mu\hbox{.}\mkern \@dotsep mu$}\hfill
    \nobreak
    \hbox to\@pnumwidth{\@tocpagenum{\ifnum#1=0\bfseries\fi#7}}\par% <-- \bfseries for \chapter page
    \nobreak
    \endgroup
  \fi}
\AtBeginDocument{%
\expandafter\renewcommand\csname r@tocindent0\endcsname{0pt}
}
\def\l@subsection{\@tocline{2}{0pt}{2.5pc}{5pc}{}}
\def\l@subsubsection{\@tocline{2}{0pt}{4.5pc}{5pc}{}}
\makeatother

\advance\footskip0.4cm
\textheight=54pc    %a4paper
\textheight=50.5pc %letterpaper
\advance\textheight-0.4cm
\calclayout

%Font settings
%\usepackage{anyfontsize}
%Footnote settings
\usepackage{footmisc}
%	\renewcommand*{\thefootnote}{\fnsymbol{footnote}}
\usepackage{commath}
%Further math environments
%Further math fonts (loads amsfonts implicitely)
%Redefinition of \text
%\usepackage{amstext}
\usepackage{upref}
%Graphics
%\usepackage{graphicx}
%\usepackage{caption}
%\usepackage{subcaption}
%Frames
\usepackage{mdframed}
\allowdisplaybreaks
%\usepackage{interval}
\newcommand{\toup}{%
  \mathrel{\nonscript\mkern-1.2mu\mkern1.2mu{\uparrow}}%
}
\newcommand{\todown}{%
  \mathrel{\nonscript\mkern-1.2mu\mkern1.2mu{\downarrow}}%
}
\AtBeginDocument{\renewcommand*\d{\mathop{}\!\mathrm{d}}}
\renewcommand{\Re}{\operatorname{Re}}
\renewcommand{\Im}{\operatorname{Im}}
\DeclareMathOperator\Log{Log}
\DeclareMathOperator\Arg{Arg}
\DeclareMathOperator\id{id}
\DeclareMathOperator\sech{sech}
\DeclareMathOperator\Aut{Aut}
\DeclareMathOperator\h{h}
\DeclareMathOperator\sgn{sgn}
\DeclareMathOperator\arctanh{arctanh}
\DeclareMathOperator\supp{supp}
\DeclareMathOperator\ob{ob}
\DeclareMathOperator\mor{mor}
\DeclareMathOperator\M{M}
\DeclareMathOperator\dom{dom}
\DeclareMathOperator\cod{cod}
\DeclareMathOperator\im{im}
\DeclareMathOperator\Ab{Ab}
\DeclareMathOperator\coker{coker}
%\usepackage{hhline}
%\usepackage{booktabs} 
%\usepackage{array}
%\usepackage{xfrac} 
%\everymath{\displaystyle}
%Enumerate
\usepackage{tikz}
%\usepackgae{graphicx}
\usepackage{subcaption}
\usepackage{enumitem} 
%\renewcommand{\labelitemi}{$\bullet$}
%\renewcommand{\labelitemii}{$\ast$}
%\renewcommand{\labelitemiii}{$\cdot$}
%\renewcommand{\labelitemiv}{$\circ$}
%Colors
%\usepackage{color}
%\usepackage[cmtip, all]{xy}
%Main style theorem environment
\newtheoremstyle{main} 		             	 		%Stylename
  	{}	                                     		%Space above
  	{}	                                    		%Space below
  	{\itshape}			                     		%Body font
  	{}        	                             		%Indent
  	{\bfseries\boldmath}   	                         		%Head font
  	{.}            	                        		%Head punctuation
  	{ }           	                         		%Head space 
  	{\thmname{#1}\thmnumber{ #2}\thmnote{ (#3)}}	%Head specification
\theoremstyle{main}
\newtheorem{definition}{Definition}[chapter]
\newtheorem{proposition}{Proposition}[chapter]
\newtheorem{corollary}{Corollary}[chapter]
\newtheorem{theorem}{Theorem}[chapter]
\newtheorem{lemma}{Lemma}[chapter]
\newtheoremstyle{nonit} 		             	 		%Stylename
  	{}	                                     		%Space above
  	{}	                                    		%Space below
  	{}			                     		%Body font
  	{}        	                             		%Indent
  	{\bfseries\boldmath}   	                   		%Head font
  	{.}            	                        		%Head punctuation
  	{ }           	                         		%Head space 
  	{\thmname{#1}\thmnumber{ #2}\thmnote{ (#3)}}	%Head specification
\theoremstyle{nonit}
\newtheorem{remark}{Remark}[chapter]
\newtheorem{examples}{Examples}[chapter]
\newtheorem{example}{Example}[chapter]
\newtheorem{problem}{Problem}[chapter]
\newtheoremstyle{ex} 		             	 		%Stylename
  	{}	                                     		%Space above
  	{}	                                    		%Space below
  	{\small}			                     		%Body font
  	{}        	                             		%Indent
  	{\bfseries\boldmath}   	                         		%Head font
  	{.}            	                        		%Head punctuation
  	{ }           	                         		%Head space 
  	{\thmname{#1}\thmnumber{ #2}\thmnote{ (#3)}}	%Head specification
\theoremstyle{ex}
\newtheorem{exercise}{Exercise}[chapter]
%German non-ASCII-Characters
%Graphics-Tool
%\usepackage{tikz}
%\usepackage{tikzscale}
%\usepackage{bbm}
%\usepackage{bera}
%Listing-Setup
%Bibliographie
\usepackage[backend=bibtex, style=alphabetic]{biblatex}
%\usepackage[babel, german = swiss]{csquotes}
\bibliography{bibliography}
%PDF-Linking
%\usepackage[hyphens]{url}
\usepackage[bookmarksopen=true,bookmarksnumbered=true]{hyperref}
%\PassOptionsToPackage{hyphens}{url}\usepackage{hyperref}
\urlstyle{rm}
\hypersetup{
  colorlinks   = true, %Colours links instead of ugly boxes
  urlcolor     = blue, %Colour for external hyperlinks
  linkcolor    = blue, %Colour of internal links
  citecolor    = blue %Colour of citations
}
\newcommand{\bld}[1]{\boldmath\textit{\textbf{#1}}\unboldmath}
\newcommand{\eqclass}[1]{\sbr[0]{#1}}
\newcommand{\cat}[1]{\mathsf{#1}}
\newcommand{\Sbb}{\mathbb{S}}
\newcommand{\Zbb}{\mathbb{Z}}
\newcommand{\Nbb}{\mathbb{N}}
\newcommand{\Rbb}{\mathbb{R}}
\newcommand{\Hbb}{\mathbb{H}}
\newcommand{\Cbb}{\mathbb{C}}
\newcommand{\Tcal}{\mathcal{T}}
\newcommand{\SLrm}{\mathrm{SL}}
\newcommand{\PSLrm}{\mathrm{PSL}}
\newcommand{\SLrmstar}{\mathrm{S^*L}}
\newcommand{\PSLrmstar}{\mathrm{PS^*L}}
\newcommand{\GLrm}{\mathrm{GL}}
\newcommand{\Mrm}{\mathrm{M}}
\newcommand{\Isom}{\mathrm{Isom}}
\newcommand{\Mob}{\mathrm{M\ddot{o}b}}
\newcommand{\Ebb}{\mathbb{E}}
\newcommand{\Cscr}{\mathscr{C}}
\newcommand{\pwrm}{\mathrm{pw}}
\newcommand{\clos}[1]{\overline{#1}}
\newcommand{\Hcal}{\mathcal{H}}
\newcommand{\Hbcal}{\bm{\mathcal{H}}}
\newcommand{\Ucal}{\mathcal{U}}
\newcommand{\Ubcal}{\bm{\mathcal{U}}}
\renewcommand{\det}{\mathrm{det}}
\newcommand{\ab}{\mathrm{ab}}


\title{Additive and Abelian Categories}
\author{Yannis B\"{a}hni}
\address[Yannis B\"{a}hni]{University of Zurich, R\"{a}mistrasse 71, 8006 Zurich}
\email[Yannis B\"{a}hni]{\href{mailto:yannis.baehni@uzh.ch}{\nolinkurl{yannis.baehni@uzh.ch}}}

\begin{document}

\begin{abstract}
	We define preadditive, additive and abelian categories, where the latter is the natural generalization of $\mathsf{AbGrp}$ to study the basic results of homological algebra in. Moreover, we state and comment on two foundational results in the theory of abelian categories, namely the \emph{Mitchell Embedding Theorem} and the \emph{Eilenberg-Watts Theorem}, which roughly speaking, connect the category of modules with abelian categories.
\end{abstract}

\maketitle

\tableofcontents

\section{Introduction}
Consider the following exact commutative diagram in $\mathsf{AbGrp}$:
\begin{equation*}
	\begin{tikzcd}
		& \bullet \arrow[r]\arrow[d,"f"] & \bullet \arrow[r]\arrow[d,"g"] & \bullet \arrow[r]\arrow[d,"h"] & 0\\
		0 \arrow[r] & \bullet \arrow[r] & \bullet \arrow[r] & \bullet
	\end{tikzcd}
\end{equation*}
Those who are familiar with algebraic topology recognise it as the setting of the \emph{snake lemma}. This basic result in \emph{homological algebra} yields the existence of a morphism of groups $\delta \in \mathsf{AbGrp}(\ker h,\coker f)$ such that the sequence
\begin{equation*}
	\begin{tikzcd}
			\ker f \arrow[r] & \ker g \arrow[r] & \ker h \arrow[r,"\delta"] & \coker f \arrow[r] & \coker g \arrow[r] & \coker h
	 \end{tikzcd}
\end{equation*}
\noindent is exact. The basic proof technique used to establish this result is called \emph{diagram chasing}. It turns out, that abelian categories are the right generalization for this type of proof.

\section{Preadditive Catgeories}
Let $G,H \in \ob(\mathsf{AbGrp})$ and $\varphi,\psi \in \mathsf{AbGrp}(G,H)$. Define $\varphi + \psi$ pointwise. Since $H$ is abelian, it follows that $\varphi + \psi \in \mathsf{AbGrp}(G,H)$. Moreover, it is easy to check, that with this operation defined above, $\mathsf{AbGrp}(G,H)$ is an abelian group and 
\begin{equation*}
	\circ : \mathsf{AbGrp}(H,K) \times \mathsf{AbGrp}(G,H) \to \mathsf{AbGrp}(G,K)
\end{equation*}
\noindent is bilinear for each $K \in \ob(\mathsf{AbGrp})$. This motivates the following definition. 

\begin{definition}[Preadditive Category \cite{maclane:categories:1978}]
	A \bld{preadditive category} is a locally small category $\mathcal{C}$ in which all hom-sets $\mathcal{C}(X,Y)$ can be equipped with the structure of an abelian group and composition is bilinear, i.e. for all mophisms $f,f' : X \to Y$ and $g,g' : Y \to Z$ in $\mathcal{C}$ we have that
	\begin{equation}
		(g + g') \circ (f + f') = g \circ f + g \circ f' + g' \circ f + g' \circ f'.
	\end{equation}
\end{definition}

\section{Additive Categories}
Let us again consider $\mathsf{AbGrp}$. As in $\mathsf{Grp}$, the trivial group $0$ is both an initial and a terminal object. Unlike in $\mathsf{Grp}$, we have that $G \coprod H \cong G \prod H$ for all $G,H \in \ob(\mathsf{AbGrp})$. In a somewhat weaker sense, we will generalize this. Define $\iota_1 : G \to G \prod H$ and $\iota_2 : H \to G \prod H$ by
\begin{equation*}
	\iota_1(g) := (g,0) \qquad \text{and} \qquad \iota_2(h) := (0,h),
\end{equation*}
\noindent respectively. Then it is easy to verify that 
\begin{equation*}
	\pi_1\circ\iota_1 = \id_G, \quad \pi_2\circ\iota_2 = \id_H \quad \text{and} \quad \iota_1 \circ \pi_1 + \iota_2 \circ \pi_2 = \id_{G \prod H}
\end{equation*}
\noindent holds. Those observations motivate the following definitions.

\begin{definition}[Null Object \cite{maclane:categories:1978}]
	Let $\mathcal{C}$ be a category. A \bld{null object in $\mathcal{C}$} is a an object of $\mathcal{C}$ which is both initial and terminal.
\end{definition}

\begin{definition}[Biproduct Diagram \cite{maclane:categories:1978}]
	Let $\mathcal{C}$ be a preadditive category and $X,Y \in \ob(\mathcal{C})$. A \bld{biproduct diagram for $X$ and $Y$} is a diagram
	\begin{equation*}
		\begin{tikzcd}
			X \arrow[r, shift right, "\iota_1"'] & Z \arrow[r,shift left, "\pi_2"]\arrow[l,shift right,"\pi_1"'] & Y \arrow[l,shift left,"\iota_2"] 
		\end{tikzcd}
	\end{equation*}
	\noindent such that
	\begin{equation*}
		\pi_1\circ\iota_1 = \id_X, \quad \pi_2\circ\iota_2 = \id_Y \quad \text{and} \quad \iota_1 \circ \pi_1 + \iota_2 \circ \pi_2 = \id_Z
	\end{equation*}
	\noindent holds.
\end{definition}

\begin{definition}[Additive Category \cite{maclane:categories:1978}]
	An \bld{additive category} is a preadditive category which has a null object and a biproduct for each pair of its objects.	
\end{definition}

\section{Abelian Categories}

\begin{definition}[Zero Arrow \cite{maclane:categories:1978}]
	Let $\mathcal{C}$ be a category with a null object $0$. For $X,Y \in \ob(\mathcal{C})$, the unique composition $X \to 0 \to Y$ is called the \bld{zero arrow from $X$ to $Y$}, denoted by $0 : X \to Y$.
\end{definition}

\begin{definition}[Kernel and Cokernel \cite{maclane:categories:1978}]
	Let $\mathcal{C}$ be a category with a null object $0$. A \bld{kernel of a morphism $f : X \to Y$} is defined to be an equalizer of
	\begin{equation*}
		\begin{tikzcd}
			X \arrow[r, shift left, "f"]\arrow[r,shift right,"0"'] & Y.
		\end{tikzcd}
	\end{equation*}
	Dually, a \bld{cokernel of a morphism $f : X \to Y$} is a coequalizer of the above diagram. 
\end{definition}

\begin{lemma}
	In $\mathsf{Grp}$, every monic is a kernel and every epic is a cokernel.
\end{lemma}

\begin{proof}
	Let $m : G \to H$ be a monic in $\mathsf{Grp}$. Consider the fork
	\begin{equation*}
		\begin{tikzcd}
			G \arrow[r,"m"] & H \arrow[r,shift left,"\pi"]\arrow[r,shift right,"0"'] & H/m(G).
		\end{tikzcd}
	\end{equation*}
	\noindent Then one can check that this is in fact a universal fork. Similarly, one can check that
	\begin{equation*}
		\begin{tikzcd}
			\ker e \arrow[r,shift left,"\iota"]\arrow[r,shift right,"0"'] & G \arrow[r,"e"] & H
		\end{tikzcd}
	\end{equation*}
	\noindent is a universal cofork for any epic $e : G \to H$ in $\mathsf{Grp}$.
\end{proof}

\begin{definition}[Abelian Category \cite{maclane:categories:1978}]
	An \bld{abelian category} is an additive category satisfying the following additional conditions:
	\begin{enumerate}[label = \textup{(}\alph*\textup{)}, wide = 0pt]
		\item Every morphism admits a kernel.
		\item Every morphism admits a cokernel.
		\item Every monic is a kernel.
		\item Every epic is a cokernel.
	\end{enumerate}
\end{definition}

\begin{examples}
	$\mathsf{AbGrp}$, $\mathsf{Vect}_K$, $_{R}\mathsf{Mod}$ and $\mathsf{Mod}_R$.	
\end{examples}

\section{The Mitchell Embedding Theorem and its Consequences}

\begin{definition}[Exact Functor \cite{maclane:categories:1978}]
	A functor $F : \mathcal{C} \to \mathcal{D}$ between two abelian categories $\mathcal{C}$ and $\mathcal{D}$ is called \bld{exact}, if $F$ preserves all finite limits and all finite colimits.
\end{definition}

\begin{theorem}[Mitchell Embedding \cite{freyd:abelian_categories:1964}]
	For every small abelian category there is an exact, full and faithful functor into $_{R}\mathsf{Mod}$ for some ring $R$.
	\label{thm:mitchell_embedding}
\end{theorem}

A usefull application of the Mitchell embedding is that one can do proofs of basic homological algebra results in a familiar environment like $_{R}\mathsf{Mod}$ by diagram chasing. However, as \cite[202--208]{maclane:categories:1978} shows, this can also be done without using the Mitchell embedding.

\section{The Eilenberg-Watts Theorem and its Consequences}


\printbibliography
\end{document}
