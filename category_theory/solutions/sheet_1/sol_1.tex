%%%%%%%%%%%%%%%%%%%%%%%%%%%%%%%%%%%%%%%%%%%%%%%%%%%%%%%%%%%%%%%%%%%%%%%%%%
%Author:																 %
%-------																 %
%Yannis Baehni at University of Zurich									 %
%baehni.yannis@uzh.ch													 %
%																		 %
%Version log:															 %
%------------															 %
%06/02/16 . Basic structure												 %
%04/08/16 . Layout changes including section, contents, abstract.		 %
%%%%%%%%%%%%%%%%%%%%%%%%%%%%%%%%%%%%%%%%%%%%%%%%%%%%%%%%%%%%%%%%%%%%%%%%%%

%Page Setup
\documentclass[
	12pt, 
	oneside, 
	a4paper,
	reqno,
	final
]{amsbook}

\usepackage[
	left = 3cm, 
	right = 3cm, 
	top = 3cm, 
	bottom = 3cm
]{geometry}

%Headers and footers
\usepackage{fancyhdr}
	\pagestyle{fancy}
	%Clear fields
	\fancyhf{}
	%Header right
	\fancyhead[R]{
		\footnotesize
		Yannis B\"{a}hni\\
		\href{mailto:yannis.baehni@uzh.ch}{yannis.baehni@uzh.ch}
	}
	%Header left
	\fancyhead[L]{
		\footnotesize
		401-3001-61L Algebraic Topology I\\
		Autumn Semester 2017
	}
	%Page numbering in footer
	\fancyfoot[C]{\thepage}
	%Separation line header and footer
	\renewcommand{\headrulewidth}{0.4pt}
	%\renewcommand{\footrulewidth}{0.4pt}
	
	\setlength{\headheight}{19pt} 

%Title
\usepackage[foot]{amsaddr}
\usepackage{newtxtext}
\usepackage[subscriptcorrection,nofontinfo,mtpcal,mtphrb]{mtpro2}
\usepackage{mathtools}
\usepackage{bm}
\usepackage{xspace}
\usepackage[all]{xy}
\usepackage{tikz-cd}
\makeatletter
\def\@textbottom{\vskip \z@ \@plus 1pt}
\let\@texttop\relax
\usepackage{etoolbox}
\patchcmd{\abstract}{\scshape\abstractname}{\textbf{\abstractname}}{}{}
\usepackage{chngcntr}
\counterwithout{figure}{chapter}
%Section, subsection and subsubsection font
%------------------------------------------
	\renewcommand{\@secnumfont}{\bfseries}
	\renewcommand\section{\@startsection{section}{1}%
  	\z@{.7\linespacing\@plus\linespacing}{.5\linespacing}%
  	{\normalfont\bfseries\boldmath\centering}}
	\renewcommand\subsection{\@startsection{subsection}{2}%
    	\z@{.5\linespacing\@plus.7\linespacing}{-.5em}%
    	{\normalfont\bfseries\boldmath}}%
	\renewcommand\subsubsection{\@startsection{subsubsection}{3}%
    	\z@{.5\linespacing\@plus.7\linespacing}{-.5em}%
    	{\normalfont\bfseries\boldmath}}%

		\renewenvironment{proof}{\textit{Proof}.}{\hfill\qedsymbol}

%ToC
%---
\makeatletter
\setcounter{tocdepth}{3}
% Add bold to \chapter titles in ToC and remove . after numbers
\renewcommand{\tocchapter}[3]{%
  	\indentlabel{\@ifnotempty{#2}{\bfseries\ignorespaces#1 #2: }}\bfseries#3}
\renewcommand{\tocappendix}[3]{%
  	\indentlabel{\@ifnotempty{#2}{\bfseries\ignorespaces#1 #2: }}\bfseries#3}
% Remove . after numbers in \section and \subsection
\renewcommand{\tocsection}[3]{%
  	\indentlabel{\@ifnotempty{#2}{\ignorespaces#1 #2\quad}}#3}
\renewcommand{\tocsubsection}[3]{%
  	\indentlabel{\@ifnotempty{#2}{\ignorespaces#1 #2\quad}}#3}
\let\tocsubsubsection\tocsubsection% Update for \subsubsection
%...
\newcommand\@dotsep{4.5}
\def\@tocline#1#2#3#4#5#6#7{\relax
  \ifnum #1>\c@tocdepth % then omit
  \else
    \par \addpenalty\@secpenalty\addvspace{#2}%
    \begingroup \hyphenpenalty\@M
    \@ifempty{#4}{%
      \@tempdima\csname r@tocindent\number#1\endcsname\relax
    }{%
      \@tempdima#4\relax
    }%
    \parindent\z@ \leftskip#3\relax \advance\leftskip\@tempdima\relax
    \rightskip\@pnumwidth plus1em \parfillskip-\@pnumwidth
    #5\leavevmode\hskip-\@tempdima{#6}\nobreak
    \leaders\hbox{$\m@th\mkern \@dotsep mu\hbox{.}\mkern \@dotsep mu$}\hfill
    \nobreak
    \hbox to\@pnumwidth{\@tocpagenum{\ifnum#1=0\bfseries\fi#7}}\par% <-- \bfseries for \chapter page
    \nobreak
    \endgroup
  \fi}
\AtBeginDocument{%
\expandafter\renewcommand\csname r@tocindent0\endcsname{0pt}
}
\def\l@subsection{\@tocline{2}{0pt}{2.5pc}{5pc}{}}
\def\l@subsubsection{\@tocline{2}{0pt}{4.5pc}{5pc}{}}
\makeatother

\advance\footskip0.4cm
\textheight=54pc    %a4paper
\textheight=50.5pc %letterpaper
\advance\textheight-0.4cm
\calclayout

%Font settings
%\usepackage{anyfontsize}
%Footnote settings
\usepackage{footmisc}
%	\renewcommand*{\thefootnote}{\fnsymbol{footnote}}
\usepackage{commath}
%Further math environments
%Further math fonts (loads amsfonts implicitely)
%Redefinition of \text
%\usepackage{amstext}
\usepackage{upref}
%Graphics
%\usepackage{graphicx}
%\usepackage{caption}
%\usepackage{subcaption}
%Frames
\usepackage{mdframed}
\allowdisplaybreaks
%\usepackage{interval}
\newcommand{\toup}{%
  \mathrel{\nonscript\mkern-1.2mu\mkern1.2mu{\uparrow}}%
}
\newcommand{\todown}{%
  \mathrel{\nonscript\mkern-1.2mu\mkern1.2mu{\downarrow}}%
}
\AtBeginDocument{\renewcommand*\d{\mathop{}\!\mathrm{d}}}
\renewcommand{\Re}{\operatorname{Re}}
\renewcommand{\Im}{\operatorname{Im}}
\DeclareMathOperator\Log{Log}
\DeclareMathOperator\Arg{Arg}
\DeclareMathOperator\id{id}
\DeclareMathOperator\sech{sech}
\DeclareMathOperator\Aut{Aut}
\DeclareMathOperator\h{h}
\DeclareMathOperator\sgn{sgn}
\DeclareMathOperator\arctanh{arctanh}
\DeclareMathOperator\supp{supp}
\DeclareMathOperator\ob{ob}
\DeclareMathOperator\mor{mor}
\DeclareMathOperator\M{M}
\DeclareMathOperator\dom{dom}
\DeclareMathOperator\cod{cod}
\DeclareMathOperator\im{im}
\DeclareMathOperator\Ab{Ab}
\DeclareMathOperator\coker{coker}
%\usepackage{hhline}
%\usepackage{booktabs} 
%\usepackage{array}
%\usepackage{xfrac} 
%\everymath{\displaystyle}
%Enumerate
\usepackage{tikz}
%\usepackgae{graphicx}
\usepackage{subcaption}
\usepackage{enumitem} 
%\renewcommand{\labelitemi}{$\bullet$}
%\renewcommand{\labelitemii}{$\ast$}
%\renewcommand{\labelitemiii}{$\cdot$}
%\renewcommand{\labelitemiv}{$\circ$}
%Colors
%\usepackage{color}
%\usepackage[cmtip, all]{xy}
%Main style theorem environment
\newtheoremstyle{main} 		             	 		%Stylename
  	{}	                                     		%Space above
  	{}	                                    		%Space below
  	{\itshape}			                     		%Body font
  	{}        	                             		%Indent
  	{\bfseries\boldmath}   	                         		%Head font
  	{.}            	                        		%Head punctuation
  	{ }           	                         		%Head space 
  	{\thmname{#1}\thmnumber{ #2}\thmnote{ (#3)}}	%Head specification
\theoremstyle{main}
\newtheorem{definition}{Definition}[chapter]
\newtheorem{proposition}{Proposition}[chapter]
\newtheorem{corollary}{Corollary}[chapter]
\newtheorem{theorem}{Theorem}[chapter]
\newtheorem{lemma}{Lemma}[chapter]
\newtheoremstyle{nonit} 		             	 		%Stylename
  	{}	                                     		%Space above
  	{}	                                    		%Space below
  	{}			                     		%Body font
  	{}        	                             		%Indent
  	{\bfseries\boldmath}   	                   		%Head font
  	{.}            	                        		%Head punctuation
  	{ }           	                         		%Head space 
  	{\thmname{#1}\thmnumber{ #2}\thmnote{ (#3)}}	%Head specification
\theoremstyle{nonit}
\newtheorem{remark}{Remark}[chapter]
\newtheorem{examples}{Examples}[chapter]
\newtheorem{example}{Example}[chapter]
\newtheorem{problem}{Problem}[chapter]
\newtheoremstyle{ex} 		             	 		%Stylename
  	{}	                                     		%Space above
  	{}	                                    		%Space below
  	{\small}			                     		%Body font
  	{}        	                             		%Indent
  	{\bfseries\boldmath}   	                         		%Head font
  	{.}            	                        		%Head punctuation
  	{ }           	                         		%Head space 
  	{\thmname{#1}\thmnumber{ #2}\thmnote{ (#3)}}	%Head specification
\theoremstyle{ex}
\newtheorem{exercise}{Exercise}[chapter]
%German non-ASCII-Characters
%Graphics-Tool
%\usepackage{tikz}
%\usepackage{tikzscale}
%\usepackage{bbm}
%\usepackage{bera}
%Listing-Setup
%Bibliographie
\usepackage[backend=bibtex, style=alphabetic]{biblatex}
%\usepackage[babel, german = swiss]{csquotes}
\bibliography{bibliography}
%PDF-Linking
%\usepackage[hyphens]{url}
\usepackage[bookmarksopen=true,bookmarksnumbered=true]{hyperref}
%\PassOptionsToPackage{hyphens}{url}\usepackage{hyperref}
\urlstyle{rm}
\hypersetup{
  colorlinks   = true, %Colours links instead of ugly boxes
  urlcolor     = blue, %Colour for external hyperlinks
  linkcolor    = blue, %Colour of internal links
  citecolor    = blue %Colour of citations
}
\newcommand{\bld}[1]{\boldmath\textit{\textbf{#1}}\unboldmath}
\newcommand{\eqclass}[1]{\sbr[0]{#1}}
\newcommand{\cat}[1]{\mathsf{#1}}
\newcommand{\Sbb}{\mathbb{S}}
\newcommand{\Zbb}{\mathbb{Z}}
\newcommand{\Nbb}{\mathbb{N}}
\newcommand{\Rbb}{\mathbb{R}}
\newcommand{\Hbb}{\mathbb{H}}
\newcommand{\Cbb}{\mathbb{C}}
\newcommand{\Tcal}{\mathcal{T}}
\newcommand{\SLrm}{\mathrm{SL}}
\newcommand{\PSLrm}{\mathrm{PSL}}
\newcommand{\SLrmstar}{\mathrm{S^*L}}
\newcommand{\PSLrmstar}{\mathrm{PS^*L}}
\newcommand{\GLrm}{\mathrm{GL}}
\newcommand{\Mrm}{\mathrm{M}}
\newcommand{\Isom}{\mathrm{Isom}}
\newcommand{\Mob}{\mathrm{M\ddot{o}b}}
\newcommand{\Ebb}{\mathbb{E}}
\newcommand{\Cscr}{\mathscr{C}}
\newcommand{\pwrm}{\mathrm{pw}}
\newcommand{\clos}[1]{\overline{#1}}
\newcommand{\Hcal}{\mathcal{H}}
\newcommand{\Hbcal}{\bm{\mathcal{H}}}
\newcommand{\Ucal}{\mathcal{U}}
\newcommand{\Ubcal}{\bm{\mathcal{U}}}
\renewcommand{\det}{\mathrm{det}}
\newcommand{\ab}{\mathrm{ab}}


\title{Solutions Sheet 1}
\author{Yannis B\"{a}hni}
\address[Yannis B\"{a}hni]{University of Zurich, R\"{a}mistrasse 71, 8006 Zurich}
\email[Yannis B\"{a}hni]{\href{mailto:yannis.baehni@uzh.ch}{yannis.baehni@uzh.ch}}

\begin{document}
\maketitle
\thispagestyle{fancy}

\begin{enumerate}[label = \textbf{Exercise \arabic*.},wide = 0pt, itemsep=1.5ex]
	\item \label{exercise_1}~
		\begin{enumerate}[label = \textup{(}\alph*\textup{)}]
			\item The pair $(D(X),i)$ has the universal property
				\begin{figure}[h!tb]
					\begin{displaymath}
    					\xymatrix{X \ar[r]^i\ar[dr]_{\forall \text{ functions } f} & D(X) \ar[d]^{\exists ! \text{ continuous } \wbar{f}}\\
						 & \forall (Y,\Tcal_Y).}
					\end{displaymath}
				\end{figure}\\
				Assume, that there is another pair $(D'(X),i')$ with this property. Thus we get the two commuting diagrams

				\begin{figure}[h!tb]
					\begin{subfigure}{0.3\textwidth}
						\begin{displaymath}
    						\xymatrix{X \ar[r]^i\ar[dr]_{i'} & D(X) \ar[d]^{\wbar{i'}}\\
						 	& D'(X),}
						\end{displaymath}
					\end{subfigure}
					\quad
					\begin{subfigure}{0.3\textwidth}
						\begin{displaymath}
							\xymatrix{X \ar[r]^{i'}\ar[dr]_{i} & D'(X) \ar[d]^{\wbar{i}}\\
						 	& D(X).}
						\end{displaymath}
					\end{subfigure}
				\end{figure}
				Putting them together yields
				\begin{figure}[h!tb]
					\begin{subfigure}{0.3\textwidth}
						\begin{displaymath}
							\xymatrix{& D(X) \ar[d]^{\wbar{i'}}\\
							X \ar[ur]^i \ar[r]^{i'}\ar[dr]_{i} & D'(X) \ar[d]^{\wbar{i}}\\
						 		& D(X),}
						\end{displaymath}
					\end{subfigure}
					\quad
					\begin{subfigure}{0.3\textwidth}
						\begin{displaymath}
							\xymatrix{& D'(X) \ar[d]^{\wbar{i}}\\
							X \ar[ur]^{i'} \ar[r]^{i}\ar[dr]_{i'} & D(X) \ar[d]^{\wbar{i'}}\\
						 		& D'(X).}
						\end{displaymath}
					\end{subfigure}
				\end{figure}\\
				Hence
				\begin{equation*}
					(\wbar{i} \circ \wbar{i'}) \circ i = i \qquad \text{and} \qquad (\wbar{i'} \circ \wbar{i}) \circ i' = i'.
				\end{equation*}
				Since also $\id_{D(X)} \circ i = i$ and $\id_{D'(X)} \circ i' = i'$, uniqueness implies that 
				\begin{equation*}
					\wbar{i} \circ \wbar{i'} = \id_{D(X)} \qquad \text{and} \qquad \wbar{i'} \circ \wbar{i} = \id_{D'(X)}.
				\end{equation*}
				Thus $D(X) \cong D'(X)$ uniquely.
			\item Let $I(X) := (X,\cbr[0]{\varnothing,X})$ be the \bld{indiscrete topological space}. Define a mapping $\pi : I(X) \to X$ by $\pi(x) := x$. Then the tuple $(I(X),\pi)$ has the following universal property:
				\begin{figure}[h!tb]
					\begin{displaymath}
						\xymatrix{X & I(X) \ar[l]_\pi \\
						 & \forall (Y,\Tcal_Y). \ar[ul]^{\forall \text{ functions } f}\ar[u]_{\exists ! \text{ continuous } \wbar{f}}}
					\end{displaymath}
				\end{figure}\\
				Now the argumentation is the same as in part (a).
		\end{enumerate}
	\item ~
		\begin{enumerate}[label = \textup{(}\alph*\textup{)}]
			\item First we show that $(\Zbb\sbr[0]{X},X)$ has the claimed property. Let $R$ be a unital ring with $r \in R$. Then there exists a unique homomorphism of rings $\varphi : \Zbb \to R$. Define $f : \Zbb\sbr[0]{X} \to R$ by 
				\begin{equation*}
					f\PARENS{\sum_{\iota = 0}^n a_\iota X^\iota} := \sum_{\iota = 0}^n \varphi(a_\iota)r^\iota.
				\end{equation*}
				Clearly, $f(X) = r$. Also it is easy to check that $f$ is a homomorphism of rings. Assume that $g : \Zbb\sbr[0]{X} \to R$ is a homomorphism of rings such that $g(X) = r$. Then
				\begin{equation*}
					g\PARENS{\sum_{\iota = 0}^n a_\iota X^\iota} = \sum_{\iota = 0}^n g(a_\iota)g(X)^\iota = \sum_{\iota = 0}^n g(a_\iota)r^\iota = \sum_{\iota = 0}^n \varphi(a_\iota)r^\iota = f\PARENS{\sum_{\iota = 0}^n a_\iota X^\iota}
				\end{equation*}
				\noindent by the uniqueness of $\varphi$ ($g$ induces a homomorphism of rings $\Zbb \to R$). Consider the following diagram:
				\begin{figure}[h!tb]
					\begin{displaymath}
						\xymatrixcolsep{2cm}\xymatrix{(\Zbb\sbr[0]{X},X) \ar[r]^{\exists ! f, f(X) = a } & (A,a) \ar[r]^{\exists ! g, g(a) = X} & (\Zbb\sbr[0]{X},X) \ar[r]^{\exists ! f, f(X) = a } & (A,a)}.
					\end{displaymath}	
				\end{figure}\\
				Now $\id_{(\Zbb\sbr[0]{X},X)}(X) = X$ and $g(f(X)) = X$, thus by uniqueness $g \circ f = \id_{(\Zbb\sbr[0]{X},X)}$ and similarly $f \circ g = \id_{(A,a)}$.
			\item
		\end{enumerate}
	\item Existence was shown in the lecture, the so-called \bld{free group}. Uniqueness is shown exactly as in \ref{exercise_1}.
	\item Let $g,\wtilde{g} : Y \to X$ be inverses of $f$. Then we have
		\begin{equation*}
			g = g \circ \id_Y = g \circ (f \circ \wtilde{g}) = (g \circ f) \circ \wtilde{g} = \id_X \circ \wtilde{g} = \wtilde{g}.
		\end{equation*}
		Thus we can unambiguously write $f^{-1} := g$.
	\item That $h \circ g \circ f$ is an isomorphism immediately follows by
		\begin{align*}
			& \PARENS{(g \circ f)^{-1} \circ g \circ (h \circ g)^{-1}} \circ (h \circ g \circ f) = \id_X\\
			& (h \circ g \circ f) \circ \PARENS{(g \circ f)^{-1} \circ g \circ (h \circ g)^{-1}} = \id_W.
		\end{align*}
		Moreover
		\begin{align*}
			&\PARENS{(h \circ g)^{-1} \circ h} \circ g = (h \circ g)^{-1} \circ (h \circ g) = \id_Y\\
			&g \circ \PARENS{f \circ (g \circ f)^{-1}} = (g \circ f) \circ (g \circ f)^{-1} = \id_Z.
		\end{align*}
		\begin{lemma}
			Let $\Ccat$ be a category and $f : X \to Y$. Assume that there exsist $g,\wtilde{g}: Y \to X$ such that $g \circ f = \id_X$ and $f \circ \wtilde{g} = \id_Y$. Then $f$ is an isomorphism with $f^{-1} = g = \wtilde{g}$.
		\end{lemma}

		\begin{proof}
			We have that 
			\begin{equation*}
				g = g \circ \id_Y = g \circ (f \circ \wtilde{g}) = (g \circ f) \circ \wtilde{g} = \id_X \circ \wtilde{g} = \wtilde{g}.
			\end{equation*}
		\end{proof}
		Thus $g$ is invertible. 
		\begin{lemma}
			Let $\Ccat$ be a category and $f : X \to Y$, $g : Y \to Z$ isomorphisms. Then also $g \circ f$ is an isomorphism with $(g \circ f)^{-1} = f^{-1} \circ g^{-1}$.
		\end{lemma}
	\begin{proof}
		We have that 
		\begin{equation*}
			(f^{-1} \circ g^{-1}) \circ (g \circ f) = \id_X \qquad \text{and} \qquad (g \circ f) \circ (f^{-1} \circ g^{-1}) = \id_Z.
		\end{equation*}
		Hence the statement follows by the uniqueness of the inverse.
	\end{proof}
	Therefore also
	\begin{equation*}
		f = (h \circ g)^{-1} \circ (h \circ g \circ f) \qquad \text{and} \qquad h = (h \circ g \circ f) \circ (g \circ f)^{-1}
	\end{equation*}
	\noindent are isomorphisms.

	\item Assume $f : X \to Y$ has the left cancellation property. Let $x,y \in X$ such that $f(x) = f(y)$. Now let $Z := \cbr[0]{x,y}$. Define two functions $c_x,c_y : Z \to X$ by $c_x(z) := x$ and $c_y(z) := y$, respectively. Now
		\begin{equation*}
			f \circ c_x = f(x) = f(y) = f \circ c_y
		\end{equation*}
		\noindent holds by assumption. Thus the left cancellation property implies that $c_x = c_y$, hence $x = y$ and $f$ is injective. Conversly, assume that $f$ is injective. Let $\alpha,\beta : Z \to X$ such that $f \circ \alpha = f \circ \beta$ and $z \in Z$. Then we have that $f(\alpha(z)) = f(\beta(z))$ and thus by injectivity, $\alpha(z) = \beta(z)$. 
\end{enumerate}
\printbibliography
\end{document}
